\documentclass[english]{article}
\usepackage{times}
\usepackage{float}
\usepackage{babel}
\usepackage[top=1.0in, bottom=1.0in, left=1.0in, right=1.0in]{geometry}

\floatstyle{ruled}
\newfloat{program}{thp}{lop}
\floatname{program}{Program}

\begin{document}

\title{MINI Compiler -- Final Report}
\author{Mark Randles\\
		Department of Computer Science\\
		CS409 - Language Design \& Implementation\\
  \texttt{randlem@bgnet.bgsu.edu}}
\date{\today}
\maketitle

\section{Environment}

The environment that was used to produce this compiler was GCC/G++ version 3.3.4
from the GNU organization.  The environment that this compiler was written was
Slackware 10.  The compiler also appears to compile under the BGUNIX environment,
but the executable produced by the outdated version of GCC/G++ on BGUNIX appears
to enter an infinite loop while compiling a MINI program.  The cause of this is
unknown.

\section{Extra Classes}

There were no extra classes defined for this compiler.  However, there were extra
function calls added, and different functionality added to various support classes.
The main classes remain quite virgin.

\section{Problems and Issues}

This compiler is incomplete.  It appears to implement about 85-95% of the MINI
language, implementing everything but forward declarations of procedures.  For most
other syntaxes it appears to produce correct object code.  The testing of this compiler
was such that each individual code generator was tested extensively, and all apparently
passed this authors battery of tests.

There is one reproducible segmentation fault with this compiler currently.  It would
appear that a NULL pointer can be introduced into either Parser::currentOp or
Parser::nextOp.  When this pointer is used it causes a complete failure of the
program, and it's termination.  The most common point of failure is when the
getKey() method of the NULL pointer is being called.  The author believes the
solution to this problem is in the scanner portion of the project.  During the
call to Scanner::nextToken() the scanner needs to see if the token is actually
in the token table, and if not then throw an error.

Also, there seems to be a problem with the control flow of the standard test
program sample.mini, provided by the professor.  When compiled without error
on this compiler, it seems to run erratically in the testing environment.  The
output of the program seems to be off of what should appear.  However, the author
was unable to find a quick-fix without intensely studying the code produced, which
was impossible at the late hour this issue was discovered.

\section{Extensions}

There was one extension, or what this author considers one.  The functionality of
the operator "break" was extended to act like a C/C++ return statement when called
from the scope of a proc structure or the main body of the program.  Here are some examples:
\begin{program}
	\begin{verbatim}
program testbreak;
 proc myproc;
  write 0;
  break;
  write 1;
 endproc;

 call myproc;
 break;
 call myproc;
endprogram;
	\end{verbatim}
	\caption{Listing of program test\_break\_1.mini}
\end{program}
This code will produce a program flow where the main program calls myproc, which
will print a zero and return immediately, which will immediately call the code
associated with the second break to terminate the program.

\section{Compiler Interface}

No changes were made.

\section{Issues with SIPS}
While using the SIPS runtime environment a few issues presented themselves.  First
and for most was the apparent lack of good cross-platform coding practices exercised
when laying out the display.  Many of the data display boxes are not dynamically
resized to fit whatever font is used in them, and because of the font differences
between Windows and Linux, the displayed text will often run over the sides of the
boxes.

Also, one of the more nagging features of the SIPS environment is the file selection
dialog will not save it's previous folder so every time a user wishes to use
another object file, they must re-navigate the file system to find their file.  I'm
unsure if this problem extends to Windows, but it does exist on Linux.

The third and last issue that came up was the amount of helpful information that
SIPS produces on a loading error, which is exactly none.  When it tries to load
a invalid opcode it displays a message saying so, but it does not make it clear
at what position in the object file it occurs or what the offending opcode is.

These were issues that I found while using the SIPS environment.  They are for
informative purposes only.

\section{Working Code Samples}

\begin{center}
\textbf{All of these are available in the directory tests in the mini folder.}\\
\end{center}

Most example should compile normally displaying the line "Successful compilation!"
when finished.  Examples with "err" in their filename should error during
compilation, however during the compiler testing process these may have been
changed and there is no guarantee of their intended action.

\begin{program}
\begin{verbatim}
/*
   This program demonstrates many features of
   the language.

   Enter these input values:  5  7  10  -1
*/
program sample;

int largest = 0, i, j,
    listLength, list[50] = ('x', 'X', 20, 020, 0x20),
	value;

proc GetList;
    int limit = 50, negOne, NegOne,
        value;
    0xf423E - 03641077 -> negOne;  /* 999998 - 999999 */
    -1 -> NegOne;
    read value;
    while (negOne < value)
        if (limit <= listLength)
            break;
        endif;
        value -> list[listLength];
        call AddOne;
        read value;
    endwhile;
    NegOne + 1 -> NegOne;
    write negOne, NegOne;
endproc;

proc AddOne;
    listLength - NegOne -> listLength;
endproc;

/* Beginning of main program */

1234 -> value;
5 -> listLength;
call GetList;
write value;
0 -> i;
while (i < listLength)
    if (largest < list[i])
        list[i] -> largest + list[1] -> value;
        if (i < 4)
            i+1->j;
            value + list[j] -> value;
        endif;
    endif;
    i + 1 -> i;
endwhile;
write largest, value, list[1];

endprogram

\end{verbatim}
\caption{Listing of program sample.mini}
\end{program}

\begin{program}
\begin{verbatim}
program testarith;
	int one=1,two=2,three=3,result=0;

	/* test + */
	one + two -> result;
	write result;

	/* test - */
	two - one -> result;
	write result;

	/* test * */
	two * three -> result;
	write result;

	/* test / */
	three / two -> result;
	write result;

	/* test stringing */
	two * three -> result;
	write result;
	result + one + two -> result;
	write result;
	result / two + one -> result;
	write result;


	/* test literal */
	1 + 2 + 3 + 4 -> result;
	write result;
	4 / 2 + 4 -> result;
	write result;

endprogram;\end{verbatim}
\caption{Listing of program test\_arith.mini}
\end{program}

\begin{program}
\begin{verbatim}
program testary;
	int ary[101];

endprogram;
\end{verbatim}
\caption{Listing of program test\_ary.mini}
\end{program}

\begin{program}
\begin{verbatim}
program aryld;
	int ary[2] = (2,3),one=1,result=0;

	/* test a single ld op on the array */
	write ary[0];
	write ary[one];

	/* test in an expression */
	ary[0] * 2 -> result;
	write result;

	ary[1] * ary[0] -> result;
	write result;

	/* read reading into an array */
	read ary[0];
	ary[0] * 2 -> result;
	write result;


endprogram;\end{verbatim}
\caption{Listing of program test\_ary\_ld.mini}
\end{program}

\begin{program}
\begin{verbatim}
program testbreak;
	proc myproc;
		write 0;
		break;
	endproc;

	call myproc;
	break;
	call myproc;
endprogram;\end{verbatim}
\caption{Listing of program test\_break\_1.mini}
\end{program}

\begin{program}
\begin{verbatim}
program testcon;
	int one=1,two=2,three=3;

	if(one < two)
		write one;
	endif;

	if(two < three)
		write two;
	else
		write three;
	endif;

	if(one = one)
		write one;
	endif;

	if(one <= one)
		write one;
	endif;

	while(one <= three)
		write one;
		one + 1 -> one;
	endwhile;

endprogram;\end{verbatim}
\caption{Listing of program test\_con.mini}
\end{program}

\begin{program}
\begin{verbatim}
program 5;

endprogram;
\end{verbatim}
\caption{Listing of program test\_err\_prog0.mini}
\end{program}

\begin{program}
\begin{verbatim}
program test

endprogram;
\end{verbatim}
\caption{Listing of program test\_err\_prog1.mini}
\end{program}

\begin{program}
\begin{verbatim}
program testerrwrite;
	int one=1,two=2,three=3,ary[10]=(1,2,3,4,5,6,7,8,9,10);

	write one,two,three,ary[5];
endprogram;\end{verbatim}
\caption{Listing of program test\_err\_write.mini}
\end{program}

\begin{program}
\begin{verbatim}
program testint;
	int test,asdf;
endprogram;
\end{verbatim}
\caption{Listing of program test\_int.mini}
\end{program}

\begin{program}
\begin{verbatim}
program testintary;
	int largest = 0, i, j, listLenght,
		list[50] = ( 'x', 'X', 20, 020, 0x20), value;
endprogram;
\end{verbatim}
\caption{Listing of program test\_int\_ary.mini}
\end{program}

\begin{program}
\begin{verbatim}
program testproc;
	int one=1,two=2;

	proc myproc;
		int three=3;
		write one,three;
	endproc;

	break;
	one + two -> one;
	call myproc;

endprogram;\end{verbatim}
\caption{Listing of program test\_proc.mini}
\end{program}

\begin{program}
\begin{verbatim}
program test;

endprogram;
\end{verbatim}
\caption{Listing of program test\_prog.mini}
\end{program}

\end{document}