\documentclass[12pt]{report}
\usepackage{fancyheadings}
\usepackage[american]{babel}
\usepackage{alltt}
\usepackage{pslatex}
\usepackage{amssymb}
\usepackage{amsfonts}
\usepackage{graphicx}
\usepackage{graphics}

\textwidth  14cm
\textheight 21cm
\topmargin -20pt
\oddsidemargin  20pt
\evensidemargin 20pt

\setlength{\parindent}{0pt}

\setlength{\parskip}{1ex plus0.1ex minus0.1ex}

% copied from the BWP4 docs
\setlength{\headrulewidth}{0.8pt}
\setlength{\footrulewidth}{0.4pt}
\setlength{\headwidth}{\textwidth}

\pagestyle{fancy}

\lfoot{Mark Randles}
\cfoot{}
\rfoot{Edward Ondieki}
\lhead[\thepage]{\textit \rightmark}
\rhead[\textit \leftmark]{\thepage}

\newcommand{\mychapter}[1]{\chapter{#1} \label{ch:#1} \thispagestyle{fancy}}
\newcommand{\bslash}{\verb=\=}
\newcommand{\myitem}[1]{\item{\textbf{#1:}}}
\newcommand{\mycode}[1]{\section{#1} \small{\input{#1}} \newpage}

\newsavebox{\tempboxa}
\newenvironment{file}[1]{ \textbf{File:} \texttt{#1} \\ \label{file:#1} \nobreak \hrule \begin{alltt} }{ \end{alltt} \hrule \vspace{0.5cm} }
\newenvironment{code}[2]{ \textbf{Code:} \texttt{#1} \\ \label{code:#2} \nobreak \hrule \begin{alltt} }{ \end{alltt} \hrule \vspace{0.5cm} }

\begin{document}
\bibliographystyle{plain}

% title page
\begin{titlepage}
\vspace{9cm}
\begin{center}
    \huge Synchronous Relaxation and Time Warp Algorithms: A Study
\end{center}
\vspace{1.0cm}

\vspace{1cm}

\vspace {1cm}
\centerline{
\begin{tabular}{cc}
Mark Randles & Edward Ondieki \\
randlem@bgsu.edu & edwardo@bgsu.edu \\
\end{tabular}
}
\vspace{1cm}
\centerline{Academic Supervisor}
\centerline{Hassan Rajaei, Ph.D}

\vspace{1cm}
\centerline{\large Department of Computer Science}
\vspace{0.2cm}
\centerline{\large Bowling Green State University, Ohio}
\vspace{0.2cm}
\centerline{\large 07 Tuesday 2005}

\clearpage
\end{titlepage}
% end copy

% abstract page
\begin{abstract}
There were a variety of ideas and goals pursued during the course of this project.  First and foremost was an overriding goal of pulling the problem away from the system that it ran on.  Previous implementations were dependent upon a certain architecture, and thus the system defined the problem size.  This would need to change, the problem must become independent of the architecture.  To do this, a variety of things were looked at.  First we contemplated rewriting the initial work done on synchronous relaxation (SR).  This accomplished it's goal of making the problem size independent of the architecture.  However, there were still bottlenecks inherent in the system, so the use of Time Warp (TW) was studied.  This also should accomplish the goal of making the problem size independent of the underlying architecture.  Also discussed have been the use of logical processes and hybrid OpenMP/MPI mechanisms, on which work has not been done, but initial studies have been undertaken to study their feasibility.
\end{abstract}

% table of contents
\tableofcontents
\newpage
\mbox{}
\newpage
\mychapter{Introduction}

	In order to model physical processes like thin film growth on large areas and long  time scales, more efficient parallel algorithms need to be developed. In this project we first studied a proposed conservative parallel algorithm known as Synchronous Relaxation(SR) algorithm to try and see whether it can be made more efficient. To do this, we rewrote the initial implementation of the algorithm porting from C to C++ . Though the program was slightly more efficient after revision it still was not scalable, that is, it lost its efficiency as more processors were added.
	We then proposed a  new algorithm that would use a distributed simulation technique known as Time Warp. This optimistic algorithm should incur less synchronization costs  than the SR algorithm and should therefore scale better

\section{Monte Carlo Methods}

Monte Carlo methods are numerical techniques often used to calculate integrals by using (pseudo) random  numbers. A good example of this is the calculation of $\pi$ using the following algorithm
  	Monte Carlo for $\pi$
\begin{enumerate}
\item Generate i random number pairs (Xi, Yi) where $0<Xi<1$, $0<Yi<1$.
\item Count number of pairs within the unit circle radius $r=1$
\item $(count of pairs in r=1)/(total pairs) \approx \pi/4$
\end{enumerate}~\cite{ysja:sr}

Monte Carlo methods have also been successfully applied to solving various problems in statistical physics. A classic example is the Ising model used to study the effect of temperature on a ferromagnet~\cite{ae:kmc}.  Initial Monte Carlo algorithms like the Metropolis algorithm worked by randomly selecting an event to occur and accepting or rejecting the event based on a criteria known as the Detailed Balance criteria. However such algorithms proved slow and inefficient for modeling systems at large sizes and time scale as is the case in thin film growth.

\section{Modeling Thin Film Growth Using Kinetic Monte Carlo}

\subsection{Kinetic Monte Carlo}

Kinetic Monte Carlo is a subset of Monte Carlo techniques that has proved successful and efficient in stochastic modeling non-equilibrium systems. Kinetic is derived from the n-fold way algorithm. In this method all the rates of possible events are known beforehand as well as the probability of each rate occurring. Events are then performed and system time incremented stochastically.
Kinetic Monte Carlo is better suited for non-equilibrium systems like thin film growth or the Ising model because
\begin{enumerate}
\item It satisfies the Detailed Balance criteria an important property of stochastic processes.
\item It uses a finite amount of rates that is event rates are precalculated thus reducing computation
\item No events are rejected as opposed to previous algorithms like Metropolis algorithm which suffers from high rejection at low temperatures.
\end{enumerate}

The algorithm follows the following basic steps
\begin{enumerate}
\item Generate a uniformly distributed random number $0<X<1$
\item Use the number $X$ to select an event to occur depending on the probability of that event happening
\item Perform the event (e.g. flip atom or deposit monomer)
\item Generate a new random number $0<Y<1$
\item increment system time using the formula $T=T + log(Y)/(Σ(event rates))$
\item Repeat Step 1 until a stopping condition is equates true
\end{enumerate}

\subsection{Thin Film Growth}
\label{section:Thin Film Growth}

	Thin film growth is the process by which a layer of atoms (monomers) are deposited on a surface (substrate) usually at low pressure and temperature so as to form a surface coverage that is a few atoms thick. The morphology of the thin film formed is highly dependent on various factors such as the rates of different events for example the number of atoms(adatoms,monomers) being deposited as opposed to the rate at which atoms on the surface(substrate) diffuse. Thin film growth also takes place in enormous time scales when considering events at atomic levels i.e atomic vibrations take place at nanoseconds yet it takes hours or minutes to grow a thin film device~\cite{pc:kmc}.
	In our current model we studied a simple growth model known as the Fractal Growth Model. In this model atoms are deposited on the surface of an atom where they may then diffuse along the surface of the substrate. In case a diffusing (or deposited) atom encounters another atom, it reacts to form an island that binds the the atoms to the surface. In case an atom encounters an island(two or more bound atoms) it will be captured by that island.

To model Thin  film growth using KMC the basic data and data structures required are
\begin{enumerate}
\item A two dimensional lattice to act as a height map
\item A monomer list to keep track of the number of monomers as well as their positions on the lattice.
\item The surface diffusion  rates as well as deposition rates
\end{enumerate}

The algorithm proceeds like a standard KMC with the exception of steps 4 and 5.
\begin{enumerate}
\item Generate a random number $0<X<1$
\item Use the number $X$ to select an event to occur depending on the probability of that event happening
\item Perform the event (e.g. deposit monomer)
\item Update the Lattice heights to record change
\item Update the Monomer list in case monomers were captured or added by event
\item Generate a new random number $0<Y<1$
\item Increment system time using the formula $T=T + log(Y)/(Σ(event rates))$
\item Repeat Step 1 until $(total_depositions/(dimension_x * dimension_y))=1$
\end{enumerate}

\begin{figure}
\hrule
\vspace{0.5cm}
\begin{verbatim}
Sample Output
g++ -c lattice.cpp main.cpp synch.cpp comm.cpp mpiwrapper.cpp
ndep=441
time=0.00556387
coverage=1.00227
time=0.00556387
Execution Time=2.2689
**************S**********************************
1 0 0 2 1 0 1 1 0 1 1 1 1 1 1 0 1 1 1 1 1 0
1 0 2 2 0 1 2 1 1 1 1 1 1 1 1 1 1 1 1 0 1 1
1 0 0 2 2 2 1 1 1 1 1 1 1 1 1 1 1 1 1 1 1 1
1 1 1 1 2 2 1 1 1 1 0 0 1 1 2 2 2 1 2 2 1 1
2 0 1 1 2 2 2 1 1 1 2 2 1 2 1 0 1 0 1 2 2 1
0 0 2 2 2 0 1 1 1 2 1 2 2 1 1 3 1 1 2 2 0 0
0 0 1 1 2 2 1 1 0 1 1 0 2 2 3 1 0 1 1 2 0 1
1 1 1 1 1 0 1 1 1 1 1 1 2 1 3 1 0 1 1 0 1 2
2 1 1 2 1 1 2 1 1 1 1 0 1 1 1 2 1 0 0 1 2 1
2 0 2 2 1 2 2 2 1 1 1 1 2 0 1 2 3 1 0 1 2 1
1 0 1 1 1 1 2 2 1 0 2 2 1 0 1 1 2 2 1 2 1 1
2 0 1 1 1 1 1 2 0 1 2 2 2 1 1 1 1 1 1 1 2 0
0 0 1 1 1 1 1 2 1 1 2 0 1 1 1 1 1 1 1 1 0 0
1 0 1 1 2 2 2 1 1 1 1 2 2 2 1 0 1 1 1 0 1 0
2 0 0 2 1 2 1 1 2 1 0 2 0 1 0 1 1 1 2 2 2 0
2 0 1 0 0 2 2 0 2 2 2 2 1 0 0 2 2 1 2 2 2 1
1 1 2 0 2 1 2 1 2 2 1 1 1 1 2 1 1 2 0 1 1 2
1 2 2 2 1 1 2 2 2 2 1 0 1 1 2 1 2 1 1 0 2 3
1 0 2 1 2 2 1 2 2 2 2 2 2 2 2 2 0 2 1 3 1 1
4 0 1 1 1 0 1 2 2 1 1 0 1 2 2 2 2 2 1 1 4 6
mcount=1
\end{verbatim}
\hrule
\caption{Sample KMC Results}
\label{sampleKMC}
\end{figure}

In Figure \ref{sampleKMC} the bold 3 indicates the presence of a sole monomer on the lattice. This is because it is higher than other heights around it and therefore it does not get bound.  The monomer list contains the location $(X,Y)$ pair of the monomer. In this case the monomer is of size 1 and the $(X,Y)=(9,16)$.



\newpage
\mbox{}
\newpage
\include{sr}
\newpage
\mbox{}
\newpage
\include{timewarp}
\mychapter{Future Work}

\section{Synchronous Relaxation}

We feel that there is little work left to do on the SR algorithm.  The promise of our work is in the potential improvements that could be made with the Time Warp algorithm.  However, should Time Warp fail, this would be the most promising path for further research, so any work should not be scrapped, but rather shelved until a later date.

\section{Time Warp}

\subsection{Debugging and Validation}
This is by far the first and most important thing that will need to be done to this algorithm.  Only once this step is completed can any future work proceed.  To do this first the difference in results between runs must be identified and suppressed or eliminated.  Secondly, the validity of the results should be sought.  This step could prove to be difficult, as the original SR algorithm should always produce different results then the TW algorithm.  Because of this, the validity of results will nearly always been in question, and should be studied with veracity until we are certain.  Time to complete: 10-30 hours.

\subsection{Garbage Collection and Memory Management}
Right now, there is no garbage collection implemented in the TW algorithm.  Since this algorithm is very memory intensive, garbage collection would be a good thing.  To do this we would need to call a routine every time the GVT is calculated that will delete all of the events that occur before the GVT.  This will free memory for the next GVT cycle.  Also, if user-space memory management was implemented, we could reuse the memory to keep from reallocating the same chunks thus providing more speed to the algorithm. Time to complete: 10-20 hours.

\subsection{Logical Processes}
One of the primary goals of the TW algorithm was to add more processor independence to the KMC algorithm.  Thus far, the algorithm has seemed to do this, and now the limiting factor is the problem size and physical hardware restrictions such as memory.  However, we can go a step farther and apply a concept called logical processes to the algorithm to produce a truly processor independent implementation.

Logical processes essentially encapsulate a single KMC process.  Currently, each node on a cluster will execute one process.  With logical processes each node will execute tens, hundreds, or thousands of similar processes in such a way as to simulate the running of just one process on a single node.  To accomplish this, there must be a middle-ware layer between the program and the message passing interface.  This middle layer handles both external MPI communication with other nodes, and interval node communication between logical processes on the same node.  The goal of this middle layer is to make it as transparent as possible, the calling thread should not know if the process that will be receiving the message is local logical process or a remote (on another node) logical thread.

To complete this task a variety of things will need to be completed.  First a formal study of the middle-ware layer as exposed by the BGTW simulation library should be undertake to judge the usefulness of this established code base.  Secondly, any changed that will need to be made to the algorithm should be performed and then the simulation revalidated to ensure that the algorithm is still correct.  Thirdly, the middle-ware layer should be integrated into the existing algorithm and validated.  This last step will produce the final logical process version of the TW algorithm. Time to complete: 20-50 hours.

\section{General}

\subsection{2D Decomposition}
The current algorithms work with at 1D decomposition or strip lattice where there is at most only two neighboring lattices any given lattice.  This is limiting however in total scale of problem, as to increase the total area we must increase the height and width of the lattice.  Eventually the number of boundary events exchanged over a single boundary will slow the overall execution down.  At this point it will be beneficial to split a single sub-lattice into a 2D map of adjoining lattices.  We can then scale the problem up further till we reach the same limiting factor as before.  Time to complete: 20-40 hours.


\appendix

\mychapter{Synchronos Relaxiation Code}

% MAIN SOURCE FILE
\section{main.cpp}
\begin{code}{main.cpp}{sr/main.cpp}
{\ttfamily \raggedright \footnotesize
\#include\ <{}iostream>{}
using\ std::cout;
using\ std::cerr;
using\ std::endl;
\ \\
\#include\ <{}fstream>{}
using\ std::ofstream;
\ \\
\#include\ "{}exception.h"{}
\#include\ "{}lattice.h"{}
\#include\ "{}mpiwrapper.h"{}
\ \\
const\ int\ globalSyncThreshold\ =\ 1000;
\ \\
string\ makeFileName(string,string,int);
\ \\
int\ main(int\ argc,\ char* \ argv[])\ \{
\ \ Lattice\ lattice;
\ \ string\ pngFilename\ =\ "{}"{};
\ \ string\ logFilename\ =\ "{}"{};
\ \ fstream\ logFile;
\ \ double\ minGlobalTime\ =\ 0.0;
\ \ double\ maxGlobalTime\ =\ 0.0;
\ \ int\ globalTimeCounter\ =\ 0;
\ \ double\ gConvergence\ =\ 0.0;
\ \ int\ eventCount\ =\ 0;
\ \\
\ \ \textsl{//\ setup\ the\ lattice\ mpi\ stuff}
\ \ lattice.mpi.init(\&argc,\&argv);
\ \\
\ \ try\ \{
\ \\
\ \ \ \ pngFilename\ =\ makeFileName("{}height-{}node"{},"{}png"{},lattice.mpi.getRank());
\ \ \ \ logFilename\ =\ makeFileName("{}log"{},"{}txt"{},lattice.mpi.getRank());
\ \\
\ \ \ \ logFile.open(logFilename.c\underline\ str(),fstream::out|fstream::trunc);
\ \\
\ \ \ \ if(!logFile)\ \{
\ \ \ \ \ \ string\ error\ =\ "{}Couldn't\ open\ log\ file\ "{}\ +\ logFilename;
\ \ \ \ \ \ throw(Exception(error));
\ \ \ \ \}
\ \\
\ \ \ \ lattice.setMinGlobalTime(0.0);
\ \\
\ \ \ \ lattice.mpi.barrier();
\ \\
\ \ \ \ \textsl{//\ MAIN\ LOOP}
\ \ \ \ while(gConvergence\ <{}\ 1.0)\ \{
\ \ \ \ \ \ \textsl{//\ retrive\ any\ remote\ events}
\ \ \ \ \ \ lattice.negoitateEvents(logFile);
\ \\
\ \ \ \ \ \ \textsl{//\ do\ the\ next\ event}
\ \ \ \ \ \ lattice.doNextEvent();
\ \ \ \ \ \
\ \ \ \ \ \ \textsl{//\ see\ if\ it's\ time\ for\ a\ global\ sync}
\ \ \ \ \ \ if(globalTimeCounter\ >{}\ globalSyncThreshold)\ \{
\ \ \ \ \ \ \ \ lattice.mpi.barrier();
\ \\
\ \ \ \ \ \ \ \ lattice.negoitateEvents(logFile);
\ \\
\ \ \ \ \ \ \ \ \textsl{//\ allreduce\ to\ find\ the\ min\ time}
\ \ \ \ \ \ \ \ minGlobalTime\ =\ lattice.mpi.allReduceDouble(lattice.getLocalTime(),MPI\underline\ MIN);
\ \ \ \ \ \ \ \ maxGlobalTime\ =\ lattice.mpi.allReduceDouble(lattice.getLocalTime(),MPI\underline\ MAX);
\ \ \ \ \ \ \ \ eventCount\ =\ lattice.mpi.allReduceInt(lattice.getEventCount(),MPI\underline\ SUM);
\ \\
\ \ \ \ \ \ \ \ \textsl{//\ set\ the\ global\ time\ in\ the\ lattice}
\ \ \ \ \ \ \ \ lattice.setMinGlobalTime(minGlobalTime);
\ \\
\ \ \ \ \ \ \ \ \textsl{//\ clear\ the\ counter}
\ \ \ \ \ \ \ \ globalTimeCounter\ =\ 0;
\ \\
\ \ \ \ \ \ \ \ \textsl{//\ calculate\ the\ global\ convergence}
\ \ \ \ \ \ \ \ gConvergence\ =\ (double)eventCount/(double)(lattice.mpi.getNodeCount()\ *\ SIZE);
\ \\
\ \ \ \ \ \ \ \ if(lattice.mpi.isRoot())\ \{
\ \ \ \ \ \ \ \ \ \ cout\ <{}<{}\ minGlobalTime\ <{}<{}\ "{}\ "{}\ <{}<{}\ maxGlobalTime\ <{}<{}\ "{}\ "{}\ <{}<{}\ gConvergence\ <{}<{}\ endl;
\ \ \ \ \ \ \ \ \ \ cout.flush();
\ \ \ \ \ \ \ \ \}
\ \ \ \ \ \ \}\ else
\ \ \ \ \ \ \ \ ++globalTimeCounter;
\ \ \ \ \}
\ \\
\ \ \ \ logFile\ <{}<{}\ "{}exit\ main\ loop"{}\ <{}<{}\ endl;
\ \ \ \ logFile.flush();
\ \\
\ \ \ \ lattice.mpi.barrier();
\ \\
\ \ \ \ \textsl{//\ rollback\ to\ minimum\ global\ time}
\ \\
\ \ \ \ \textsl{//lattice.printLatticeHeight(logFile);}
\ \ \ \ logFile\ <{}<{}\ "{}gCovergence\ =\ "{}\ <{}<{}\ gConvergence\ <{}<{}\ endl;
\ \ \ \ lattice.printStats(logFile);
\ \ \ \ lattice.createHeightMap(pngFilename);
\ \ \ \ lattice.mpi.printStats(logFile);
\ \\
\ \ \ \ lattice.cleanup(logFile);
\ \\
\ \ \ \ logFile.close();
\ \\
\ \ \ \ lattice.mpi.barrier();
\ \\
\ \ \}\ catch(Exception\ err)\ \{
\ \ \ \ cerr\ <{}<{}\ err.error\ <{}<{}\ endl;
\ \ \}
\ \\
\ \ \textsl{//\ close\ the\ mpi\ stuff}
\ \ lattice.mpi.shutdown();
\ \ return(0);
\}
\ \\
string\ makeFileName(string\ prefix,\ string\ ext,\ int\ rank)\ \{
\ \ string\ output\ =\ prefix\ +\ "{}."{};
\ \ output\ +=\ (char)('a'\ +\ rank);
\ \ return(output\ +\ "{}."{}\ +\ ext);
\}
\ \\
 }
\normalfont\normalsize


\end{code}

% LATTICE SOURCE FILES
\section{lattice.h}
\begin{code}{lattice.h}{sr/lattice.h}
{\ttfamily \raggedright \small
\#include\ <{}vector>{}\\
using\ std::vector;\\
\ \\
\#include\ <{}queue>{}\\
using\ std::priority\underline\ queue;\\
\ \\
\#include\ <{}stack>{}\\
using\ std::stack;\\
\ \\
\#include\ <{}iomanip>{}\\
using\ std::setw;\\
using\ std::hex;\\
using\ std::dec;\\
using\ std::setprecision;\\
\ \\
\#include\ <{}fstream>{}\\
using\ std::fstream;\\
\ \\
\#include\ <{}string>{}\\
using\ std::string;\\
\ \\
\#include\ <{}png.h>{}\\
\ \\
\#include\ "{}exception.h"{}\\
\#include\ "{}latprim.h"{}\\
\#include\ "{}latconst.h"{}\\
\#include\ "{}event.h"{}\\
\#include\ "{}randgen.h"{}\\
\#include\ "{}rewindlist.h"{}\\
\#include\ "{}mpiwrapper.h"{}\\
\ \\
\#ifndef LATTICE\underline\ H\\
\#define LATTICE\underline\ H\\
\ \\
\#define GET\underline\ DIR(a)\ ((a\ <{}\ LEFT\underline\ X\underline\ BOUNDRY)\ ?\ LEFT\ :\ RIGHT)\\
\ \\
class\ Lattice\ \{\\
public:\\
\ \ Lattice();\\
\ \ \textasciitilde Lattice();\\
\ \\
\ \ void\ cleanup(fstream\&);\\
\ \\
\ \ bool\ doNextEvent();\\
\ \\
\ \ double\ getLocalTime()\ \{\\
\ \ \ \ return(localTime);\\
\ \ \}\\
\ \\
\ \ bool\ setMinGlobalTime(double\ mGT)\ \{\\
\ \ \ \ minGlobalTime\ =\ mGT;\\
\ \ \ \ return(true);\\
\ \ \}\\
\ \\
\ \ double\ getMinGlobalTime()\ \{\\
\ \ \ \ return(minGlobalTime);\\
\ \ \}\\
\ \\
\ \ bool\ negoitateEvents(fstream\&);\\
\ \\
\ \ \textsl{//\ DEBUG\ FUNCTIONS}\\
\ \ void\ printLatticeHeight(fstream\&\ file)\ \{\\
\ \ \ \ for(int\ i=0;\ i\ <{}\ DIM\underline\ X\ +\ GHOST\ +\ GHOST;\ ++i)\ \{\\
\ \ \ \ \ \ for(int\ j=0;\ j\ <{}\ DIM\underline\ Y;\ ++j)\ \{\\
\ \ \ \ \ \ \ \ file\ <{}<{}\ lattice[i][j].h\ <{}<{}\ "{}\ "{};\\
\ \ \ \ \ \ \}\\
\ \ \ \ \ \ file\ <{}<{}\ endl;\\
\ \ \ \ \}\\
\ \ \ \ file\ <{}<{}\ "{}-{}-{}-{}-{}-{}-{}-{}-{}-{}-{}-{}-{}-{}-{}-{}-{}-{}-{}-{}-{}-{}-{}-{}-{}-{}-{}-{}-{}-{}-{}-{}-{}-{}-{}-{}-{}-{}-{}-{}-{}-{}-{}-{}-{}-{}-{}-{}-{}"{}\ <{}<{}\ endl\ <{}<{}\ endl;\\
\ \ \ \ file.flush();\\
\ \ \}\\
\ \\
\ \ void\ printLatticeIndex(fstream\&\ file)\ \{\\
\ \ \ \ for(int\ i=0;\ i\ <{}\ DIM\underline\ X\ +\ GHOST\ +\ GHOST;\ ++i)\ \{\\
\ \ \ \ \ \ for(int\ j=0;\ j\ <{}\ DIM\underline\ Y;\ ++j)\ \{\\
\ \ \ \ \ \ \ \ if(lattice[i][j].listIndex\ >{}=\ 0)\\
\ \ \ \ \ \ \ \ \ \ file\ <{}<{}\ setw(4)\ <{}<{}\ lattice[i][j].listIndex;\\
\ \ \ \ \ \ \ \ else\\
\ \ \ \ \ \ \ \ \ \ file\ <{}<{}\ setw(4)\ <{}<{}\ "{}x"{};\\
\ \ \ \ \ \ \ \ file\ <{}<{}\ "{}\ "{};\\
\ \ \ \ \ \ \}\\
\ \ \ \ \ \ file\ <{}<{}\ endl;\\
\ \ \ \ \}\\
\ \ \ \ file\ <{}<{}\ "{}-{}-{}-{}-{}-{}-{}-{}-{}-{}-{}-{}-{}-{}-{}-{}-{}-{}-{}-{}-{}-{}-{}-{}-{}-{}-{}-{}-{}-{}-{}-{}-{}-{}-{}-{}-{}-{}-{}-{}-{}-{}-{}-{}-{}-{}-{}-{}-{}"{}\ <{}<{}\ endl\ <{}<{}\ endl;\\
\ \ \ \ file.flush();\\
\ \ \}\\
\ \\
\ \ void\ printMonomerList(fstream\&\ file)\ \{\\
\ \ \ \ file\ <{}<{}\ "{}monomerList["{}\ <{}<{}\ monomerList.size()\ <{}<{}\ "{}]\ at\ time="{}\ <{}<{}\ localTime\ <{}<{}\ endl;\\
\ \ \ \ for(int\ i=0;\ i\ <{}\ monomerList.size();\ ++i)\ \{\\
\ \ \ \ \ \ site$\ast$\ s\ =\ monomerList[i];\\
\ \ \ \ \ \ file\ <{}<{}\ i\ <{}<{}\ "{}:\ ("{}\ <{}<{}\ s-{}>{}p.x\ <{}<{}\ "{},"{}\ <{}<{}\ s-{}>{}p.y\ <{}<{}\ "{})\ h="{}\ <{}<{}\ s-{}>{}h\ <{}<{}\ "{}\ listIndex="{}\ <{}<{}\ s-{}>{}listIndex\ <{}<{}\ "{}\ "{}\ <{}<{}\ hex\ <{}<{}\ s\ <{}<{}\ dec\ <{}<{}\ endl;\\
\ \ \ \ \}\\
\ \ \ \ file\ <{}<{}\ "{}-{}-{}-{}-{}-{}-{}-{}-{}-{}-{}-{}-{}-{}-{}-{}-{}-{}-{}-{}-{}-{}-{}-{}-{}-{}-{}-{}-{}-{}-{}-{}-{}-{}-{}-{}-{}-{}-{}-{}-{}-{}-{}-{}-{}-{}-{}-{}-{}"{}\ <{}<{}\ endl\ <{}<{}\ endl;\\
\ \ \ \ file.flush();\\
\ \ \}\\
\ \\
\ \ void\ printStats(fstream\&\ file)\ \{\\
\ \ \ \ file\ <{}<{}\ setprecision(10)\ <{}<{}\ endl;\\
\ \ \ \ file\ <{}<{}\ "{}COLLECTED\ STATISTICS"{}\ <{}<{}\ endl;\\
\ \ \ \ file\ <{}<{}\ "{}-{}-{}-{}-{}-{}-{}-{}-{}-{}-{}-{}-{}-{}-{}-{}-{}-{}-{}-{}-{}-{}-{}"{}\ <{}<{}\ endl;\\
\ \ \ \ file\ <{}<{}\ "{}Convergence\ =\ "{}\ <{}<{}\ getConvergence()\ <{}<{}\ endl;\\
\ \ \ \ file\ <{}<{}\ "{}Total\ Event\ Count\ =\ "{}\ <{}<{}\ countEvents\ <{}<{}\ endl;\\
\ \ \ \ file\ <{}<{}\ "{}Total\ Diffusion\ Events\ =\ "{}\ <{}<{}\ countDiffusion\ <{}<{}\ endl;\\
\ \ \ \ file\ <{}<{}\ "{}Total\ Deposition\ Events\ =\ "{}\ <{}<{}\ (countEvents\ -{}\ countDiffusion)\ <{}<{}\ endl;\\
\ \ \ \ file\ <{}<{}\ "{}Total\ Boundry\ Events\ =\ "{}\ <{}<{}\ countBoundry\ <{}<{}\ endl;\\
\ \ \ \ file\ <{}<{}\ "{}Total\ Number\ Remote\ Events\ =\ "{}\ <{}<{}\ countRemote\ <{}<{}\ endl;\\
\ \ \ \ file\ <{}<{}\ "{}Total\ Rollbacks\ Performed\ =\ "{}\ <{}<{}\ countRollback\ <{}<{}\ endl;\\
\ \ \ \ file\ <{}<{}\ "{}Monomer\ List\ Size\ =\ "{}\ <{}<{}\ monomerList.size()\ <{}<{}\ endl;\\
\ \ \ \ file\ <{}<{}\ "{}Local\ Time\ =\ "{}\ <{}<{}\ localTime\ <{}<{}\ endl;\\
\ \ \ \ file\ <{}<{}\ "{}Min\ Global\ Time\ =\ "{}\ <{}<{}\ minGlobalTime\ <{}<{}\ endl;\\
\ \ \ \ file\ <{}<{}\ "{}Size\ =\ "{}\ <{}<{}\ SIZE\ <{}<{}\ "{}\ DIM\underline\ X\ =\ "{}\ <{}<{}\ DIM\underline\ X\ <{}<{}\ "{}\ DIM\underline\ Y\ =\ "{}\ <{}<{}\ DIM\underline\ Y\ <{}<{}\ endl;\\
\ \ \ \ file\ <{}<{}\ "{}-{}-{}-{}-{}-{}-{}-{}-{}-{}-{}-{}-{}-{}-{}-{}-{}-{}-{}-{}-{}-{}-{}-{}-{}-{}-{}-{}-{}-{}-{}-{}-{}-{}-{}-{}-{}-{}-{}-{}-{}-{}-{}-{}-{}-{}-{}-{}-{}"{}\ <{}<{}\ endl\ <{}<{}\ endl;\\
\ \ \ \ file.flush();\\
\ \ \}\\
\ \\
\ \ int\ getEventCount()\ \{\ return(countEvents\ -{}\ countDiffusion);\ \}\\
\ \\
\ \ bool\ createHeightMap(string\ filename);\\
\ \\
\ \ MPIWrapper\ mpi;\ \ \textsl{//\ I\ can't\ think\ of\ a\ better\ place\ for\ this.}\\
\ \\
\ \ bool\ rollback(const\ double);\\
\ \\
\ \ double\ getConvergence()\ \{\\
\ \ \ \ return((double)(countEvents\ -{}\ countDiffusion)\ /\ (double)(DIM\underline\ X\ $\ast$\ DIM\underline\ Y));\\
\ \ \}\\
\ \\
private:\\
\ \ double\ computeTime();\\
\ \ bool\ deposit();\\
\ \ bool\ diffuse();\\
\ \ bool\ doKMC();\\
\ \ EventType\ getNextEventType();\\
\ \ bool\ commitEvent(Event$\ast$);\\
\ \ site$\ast$\ randomMove(site$\ast$);\\
\ \ bool\ isBoundry(point);\\
\ \ bool\ isBound(site$\ast$);\\
\ \ bool\ clearBonded(site$\ast$,const\ double);\\
\ \ bool\ translateMessages(vector<{}Event$\ast$>{}$\ast$\ ,\ vector<{}message>{}$\ast$);\\
\ \ message$\ast$\ makeMessage(Event$\ast$);\\
\ \ bool\ hasAntiEvent(Event$\ast$);\\
\ \\
\ \ double\ localTime;\ \ \ \ \textsl{//\ the\ time\ local\ to\ the\ lattice}\\
\ \ double\ minGlobalTime;\ \textsl{//\ the\ minimum\ Global\ time\ (point\ of\ no\ return)}\\
\ \\
\ \ RewindList<{}site\ $\ast$>{}\ monomerList;\ \textsl{//\ list\ of\ all\ unbound\ monomers}\\
\ \ \textsl{//site\ lattice[DIM\underline\ X\ +\ GHOST\ +\ GHOST][DIM\underline\ Y];\ \ //\ the\ lattice\ (the\ extra\ two\ are\ the\ ghost\ region)}\\
\ \ site$\ast$$\ast$\ lattice;\\
\ \\
\ \ float\ depositionRate;\ \textsl{//\ the\ deposition\ rate\ of\ monomers}\\
\ \ float\ diffusionRate;\ \ \textsl{//\ the\ diffusion\ rate\ of\ monomers}\\
\ \\
\ \ int\ countDiffusion;\\
\ \ int\ countEvents;\\
\ \ int\ countBoundry;\\
\ \ int\ countRemote;\\
\ \ int\ countRollback;\\
\ \\
\ \ priority\underline\ queue<{}Event$\ast$>{}\ remoteEventList;\ \textsl{//\ list\ of\ all\ the\ remote\ dep/diffusion\ events}\\
\ \ stack<{}Event$\ast$>{}\ eventList;\ \ \ \ \ \ \ \ \ \ \ \ \ \ \ \ \textsl{//\ stack\ of\ all\ events\ to\ rollback\ the\ simulation}\\
\ \ vector<{}Event$\ast$>{}\ antiEvents;\ \ \ \ \ \ \ \ \ \ \ \ \ \ \textsl{//\ list\ of\ anti-{}events\ that\ will\ occur\ in\ the\ future}\\
\ \\
\ \ RandGen\ rng;\ \textsl{//\ random\ number\ generator}\\
\ \\
\ \ point\ movementDir[NUM\underline\ DIR];\ \textsl{//\ array\ of\ movement\ types}\\
\ \ message\ m;\ \textsl{//\ message\ for\ sending\ events}\\
\};\\
\ \\
\#endif\\
\ \\
 }
\normalfont\normalsize


\end{code}

\section{lattice.cpp}
\begin{code}{lattice.cpp}{sr/lattice.cpp}
{\ttfamily \raggedright \footnotesize
\#include\ "{}lattice.h"{}
point\ Lattice::ranmove(site\ mysite)
\{

\ \ \ \ point\ pt;
\ \ \ \ pt.x=0;pt.y=0;

\ \ \ \ \textsl{////cout<{}<{}xdir<{}<{}"{}\ \ "{};\ \ }
\ \ \ \
\ \ \ \ int\ prob=ranlist[iran]*4;\textsl{//rand()\%4;}
\ \ \ \ iran++;
\ \ \ \
\ \ \ \
\ \ \ \ pt.x=mysite.pos.x+mdir[prob].x;
\ \ \ \ pt.y=mysite.pos.y+mdir[prob].y;

\ \ \ \ \textsl{//Allow\ boundary\ movement}
\ \ \ \ \textsl{//No!!!}
\ \ \ \
\ \ \ \ if(pt.x>{}=size+2)\ pt.x-{}=1;
\ \ \ \ if(pt.x<{}0)pt.x+=1;
\ \ \ \
\ \ \ \
\ \ \ \ if(pt.y>{}=size)pt.y-{}=2;
\ \ \ \ if(pt.y<{}0)pt.y+=2;
\ \ \ \
\ \ \ \ return\ pt;
\}

Lattice::Lattice()
\{

\ \ \ \ float\ ratio=0.0;
\ \ \ \ float\ prob;
\ \ \ \ point\ newsite;
\ \ \ \ mcount=0;
\ \ \ \ deprate=1.0f,difrate=1.0e3;
\ \ \ \ ndep=0;
\ \ \ \ nevent=0;
\ \ \ \ time=0;
\ \ \ \ iran=0;
\ \ \ \ subcycle=0;
\ \ \ \ T=0.001;

\ \ \ \ int\ i=0,j=0;
\ \ \ \
\ \ \ \ for\ (i=0;i<{}2;i++)
\ \ \ \ \ \ \ \ bdycount[i]=0;

\ \ \ \ for\ (i=0;i<{}2;i++)
\ \ \ \ \ \ bdycountrec[i]=0;

\ \ \ \ for\ (i=0;i<{}2;i++)
\ \ \ \ \ \ oldbdycountrec[i]=0;

\ \ \ \ for\ (i=0;i<{}size+2;i++)
\ \ \ \ \{
\ \ \ \ for\ (j=0;j<{}size;j++)
\ \ \ \ \ \ \{
\ \ \ \ \ \ \ \ newsite.x=i;
\ \ \ \ \ \ \ \ newsite.y=j;
\ \ \ \ \ \ \ \ location[i][j].pos=newsite;
\ \ \ \ \ \ \ \ location[i][j].h=0;
\ \ \ \ \ \ location[i][j].index=-{}1;
\ \ \ \ \ \ \}
\ \ \ \ \}
\ \ \ \
\ \ \ \
\ \ \ \ \textsl{//generate\ random\ numbers}
\ \ \ \
\ \ \ \ \textsl{//randgen();}
\ \ \ \
\ \ \ \ \textsl{/*
\ \ \ \ generate\ points\ in\ the\ form\ e.g\ (1,-{}1)\ move\ one\ unit\ left\ in\ x\ and\ and\ one\ unit\ up
\ \ \ \ do\ not\ allow\ particle\ to\ remain\ (0,0)
\ \ \ \ */}


\ \ \ \ \textsl{/*Set\ directions*/}



\ \ \ \ mdir[0].y=\ 1;\ \ mdir[0].x=\ 0;
\ \ \ \ mdir[1].y=\ 0;\ \ mdir[1].x=\ 1;
\ \ \ \ mdir[2].y=-{}1;\ \ mdir[2].x=\ 0;
\ \ \ \ mdir[3].y=\ 0;\ \ mdir[3].x=-{}1;
\ \ \ \ mdir[4].y=\ 1;\ \ mdir[4].x=\ 1;
\ \ \ \ mdir[5].y=\ 1;\ \ mdir[5].x=-{}1;
\ \ \ \ mdir[6].y=-{}1;\ \ mdir[6].x=\ 1;
\ \ \ \ mdir[7].y=-{}1;\ \ mdir[7].x=-{}1;

\}
void\ Lattice::calctime()
\{
\ \ float\ \ \ monomer=(float)mcount,
\ \ \ \ Drate=difrate*monomer*0.25f,
\ \ \ \ N=(float)latsize,dt,prob,
\ \ \ \ totaldep=deprate*N;
\ \ \
\ \ prob\ =\ ranlist[iran];
\ \ \ \ \ \ \ \ iran++;
\ \ dt=-{}log(prob)/(Drate+totaldep);
\ \ time+=dt;
\}
Lattice::\textasciitilde Lattice()
\{

\}

int\ Lattice::getbonds(site\ mysite,point\ *\ bondpt)
\{
int\ ctr=0,i=0;
point\ pt;
pt=mysite.pos;
for(;i<{}dir;i++)
\{
\ \ \ pt.x+=mdir[i].x;
\ \ \ pt.y+=mdir[i].y;
\ \ \ if((pt.x<{}0)
\ \ \ \ \ \ ||(pt.x>{}=size+2)
\ \ \ \ \ \ ||(pt.y<{}0)
\ \ \ \ \ \ ||(pt.y>{}=size))
\ \ \{
\
\ \ \}
\ \ else
\ \ \{
\ \ \ \ \ if(location[pt.x][pt.y].h>{}=mysite.h)
\ \ \ \ \ \{
\ \ \ \ \ \ \ bondpt[ctr]=pt;
\ \ \ \ ctr++;
\ \ \ \ \ \}
\ \ \ \}
\ \ \ pt=mysite.pos;
\}
return\ ctr;
\}

int\ Lattice::getnbhrs(site\ mysite,point\ *\ bondpt)
\{
\ \ \ \ int\ ctr=0,i=0;
\ \ \ \ point\ pt;
\ \ \ \ pt=mysite.pos;
\ \ \ \ for(;i<{}dir;i++)
\ \ \ \ \{
\ \ \ \ pt.x+=mdir[i].x;
\ \ \ \ pt.y+=mdir[i].y;
\ \ \ \ if((pt.x<{}0)
\ \ \ \ \ \ ||(pt.x>{}=size+2)
\ \ \ \ \ \ ||(pt.y<{}0)
\ \ \ \ \ \ ||(pt.y>{}=size))
\ \ \ \ \{
\ \ \ \ \
\ \ \ \ \}
\ \ \ \ else
\ \ \ \ \{
\ \ \ \ \ \ \ \ \ bondpt[ctr]=pt;
\ \ \ \ \ \ ctr++;
\ \ \ \ \ \ \ \
\ \ \ \ \}
\ \ \ \ pt=mysite.pos;
\ \ \ \ \}
\ \ \ \ return\ ctr;
\}

void\ Lattice::deletemonomer(point\ pos)
\{
\ \ point\ lastpt;
\ \ if(mcount>{}1)
\ \ \{
\ \ \ \ lastpt=monomerloc[mcount-{}1];
\ \ \ \ addmonomerchange(diff,lastpt);
\ \ \ \ location[lastpt.x][lastpt.y].index=location[pos.x][pos.y].index;
\ \ \ \ monomerloc[location[pos.x][pos.y].index]=monomerloc[mcount-{}1];
\ \ \ \ location[pos.x][pos.y].index=-{}1;
\ \ \ \ mcount-{}-{};
\ \ \}
\ \ else
\ \ \{
\ \ \ \ location[pos.x][pos.y].index=-{}1;
\ \ \ \ mcount-{}-{};
\ \ \}
\}

bool\ Lattice::neighborIsMonomer(point\ pt,site\ mysite)
\{
\ \ bool\ bMonomer=false;
\ \ \textsl{//if\ index\ is\ not\ -{}1\ and\ same\ height\ as\ my\ location}
\ \ if\ (location[pt.x][pt.y].index!=-{}1\ \&\&\ (location[pt.x][pt.y].h==location[mysite.pos.x][mysite.pos.y].h))
\ \ \ \ \{
\ \ \ \ \ \ bMonomer=true;\ \ \ \
\ \ \ \ \}
\ \ return\ bMonomer;
\}
bool\ Lattice::checkupdatebonds(site\ mysite)
\{
\ \ \textsl{/*
\ \ Possible\ Scenarios
\ \ Deposit
\ \ 1.\ Monomer\ encounters\ no\ cluster\ or\ other\ neighbor\ monomer
\ \ -{}No\ bonds
\ \ 2.\ Monomer\ encounters\ cluster
\ \ -{}bond\ delete\ monomer\ from\ list
\ \ 3.\ Monomer\ encounters\ single\ monomer
\ \ -{}bond.\ delete\ BOTH\ from\ list

\ \ Diffusion
\ \ 1.\ Monomer\ encounters\ no\ cluster\ or\ other\ neighbor\ monomer
\ \ -{}No\ bonds
\ \ 2.\ Monomer\ encounters\ cluster
\ \ -{}bond\ delete\ monomer\ from\ list
\ \ 3.\ Monomer\ encounters\ single\ monomer
\ \ -{}bond.\ delete\ BOTH\ from\ list
\ \ */}
\ \ bool\ bond=false;
\ \ point\ pt,\ bondpt[dir],lastpt;
\ \ int\ i=0,j=0,ctr=0;
\ \ ctr=getbonds(mysite,bondpt);
\ \ if(ctr==0)
\ \ \{
\ \ \textsl{//No\ cluster\ or\ monomer;}
\ \ bond=false;
\ \ \}
\ \ else
\ \ \{
\ \ \textsl{//for\ each\ bond\ recieved}
\ \ bond=true;
\ \ for(;j<{}ctr;j++)
\ \ \ \ \ \ \ \{
\ \ \ \ pt=bondpt[j];
\ \ \ \ \textsl{//if\ monomer\ means\ it\ has\ an\ index}
\ \ \ \ if(neighborIsMonomer(pt,mysite))
\ \ \ \ \ \ \{
\ \ \ \ \ \ \textsl{//delete\ both\ you\ and\ monomer}
\ \ \ \ \ \ \textsl{//delete\ monomer}
\ \ \ \ \ \ deletemonomer(pt);
\ \ \ \ \ \ \}
\ \ \ \ \}
\ \ \ \ \ \ \textsl{//delete\ your\ self\ IF\ you\ are\ a\ monomer}
\ \ \ \ if(mysite.index!=-{}1)
\ \ \ \ \{
\ \ \ \ \textsl{//rearrange\ list\ if\ mcount\ is\ greater\ than\ 1}
\ \ \ \ \ \ deletemonomer(mysite.pos);
\ \ \ \ \}

\ \ \}
\ \ return\ bond;
\}

void\ Lattice::upnbhd(site\ mysite)
\{
\ \ \ \ \ \ \ \ bool\ bond=false;
\ \ point\ pt,\ bondpt[dir],lastpt;
\ \ int\ x,y;
\ \ int\ i=0,j=0,ctr=0;
\ \ ctr=getnbhrs(mysite,bondpt);
\ \ \textsl{//cout<{}<{}"{}getcte="{}<{}<{}ctr<{}<{}endl;}
\ \ \ checksite(mysite);
\ \ \ \ for(i=0;i<{}ctr;i++)
\ \ \ \ \{
\ \ \ \ \ \ \ x=bondpt[i].x;
\ \ \ \ \ \ \ y=bondpt[i].y;
\ \ \ \ \ \ \ checksite(location[x][y]);\textsl{//bond=checkupdatebonds(location[bondpt[i].x][bondpt[i].y]);}
\ \ \ \ \}
\ \ \ \ \ \ \ if(mysite.h<{}=0\ \&\&\ mysite.index!=-{}1)
\ \ \ \ \ \ \ \ \ \{
\ \ \ \ \ \ deletemonomer(mysite.pos);
\ \ \ \}

\}
void\ Lattice::checksite(site\ mysite)
\{
\ \ bool\ bond=false;
\ \ bond=checkupdatebonds(location[mysite.pos.x][mysite.pos.y]);\ \ \
\ \ \ \ if(bond==false)
\ \ \ \ \{
\ \ \ \ \ \ \ \ if((location[mysite.pos.x][mysite.pos.y].index==-{}1)\ \&\&(location[mysite.pos.x][mysite.pos.y].h>{}0))
\ \ \ \ \ \ \ \ \ \{
\ \ \ \ \ \ \ \ \ \ \ \ location[mysite.pos.x][mysite.pos.y].index=mcount;
\ \ \ \ \ \ \ \ \ \ \ \ monomerloc[mcount]=location[mysite.pos.x][mysite.pos.y].pos;
\ \ \ \ \ \ \ \ \ \ \ \ mcount++;
\ \ \ \ \ \ \ addmonomerchange(0,mysite.pos);
\ \ \ \ \ \ \ \ \ \}
\ \ \ \ \}
\}
void\ Lattice::addmonomerchange(int\ tag,int\ lastpoint)
\{
\ \ change[changecount].time=time;
\ \ if(tag==0)
\ \ \{
\ \ \ \ myevents[eventcount].oldval=monomerloc[count];
\ \ \ \ change[changecount].newsite=mcount;
\ \ \ \ change[changecount].newval=monomerloc[mcount];
\ \ \ \ change[changecount].tag=0;
\ \ \ \ changecount++;
\ \ \}
\ \ else
\ \ \{
\ \ \ \ change[changecount].oldsite=lastpoint;
\ \ \ \ change[changecount].oldval=monomerloc[lastpoint];
\ \ \ \ change[changecount].newsite=count-{}1;
\ \ \ \ change[changecount].newval=monomerloc[count-{}1];
\ \ \ \ change[changecount].tag=1;
\ \ \ \ changecount++;
\ \ \}

\}
void\ Lattice::restorelist(float\ Ctime)
\{
\ \ int\ oldloc,newloc,tag;
\ \ site\ oldval,newval;
\ \ int\ j=changecount-{}1;
\ \ while(change[j].time>{}Ctime)
\ \ \{
\ \ oldloc=change[j].oldsite;
\ \ newloc=change[j].newsite;
\ \ oldval=change[j].oldval;
\ \ newval=change[j].newval;
\ \ tag=change[j].tag;
\ \ if(tag==1)
\ \ \ \ \ \ \ \{
\ \ \ \ monomerloc[newloc]=newval;
\ \ \ \ \ \ \ monomerloc[oldloc]=oldval;
\ \ \ \ mcount++;
\ \ \ \ \}
\ \ else
\ \ \ \ \{
\ \ \ \ monomerloc[newloc]=oldval;
\ \ \ \ mcount-{}-{};
\ \ \ \ \}
\ \ \ \ j-{}-{};
\ \ \}
\}
void\ Lattice::saveconfig()
\{
\ \ int\ j;
\ \ for\ (j=0;j<{}mcount;j++)
\ \ \{
\ \ \ \ \ \ oldlist.monomerloc[j]=monomerloc[j];
\ \ \}
\ \ \
\ \ oldlist.mcount=mcount;
\ \ oldlist.ndep=ndep;
\}

void\ Lattice::restorelist()
\{
\ \ int\ j;
\ \ for\ (j=0;j<{}oldlist.mcount;j++)
\ \ \{
\ \ \ \ monomerloc[j]=oldlist.monomerloc[j];
\ \ \}
\
\ \ mcount=oldlist.mcount;
\ \ ndep=oldlist.ndep;
\}

void\ Lattice::restoreLattice()
\{
\ \ \ undoevent();
\ \ \ restorelist();
\ \ \ int\ i,j;
\ \ \ int\ x,y;
\ \ \ \textsl{/**clear\ lattice**/}
\ \ \ for(i=0;i<{}size+2;i++)
\ \ \ \{
\ \ \ \ \ for(j=0;j<{}size;j++)
\ \ \ \ \ \{
\ \ \ \ \ \ \ \ location[i][j].index=-{}1;
\ \ \ \ \ \}
\ \ \ \}
\ \
\ \ \ \textsl{/** restore\ indexes*/}
\ \ \ for(i=0;i<{}mcount;i++)
\ \ \ \{
\ \ \ \ \ x=monomerloc[i].x;
\ \ \ \ \ y=monomerloc[i].y;
\ \ \ \ \ location[x][y].index=i;
\ \ \ \}
\}

void\ Lattice::restoreLattice(float\ Ctime)
\{
\ \ undoevent(Ctime);
\ \ restorelist(Ctime);
\ \ int\ i,j;
\ \ int\ x,y;
\ \ \textsl{/**clear\ lattice**/}
\ \ for(i=0;i<{}size+2;i++)
\ \ \{
\ \ \ \ for(j=0;j<{}size;j++)
\ \ \ \ \{
\ \ \ \ \ \ location[i][j].index=-{}1;
\ \ \ \ \}
\ \ \}
\ \ \ \
\ \ \textsl{/** restore\ indexes*/}
\ \ for(i=0;i<{}mcount;i++)
\ \ \{
\ \ \ \ x=monomerloc[i].x;
\ \ \ \ y=monomerloc[i].y;
\ \ \ \ location[x][y].index=i;
\ \ \}
\}

void\ Lattice::addbdyevent(site\ oldsite,site\ newsite,float,int\ tag)
\{
\ \ //add\ boundary\ events\ to\ list
\ \ \ \ int\ bdyrightcount=bdycount[right],bdyleftcount=bdycount[left];
\ \ //erase\ pointers
\ \ \ \ oldsite.index=-{}1;newsite.index=-{}1;
\ \ if((oldsite.pos.x>{}=size)\ ||\ (newsite.pos.x>{}=size))
\ \ \{
\ \ \ \ if(oldsite.pos.x==size)
\ \ \ \ \{
\ \ \ \ \ \ oldsite.pos.x=0;
\ \ \ \ \}
\ \ \ \
\ \ \ \ if(oldsite.pos.x==size+1)
\ \ \ \ \{
\ \ \ \ \ \ oldsite.pos.x=1;
\ \ \ \ \}
\ \ \ \
\ \ \ \ if(newsite.pos.x==size)
\ \ \ \ \{
\ \ \ \ \ \ newsite.pos.x=0;
\ \ \ \ \}
\ \ \ \
\ \ \ \ if(newsite.pos.x==size+1)
\ \ \ \ \{
\ \ \ \ \ \ newsite.pos.x=1;
\ \ \ \ \}
\ \ \ \
\ \ \ \ bdyevent[right][bdyrightcount].oldsite=oldsite;
\ \ \ \ bdyevent[right][bdyrightcount].newsite=newsite;
\ \ \ \ bdyevent[right][bdyrightcount].t=time;
\ \ \ \ bdyevent[right][bdyrightcount].tag=tag;
\ \ \ \ bdycount[right]++;
\ \ \}

\ \ if((oldsite.pos.x<{}=1)\ ||\ (newsite.pos.x<{}=1))
\ \ \{\ \ \
\ \ \ \ if(oldsite.pos.x==0)
\ \ \ \ \{
\ \ \ \ \ \ oldsite.pos.x=size;
\ \ \ \ \}
\ \ \ \
\ \ \ \ if(oldsite.pos.x==1)
\ \ \ \ \{
\ \ \ \ \ \ oldsite.pos.x=size+1;
\ \ \ \ \}
\ \ \ \
\ \ \ \ if(newsite.pos.x==0)
\ \ \ \ \{
\ \ \ \ \ \ newsite.pos.x=size;
\ \ \ \ \}
\ \ \ \
\ \ \ \ if(newsite.pos.x==1)
\ \ \ \ \{
\ \ \ \ \ \ newsite.pos.x=size+1;
\ \ \ \ \}\
\ \ \ \ bdyevent[left][bdyleftcount].oldsite=oldsite;
\ \ \ \ bdyevent[left][bdyleftcount].newsite=newsite;
\ \ \ \ bdyevent[left][bdyleftcount].t=time;
\ \ \ \ bdyevent[left][bdyleftcount].tag=tag;
\ \ \ \ bdycount[left]++;
\ \ \}

\}


void\ Lattice::deposit()
\{
\ \ \textsl{/*
\ \ 1.\ Find\ a\ location
\ \ 2.\ place\ monomer\ in\ monomer\ list\ IF\ NO\ neighbours\ around!
\ \ 3.\ \ do\ not\ deposit\ on\ ghost\ region;
\ \ */}
\ \ \ \ float\ xrand=ranlist[iran];
\ \ \ \ iran++;
\ \ \ \ float\ yrand=ranlist[iran];
\ \ \ \ iran++;

\ \ \ \ int\ locx=xrand*(size)+1;
\ \ \ \ int\ locy=yrand*(size);

\ \ \ \ \textsl{//add\ height}
\ \ \ \ location[locx][locy].h+=1;
\ \ \ \ if((locx==1)\ ||\ (locy==size))
\ \ \ \ \{
\ \ \ \ \ \ \textsl{//add\ event\ to\ bdylist}
\ \ \ \ \ \ addbdyevent(location[locx][locy],location[locx][locy],time,depevent);
\ \ \ \ \ \ \ \ \ \
\ \ \ \ \}\ \ \ \ \ \ \ \
\ \ \ \ upnbhd(location[locx][locy]);
\ \ \ \ \textsl{//cout<{}<{}"{}newdeploc.x=\ "{}<{}<{}locx<{}<{}"{}\ newdeploc.y=\ "{}<{}<{}locy<{}<{}endl;\ }
\ \ \ \ \textsl{//add\ to\ eventlist;}
\ \ \
\ \ \ \ myeventlist[nevent].oldsite=location[locx][locy];
\ \ \ \ myeventlist[nevent].newsite=location[locx][locy];
\ \ \ \ myeventlist[nevent].ranseq=nevent;
\ \ \ \ myeventlist[nevent].t=time;
\ \ \ \ myeventlist[nevent].tag=depevent;
\}

void\ Lattice::diffuse()
\{

\ \ point\ newloc,oldloc,lastloc;
\ \ bool\ bonded=false;
\ \ \
\ \ float\ ranm=ranlist[iran];
\ \ iran++;
\ \ \
\ \ if(mcount>{}0)
\ \ \{
\ \ \ \ \ \ int\ loc=(ranm)*(mcount-{}1);
\ \ \ \ \ \ oldloc=monomerloc[loc];
\ \ \ \ \ \ newloc=ranmove(location[oldloc.x][oldloc.y]);
\ \ \ \ \ \ location[newloc.x][newloc.y].h+=1;
\ \ \ \ \ \ location[oldloc.x][oldloc.y].h-{}=1;
\ \ \ \ \textsl{//cout<{}<{}"{}locx"{}<{}<{}oldloc.x<{}<{}"{}locy"{}<{}<{}oldloc.y<{}<{}endl;}
\ \ \ \ \textsl{//cout<{}<{}"{}nlocx"{}<{}<{}newloc.x<{}<{}"{}nlocy"{}<{}<{}newloc.y<{}<{}endl;}
\ \ \ \ \textsl{//cout<{}<{}"{}bdyevent"{}<{}<{}bdycount<{}<{}endl;}
\ \ \ \ if(location[oldloc.x][oldloc.y].h<{}0)
\ \ \ \ \ \ \{
\ \ \ \ \ \ cout<{}<{}"{}fuuuuuuuuuuuuuuuuck"{}<{}<{}endl;
\ \ \ \ \}
\ \ \ \ \textsl{//boundary\ event}
\ \ \ \ \ \ \ \ \
\ \ \ \ if((newloc.x<{}1\ )||\ (newloc.x>{}size))
\ \ \ \ \ \ \{
\ \ \ \ \textsl{//add\ event\ to\ bdylist}
\ \ \ \ deletemonomer(oldloc);
\ \ \ \ addbdyevent(location[oldloc.x][oldloc.y],location[newloc.x][newloc.y],time,diffevent);
\ \ \ \ \
\ \ \ \ \}
\ \ \ \ if(newloc.x==1\ ||\ newloc.x==size)
\ \ \ \ \{
\ \ \ \ addbdyevent(location[oldloc.x][oldloc.y],location[newloc.x][newloc.y],time,diffevent);
\ \ \ \ \}
\ \ \
\ \ \ \ if((oldloc.x<{}=1\ )||\ (oldloc.x>{}=size))
\ \ \ \ \ \ \{
\ \ \ \ \textsl{//add\ event\ to\ bdylist}
\ \ \ \ addbdyevent(location[oldloc.x][oldloc.y],location[newloc.x][newloc.y],time,diffevent);
\ \ \ \ \
\ \ \ \ \}
\ \ \ \ \ \ \
\ \ \ \ \textsl{//diffusion\ may\ release\ trapped\ monomer\ but\ capture\ released\ monomer}
\ \ \ \ if(location[oldloc.x][oldloc.y].h>{}location[newloc.x][newloc.y].h)
\ \ \ \ \{
\ \ \ \ upnbhd(location[oldloc.x][oldloc.y]);\ \
\ \ \ \ \}
\ \ \ \ else
\ \ \ \ \{
\ \ \ \ \textsl{//Move\ Monomer\ by\ changing\ index\ location\ }
\ \ \ \ location[newloc.x][newloc.y].index=location[oldloc.x][oldloc.y].index;
\ \ \ \ monomerloc[location[oldloc.x][oldloc.y].index]=location[newloc.x][newloc.y].pos;
\ \ \ \ location[oldloc.x][oldloc.y].index=-{}1;
\ \ \ \ upnbhd(location[newloc.x][newloc.y]);
\ \ \ \ \}
\ \ \
\ \ \}
\ \ \ \
\ \ \textsl{//add\ to\ eventlist;}
\ \ myeventlist[nevent].oldsite=location[oldloc.x][oldloc.y];
\ \ myeventlist[nevent].newsite=location[newloc.x][newloc.y];
\ \ myeventlist[nevent].ranseq=nevent;
\ \ myeventlist[nevent].t=time;
\ \ myeventlist[nevent].tag=diffevent;
\ \ \ \
\ \ \textsl{//cout<{}<{}"{}****End\ Diffuson*************************"{}<{}<{}endl;\ \ }
\}
void\ Lattice::savebdylist()
\{
\ \ \ \ int\ a,b,i,bdyrightcrec=bdycountrec[right],bdyleftcrec=bdycountrec[left];

\ \ \ \ for(i=0;i<{}bdyleftcrec;i++)
\ \ \ \ \{
\ \ \ \ \ \ oldbdyeventrec[left][i]=bdyeventrec[left][i];
\ \ \ \ \}
\ \ \ \ oldbdycountrec[left]=bdyleftcrec;


\ \ \ \ for(i=0;i<{}bdyrightcrec;i++)
\ \ \ \ \{
\ \ \ \ \ \ oldbdyeventrec[right][i]=bdyeventrec[right][i];
\ \ \ \ \}

\ \ \ \ oldbdycountrec[right]=bdyrightcrec;
\}

int\ Lattice::comparebdylist()
\{
\ \ int\ a,\ b,\ acheck,\ bcheck;

\ \ \ \ acheck\ =\ 0;
\ \ \ \ bcheck\ =\ 0;
\ \ \ \ for\ (a=0;\ a\ <{}\ 2;\ a++)\ \{
\ \ \ \ \ \ if\ (oldbdycountrec[a]!=\ bdycountrec[a])\ \{
\ \ \ \ \ \ \ \ redoflag\ =\ 1;
\ \ \ \ \ \ \ \ acheck\ \ \ =\ 1;
\ \ \ \ \ \ \}\ else\ \{
\ \ \ \ \ \ \ \ for\ (b=0;\ b\ <{}\ bdycountrec[a];\ )\ \{
\ \ \ \ \ \ \ \ \ \ if\ (oldbdyeventrec[a][b].t\ !=\ bdyeventrec[a][b].t)\ \{
\ \ \ \ \ \ \ \ \ \ \ \ redoflag\ =\ 1;
\ \ \ \ \ \ \ \ \ \ \ \ bcheck\ \ \ =\ 1;
\ \ \ \ \ \ \ \ \ \ \ \ b\ \ \ \ \ \ \ \ =\ bdycountrec[a];
\ \ \ \ \ \ \ \ \ \ \}
\ \ \ \ \ \ \ \ \ \ if\ (oldbdyeventrec[a][b].newsite.pos.x\ !=\ bdyeventrec[a][b].newsite.pos.x)\ \{
\ \ \ \ \ \ \ \ \ \ redoflag\ =\ 1;
\ \ \ \ \ \ \ \ \ \ \ \ bcheck\ \ \ =\ 1;
\ \ \ \ \ \ \ \ \ \ \ \ b\ \ \ \ \ \ \ \ =\ bdycountrec[a];
\ \ \ \ \ \ \ \ \ \ \}
\ \ \ \ \ \ \ \ \ \ if\ (oldbdyeventrec[a][b].newsite.pos.y!=bdyeventrec[a][b].newsite.pos.y)\ \{
\ \ \ \ \ \ \ \ \ \ redoflag\ =\ 1;
\ \ \ \ \ \ \ \ \ \ \ \ bcheck\ \ \ =\ 1;
\ \ \ \ \ \ \ \ \ \ \ \ b\ \ \ \ \ \ \ \ =\ bdycountrec[a];
\ \ \ \ \ \ \ \ \ \ \}
\ \ \ \ \ \ \ \ \ \ if\ (oldbdyeventrec[a][b].newsite.h!=bdyeventrec[a][b].newsite.h)\ \{
\ \ \ \ \ \ \ \ \ \ \ \ redoflag\ =\ 1;
\ \ \ \ \ \ \ \ \ \ \ \ bcheck\ \ \ =\ 1;
\ \ \ \ \ \ \ \ \ \ \ \ b\ \ \ \ \ \ \ \ =\ bdycountrec[a];
\ \ \ \ \ \ \ \ \ \ \}
\ \ \ \ \ \ \ \ \ \ b++;
\ \ \ \ \ \ \ \ \}
\ \ \ \ \ \ \}
\ \ \ \ \}

\}


void\ Lattice::p()
\{
\ \ cout<{}<{}"{}**************S**********************************"{};
\ \ float\ theta=0,vacancy=(float)\ mcount\ ,lat=(float)latsize;
\ \ float\ x;
\ \ for\ (int\ i=0;i<{}size;i++)
\ \ \{
\ \ cout<{}<{}endl;
\ \ for\ (int\ j=0;j<{}size+2;j++)
\ \ cout<{}<{}location[j][i].h;
\ \ \}

\ \ cout<{}<{}endl<{}<{}"{}mcount="{}<{}<{}mcount<{}<{}endl;
\ \ \textsl{//cout<{}<{}"{}**************E**********************************"{}<{}<{}endl;}
\ \ theta=(lat-{}vacancy)/lat;
\ \ \textsl{//cout<{}<{}"{}Theta="{}<{}<{}theta<{}<{}endl;}
\ \ for\ (int\ i=0;i<{}=size;i++)
\ \ \{
\ \ cout<{}<{}endl;
\ \ for\ (int\ j=0;j<{}size+2;j++)
\ \ cout<{}<{}location[j][i].index<{}<{}"{}\ \ \ "{};
\ \ \}
\}
void\ Lattice::doKMC()
\{
\ \ \textsl{///create\ and\ save\ random\ number}
\ \ float\ ranX;

\ \ ranX=ranlist[iran];
\ \ iran++;

\ \ float\ Trate,Drate;
\ \ Drate=.25*mcount*difrate;
\ \ Trate=Drate+(deprate*\ (float)\ latsize);

\ \ float\ prob=(Drate/Trate);

\ \ \textsl{//cout<{}<{}"{}ranX="{}<{}<{}ranlist[iran]<{}<{}endl;}
\ \ if(ranX<{}prob)
\ \ diffuse();
\ \ else
\ \ \{
\ \ deposit();
\ \ ndep++;
\ \ \}
\ \ nevent++;
\
\}

void\ Lattice::undoevent()\ \{
\ \ \ \ int\ a,\ xi,\ yi,\ xf,\ yf,\ tag;
\ \ \ \ double\ t;

\ \ \ \ if\ (redoflag\ ==\ 0)\ \{
\ \ \ \ \ \ \ \ return;
\ \ \}

\ \ \ \ for\ (a=nevent-{}1;\ a\ >{}=0;\ a-{}-{})\ \{
\ \ \ \ \ \ \ \ tag\ \ =\ myeventlist[a].tag;
\ \ \ \ \ \ \ \ xi\ \ \ =\ myeventlist[a].oldsite.pos.x;
\ \ \ \ \ \ \ \ yi\ \ \ =\ myeventlist[a].oldsite.pos.y;
\ \ \ \ \ \ \ \ xf\ \ \ =\ myeventlist[a].newsite.pos.x;
\ \ \ \ \ \ \ \ yf\ \ \ =\ myeventlist[a].newsite.pos.y;
\ \ \ \ \ \ \ \ t\ \ \ \ =\ myeventlist[a].t;

\ \ \ \ \ \ \ \ switch\ (tag)\ \{
\ \ \ \ \ \ \ \ case\ 0:
\ \ \ \ \ \ \ \ \ \ \ \ location[xi][yi].h\ =\ myeventlist[a].oldsite.h;
\ \ \ \ \ \ \ \ \ \ \ \ if\ (myeventlist[a].newsite.h\ !=\ -{}1)
\ \ \ \ \ \ \ \ \ \ \ \ \ \ \ \ location[xf][yf].h\ =\ myeventlist[a].newsite.h;
\ \ \ \ \ \ \ \ \ \ \ \ break;
\ \ \ \ \ \ \ \ case\ 1:
\ \ \ \ \ \ \ \ \ \ \ \ location[xi][yi].h\ =\ location[xi][yi].h\ +\ 1;
\ \ \ \ \ \ \ \ \ \ \ \ location[xf][yf].h\ =\ location[xf][yf].h\ -{}\ 1;
\ \ \ \ \ \ \ \ \ \ \ \ break;
\ \ \ \ \ \ \ \ case\ 2:
\ \ \ \ \ \ location[xi][yi].h\ =\ location[xi][yi].h\ -{}\ 1;
\ \ \ \ \ \ \ \ \ \ \ \ ndep-{}-{};
\ \ \ \ \ \ \ \ \ \ \ \ break;
\ \ \ \ \ \ \ \ \ \ \
\ \ \ \ \ \ \ \ default:
\ \ \ \ \ \ \ \ \ \ \ \ cout<{}<{}"{}Error\ in\ tag"{}<{}<{}endl;
\ \ \ \ \ \ \ \ \ \ \ \ return;\ \textsl{/*\ SHOULDN'T\ THIS\ EXIT()?!?\ */}
\ \ \ \ \ \ \ \ \}
\ \ \ \ \}
\}

void\ Lattice::undoevent(float\ Ctime)\ \{
\ \ \ \ int\ a,\ xi,\ yi,\ xf,\ yf,\ tag;
\ \ \ \ double\ t;

\ \ \ \ if\ (redoflag\ ==\ 0)\ \{
\ \ \ \ \ \ \ \ return;
\ \ \}
\ \ \ \ a=nevent-{}1;
\ \ \ t\ \ \ \ =\ myeventlist[a].t;
\ \ \ \ while(t>{}Ctime)\ \{
\ \ \ \ \ \ \ \ tag\ \ =\ myeventlist[a].tag;
\ \ \ \ \ \ \ \ xi\ \ \ =\ myeventlist[a].oldsite.pos.x;
\ \ \ \ \ \ \ \ yi\ \ \ =\ myeventlist[a].oldsite.pos.y;
\ \ \ \ \ \ \ \ xf\ \ \ =\ myeventlist[a].newsite.pos.x;
\ \ \ \ \ \ \ \ yf\ \ \ =\ myeventlist[a].newsite.pos.y;
\ \ \ \ \ \ \ \ t\ \ \ \ =\ myeventlist[a].t;

\ \ \ \ \ \ \ \ switch\ (tag)\ \{
\ \ \ \ \ \ \ \ case\ 0:
\ \ \ \ \ \ \ \ \ \ \ \ location[xi][yi].h\ =\ myeventlist[a].oldsite.h;
\ \ \ \ \ \ \ \ \ \ \ \ if\ (myeventlist[a].newsite.h\ !=\ -{}1)
\ \ \ \ \ \ \ \ \ \ \ \ \ \ \ \ location[xf][yf].h\ =\ myeventlist[a].newsite.h;
\ \ \ \ \ \ \ \ \ \ \ \ break;
\ \ \ \ \ \ \ \ case\ 1:
\ \ \ \ \ \ \ \ \ \ \ \ location[xi][yi].h\ =\ location[xi][yi].h\ +\ 1;
\ \ \ \ \ \ \ \ \ \ \ \ location[xf][yf].h\ =\ location[xf][yf].h\ -{}\ 1;
\ \ \ \ \ \ \ \ \ \ \ \ break;
\ \ \ \ \ \ \ \ case\ 2:
\ \ \ \ \ \ location[xi][yi].h\ =\ location[xi][yi].h\ -{}\ 1;
\ \ \ \ \ \ \ \ \ \ \ \ ndep-{}-{};
\ \ \ \ \ \ \ \ \ \ \ \ break;
\ \ \ \ \ \ \ \ \ \ \
\ \ \ \ \ \ \ \ default:
\ \ \ \ \ \ \ \ \ \ \ \ cout<{}<{}"{}Error\ in\ tag"{}<{}<{}endl;
\ \ \ \ \ \ \ \ \ \ \ \ return;\ \textsl{/*\ SHOULDN'T\ THIS\ EXIT()?!?\ */}
\ \ \ \ \ \ \ \ \}
\ \ \ \ a-{}-{};
\ \ \ \ \}
\}

void\ Lattice::randgen()
\{
\ \ int\ i;
\ \ for(i=0;i<{}10000;i++)
\ \ \{
\ \ \ \ ranlist[i]=((float)rand()/(float)RAND\underline\ MAX);
\ \ \}
\}
void\ Lattice::updateBuffer(int\ iranflag)\ \{
\ \ \ \ int\ a,\ b,\ am1,\ x,\ y,\ xi,\ ii,\ abflag,\ mflag,\ sdir,\ dir,\ aid,\ i,\ j,\ hij,\ hxy,tag;
\ \ \ \ double\ newTrate,\ oldTrate;
\ \ \ \ point\ oldsite,newsite;
\ \ \
\ \ \ \ i=sortbdyevent[nupdate].oldsite.pos.x;
\ \ \ \ j=sortbdyevent[nupdate].oldsite.pos.y;
\ \ \
\ \ \ \ x=sortbdyevent[nupdate].newsite.pos.x;
\ \ \ \ y=sortbdyevent[nupdate].newsite.pos.y;
\ \ \
\ \ \ \ tag=sortbdyevent[nupdate].tag;

\ \ \ \ time\ =\ sortbdyevent[nupdate].t;
\ \
\ \ if\ (redoflag\ ==\ 0)\ \{
\ \ \ \ \ \ \ \ return;
\ \ \}
\ \ \ \textsl{//cout<{}<{}"{}update\ buffer!"{}<{}<{}endl;}
\ \ \ \textsl{//cout<{}<{}"{}x="{}<{}<{}x<{}<{}"{}\ y="{}<{}<{}nupdate<{}<{}endl;}
\ \ \ \textsl{//cout<{}<{}"{}i="{}<{}<{}x<{}<{}"{}\ j="{}<{}<{}y<{}<{}endl;}
\ \ \
\ \ \ \
\ \ \
\ \ \ \ if(tag==diffevent)
\ \ \{
\ \ \ \ \ \ \ \ location[i][j].h\ =\ sortbdyevent[nupdate].oldsite.h;
\ \ \ \ \ \ \ \ upnbhd(sortbdyevent[nupdate].oldsite);
\ \ \ \ \ \ \ \ \}
\ \ \ \ else
\ \ \{
\ \ location[x][y].h\ =\ sortbdyevent[nupdate].newsite.h;
\ \ \ \ \ \ \ \ upnbhd(sortbdyevent[nupdate].newsite);
\ \ \}\ \

\ \textsl{/*\ add\ this\ event\ in\ my\ event\ list\ */}
\ \ \ \ myeventlist[nevent].oldsite=location[x][y];
\ \ \ \ myeventlist[nevent].newsite=location[i][j];
\ \ \ \ myeventlist[nevent].ranseq=iran\ -{}\ iranflag;
\ \ \ \ myeventlist[nevent].t=time;
\ \ \ \ myeventlist[nevent].tag=0;
\ \
\ \ \ \ nupdate++;
\ \ \ \ nevent++;
\}

void\ Lattice::sorting\underline\ nbevent()\ \{
\ \ \ \ int\ a,\ b,\ nxcv,\ i,\ j,\ caselabel,\ dir,\ idn;
\ \ \ \ double\ t;
\ \ \ \ boundaryevent\ swap;

\ \ \ \ if\ (bdycountrec[left]\ >{}\ 0\ \&\&\ bdycountrec[right]\ ==\ 0)
\ \ \ \ \ \ \ \ caselabel\ =\ 0;
\ \ \ \ if\ (bdycountrec[left]\ ==\ 0\ \&\&\ bdycountrec[right]\ >{}\ 0)
\ \ \ \ \ \ \ \ caselabel\ =\ 1;
\ \ \ \ if\ (bdycountrec[left]\ >{}\ 0\ \&\&\ bdycountrec[right]\ >{}\ 0)
\ \ \ \ \ \ \ \ caselabel\ =\ 2;

\ \ \ \ switch(caselabel)\ \{
\ \ \ \ case\ 0:
\ \ \ \ \ \ \ \ for\ (a=0;\ a\ <{}\ bdycountrec[left];\ a++)\ \{
\ \ \ \ \ \ \ \ \ \ \ \ sortbdyevent[a]=\ bdyeventrec[0][a];
\ \ \ \ \ \ \ \ \ \ \ \ \}
\ \ \ \ \ \ \ \ tnbdyevent\ =\ bdycountrec[0];
\ \ \ \ \ \ \ \ break;
\ \ \ \ case\ 1:
\ \ \ \ \ \ \ \ for\ (a=0;\ a\ <{}\ bdycountrec[1];\ a++)\ \{
\ \ \ \ \ \ \ \ \ \ \ \ sortbdyevent[a]\ =\ bdyeventrec[1][a];
\ \ \ \ \}
\ \ \ \ \ \ \ \ tnbdyevent\ =\ bdycountrec[1];
\ \ \ \ \ \ \ \ break;
\ \ \ \ case\ 2:
\ \ \ \ \ \ \ \ tnbdyevent\ =\ bdycountrec[0]\ +\ bdycountrec[1];
\ \ \ \ \ \ \ \ nxcv\ =\ 0;
\ \ \ \ \ \ \ \ \textsl{/*\ sort\ the\ events\ in\ early\ time\ order\ */}
\ \ \ \ \ \ \ \ for\ (a\ =\ 0;\ a\ <{}\ bdycountrec[0];\ a++)\ \{
\ \ \ \ \ \ \ \ \ \ \ \ sortbdyevent[nxcv]=\ bdyeventrec[0][a];
\ \ \ \ \ \ \ \ \ \ \ \ nxcv++;
\ \ \ \ \ \ \ \ \}

\ \ \ \ \ \ \ \ for\ (a=0;\ a\ <{}\ bdycountrec[1];\ a++)\ \{
\ \ \ \ \ \ \ \ \ \ \ \ sortbdyevent[nxcv]=\ bdyeventrec[1][a];
\ \ \ \ \ \ \ \ \ \ \ \ nxcv++;
\ \ \ \ \ \ \ \ \}

\ \ \ \ \ \ \ \ for\ (j=1;\ j\ <{}\ tnbdyevent;\ j++)\ \{
\ \ \ \ \ \ \ \ \ \ \ \ swap=sortbdyevent[j];
\ \ \ \ \ \ \ \ \ \ \ \ i\ =\ j\ -{}\ 1;
\ \ \ \ \ \ \ \ \ \ \ \ while\ (i\ >{}=\ 0\ \&\&\ sortbdyevent[i].t\ >{}\ t)\ \{
\ \ \ \ \ \ \ \ \ \ \ \ \ \ \ \ sortbdyevent[i+1]\ =\ sortbdyevent[i];
\ \ \ \ \ \ \ \ \ \ \ \ \ \ \ \ i-{}-{};
\ \ \ \ \ \ \ \ \ \ \ \ \}
\ \ \ \ \ \ \ \ \ \ \ \ sortbdyevent[i+1]=swap;
\ \ \ \ \ \ \ \ \}
\ \ \ \ \ \ \ \ break;
\ \ \ \ default:
\ \ \ \ \ \ \ \ break;
\ \ \ \ \}
\}

 }
\normalfont\normalsize


\end{code}

% SYNCH SOURCE FILES
\section{synch.cpp}
\begin{code}{synch.cpp}{sr/synch.cpp}
{\ttfamily \raggedright \footnotesize
\#include\ "{}lattice.h"{}
extern\ MPIWrapper\ mpi;
void\ synch(Lattice\ *\ newlatt)
\{
int\ ctr,iranflag;
int\ tchange=1;
float\ tmytime;
newlatt-{}>{}subcycle=1;
int\ ctcycles=0;
while(tchange>{}0\ \&\&\ ctcycles<{}20)
\{
\ \ \ \ \ \ \ \
\ \ \ \ \ \ \ \ \ \ \ ctcycles++;
\ \ \ \ \ \ \ \ \ \ \ if(newlatt-{}>{}myid==1)
\ \ \ \ \ \{
\ \ \ \ \ \ \ \ \ \ \ \textsl{//\ cout<{}<{}"{}id="{}<{}<{}newlatt-{}>{}myid<{}<{}"{}count\ cycles"{}<{}<{}ctcycles<{}<{}endl;}
\ \ \ \ \ \}
\ \ \ \ \ \ \ \ \ \ \ \ \textsl{/*\ start\ iteration\ \ from\ here\ */}
\ \ \ \ \ \ \ \ \ \ \ \ \textsl{//newlatt-{}>{}undoflag\ \ \ =\ -{}1;}
\ \ \ \ \ \ \ \ \ \ \ \ newlatt-{}>{}redoflag\ \ \ =\ 0;
\ \ \ \ \ \ \ \ \ \ \ \ newlatt-{}>{}tnbdyevent\ =\ 0;
\ \ \ \ \ \ \ \ \ \ \

\ \ \ \ \ \ \ \ \ \ \ \ \textsl{/*\ check\ whether\ new\ iteration\ is\ needed\ */}
\ \ \ \ \ \ \ \ \ \ \ \ if\ (newlatt-{}>{}subcycle\ ==\ 1)\ \{
\ \ \ \ \ \ \ \ \ \ \ \ \ \ \ \ if\ (newlatt-{}>{}bdycountrec[0]\ +\ newlatt-{}>{}bdycountrec[1]\ >{}\ 0)\ \{
\ \ \ \ \ \ \ \ \ \ \ \ \ \ \ \ \ \ \ \ newlatt-{}>{}redoflag\ =\ 1;
\ \ \ \ \ \ \ \ \ \ \ \ \ \ \ \ \ \ \ \ newlatt-{}>{}sorting\underline\ nbevent();
\ \ \ \ \ \ \ \ \ \ \ \ \ \ \ \ \}
\ \ \ \ \ \ \ \ \ \ \ \ \}\ else\ \{
\ \ \ \ \ \ \ \ \ \ \ \ \ \ \ \ newlatt-{}>{}comparebdylist();
\ \ \ \ \ \ \ \ \ \ \ \ \ \ \ \ if\ (newlatt-{}>{}redoflag\ ==\ 1)\ \{
\ \ \ \ \ \ \ \ \ \ \ \ \ \ \ \ \ \ \ \ newlatt-{}>{}savebdylist();
\ \ \ \ \ \ \
\ \ \ \ \ \ \ \ \ \ \ \ \ \ \ \ \ \ \ \ if\ (newlatt-{}>{}bdycountrec[0]\ +\ newlatt-{}>{}bdycountrec[1]\ >{}\ 0)
\ \ \ \ \ \ \ \ \ \ \ \ \ \ \ \ \ \ \ \ \ \ \ \ newlatt-{}>{}sorting\underline\ nbevent();
\ \ \ \ \ \ \ \ \ \ \ \ \ \ \ \ \}
\ \ \ \ \ \ \ \ \ \ \ \ \}

\ \ \ \ \ \ \ \ \ \ \ \ \textsl{/*\ new\ iteration\ is\ needed:\ newlatt-{}>{}redoflag=1\ */}
\ \ \ \ \ \ \ \ \ \ \ \ if\ (newlatt-{}>{}redoflag\ ==\ 1)\ \{
\ \ \ \ \ \ \ \ \ \ \ \ \ \ \ \ newlatt-{}>{}restoreLattice();\ \textsl{/*\ restore\ starting\ configuration\ */}

\ \ \ \ \ \ \ \ \ \ \ \ \ \ \ \ newlatt-{}>{}nupdate\ =\ 0;
\ \ \ \ \ \ \ \ \ \ \ \ \ \ \ \ newlatt-{}>{}time\ \ =\ 0.0;
\ \ \ \ \ \ \ \ \ \ \ \ \ \ \ \ newlatt-{}>{}iran\ \ \ \ =\ 0;
\ \ \ \ \ \ \ \ \ \ \ \ \ \ \ \ newlatt-{}>{}nevent\ \ =\ 0;

\ \ \ \ \ \ \ \ \ \ \ \ \ \ \
\ \ \ \ \ \ \ \ \ \ \ \ \ \
\ \ \ \ \ \ \ \ \ \ \ \ \ \ \ \ newlatt-{}>{}myeventlist[newlatt-{}>{}nevent].ranseq\ =\ newlatt-{}>{}iran;
\ \ \ \ \ \ \ \ \ \ \ \ \ \ \ \ newlatt-{}>{}calctime();

\ \ \ \ \ \ \ \ \ \ \ \ \ \ \ \ \textsl{/*\ save\ numbers\ of\ changes\ */}
\ \ \ \ \ \ \ \ \ \ \ \ \ \ \ \ \textsl{/*\ repeat\ kmc\ event\ :\ update\ buffers\ and\ start\ from\ there*/}
\ \ \ \ \ \ \ \ \ \ \ \ \ \ \ \ \textsl{/*\ newlatt-{}>{}time\ is\ later\ than\ 1st\ boundary\ event\ time\ */}
\ \ \ \ \ \ \ \ \ \ \ \ \ \ \ \ if\ (newlatt-{}>{}time\ >{}\ newlatt-{}>{}T\ \&\&\ newlatt-{}>{}tnbdyevent>{}0)\ \{
\ \ \ \ \ \ \ \ \ \ \ \ \ \ \ \ \ \ \ \ tmytime\ =\ newlatt-{}>{}T+1.0;
\ \ \ \ \ \ \ \ \ \ \ \ \ \ \ \ \ \ \ \ for\ (;\ tmytime\ >{}\ newlatt-{}>{}T\ \&\&\ newlatt-{}>{}nupdate\ <{}\ newlatt-{}>{}tnbdyevent;)\ \{
\ \ \ \ \ \ \ \ \ \ \ \ \ \ \ \ \ \ \ \ \ \ \ \ iranflag\ =\ 0;
\ \ \ \ \ \ \ \ \ \ \ \ \ \ \ \ \ \ \ \ \ \ \ \ newlatt-{}>{}updateBuffer(iranflag);
\ \ \ \ \ \ \ \ \ \ \ \ \ \ \ \ \ \ \ \ \ \ \ \ newlatt-{}>{}myeventlist[newlatt-{}>{}nevent].ranseq\ =\ newlatt-{}>{}iran;
\ \ \ \ \ \ \ \ \ \ \ \ \ \ \ \ \ \ \ \ \ \ \ \ newlatt-{}>{}calctime();
\ \ \ \ \ \ \ \ \ \ \ \ \ \ \ \ \ \ \ \ \ \ \ \ tmytime\ =\ newlatt-{}>{}time;
\ \ \ \ \
\ \ \ \ \ \ \ \ \ \ \ \ \ \ \ \ \ \ \ \ \}
\ \ \ \ \ \ \ \ \ \ \ \ \ \ \ \ \}

\ \ \ \ \ \ \ \ \ \ \ \ \ \ \ \ \textsl{/*\ newlatt-{}>{}time\ is\ earlier\ than\ 1st\ boundary\ event\ time\ */}
\ \ \ \ \ \ \ \ \ \ \ \ \ \ \ \ while\ (newlatt-{}>{}time\ <{}\ newlatt-{}>{}T)\ \{
\ \ \ \ \ \ \ \ \ \ \ \ \ \ \ \ \ \ \ \ if\ (newlatt-{}>{}time\ <{}\ newlatt-{}>{}T)\ \{
\ \ \ \ \ \ \ \ \ \ \ \ \ \ \ \ \ \ \ \ \ \ \ \ if\ (newlatt-{}>{}nupdate\ <{}\ newlatt-{}>{}tnbdyevent)\ \{
\ \ \ \ \ \ \ \ \ \ \ \ \ \ \ \ \ \ \ \ \ \ \ \ \ \ \ \ if\ (newlatt-{}>{}time\ <{}\ newlatt-{}>{}sortbdyevent[newlatt-{}>{}nupdate].t)\ \{
\ \ \ \ \ \ \ \ \ \ \ \ \ \ \ \ \ \ \ \ \ \ \ \ \ \ \ \ \ \ \ \ newlatt-{}>{}doKMC();
\ \ \ \ \ \ \ \ \ \ \ \ \ \ \ \ \ \ \ \ \ \ \ \ \ \ \ \ \}\ else\ \{
\ \ \ \ \ \ \ \ \ \ \ \ \ \ \ \ \ \ \ \ \ \ \ \ \ \ \ \ \ \ \ \ iranflag\ =\ 1;
\ \ \ \ \ \ \ \ \ \ \ \ \ \ \ \ \ \ \ \ \ \ \ \ \ \ \ \ \ \ \ \ newlatt-{}>{}updateBuffer(iranflag);
\ \ \ \ \ \ \ \ \ \ \ \ \ \ \ \ \ \ \ \ \ \ \ \ \ \ \ \ \}
\ \ \ \ \ \ \ \ \ \ \ \ \ \ \ \ \ \ \ \ \ \ \ \ \}\ else\ \{
\ \ \ \ \ \ \ \ \ \ \ \ \ \ \ \ \ \ \ \ \ \ \ \ \ \ \ \ newlatt-{}>{}doKMC();
\ \ \ \ \ \ \ \ \ \ \ \ \ \ \ \ \ \ \ \ \ \ \ \ \}
\ \ \ \ \ \ \ \ \ \ \ \ \ \ \ \ \ \ \ \ \}
\ \ \ \ \ \ \ \ \ \ \ \ \ \ \ \ \ \ \ \ newlatt-{}>{}myeventlist[newlatt-{}>{}nevent].ranseq\ =\ newlatt-{}>{}iran;
\ \ \ \ \ \ \ \ \ \ \ \ \ \ \ \ \ \ \ \ newlatt-{}>{}calctime();
\ \ \ \ \ \ \ \ \ \ \ \ \ \ \ \ \ \ \ \ tmytime\ =\ newlatt-{}>{}time;
\ \ \ \ \ \ \ \ \ \ \ \ \ \ \ \ \ \ \ \ for\ (;tmytime\ >{}\ newlatt-{}>{}T\ \&\&\ newlatt-{}>{}nupdate\ <{}\ newlatt-{}>{}tnbdyevent;)\ \{
\ \ \ \ \ \ \ \ \ \ \ \ \ \ \ \ \ \ \ \ \ \ \ \ iranflag\ =\ 0;
\ \ \ \ \ \ \ \ \ \ \ \ \ \ \ \ \ \ \ \ \ \ \ \ newlatt-{}>{}updateBuffer(iranflag);
\ \ \ \ \ \ \ \ \ \ \ \ \ \ \ \ \ \ \ \ \ \ \ \ newlatt-{}>{}myeventlist[newlatt-{}>{}nevent].ranseq\ =\ newlatt-{}>{}iran;
\ \ \ \ \ \ \ \ \ \ \ \ \ \ \ \ \ \ \ \ \ \ \ \ newlatt-{}>{}calctime();
\ \ \ \ \ \ \ \ \ \ \ \ \ \ \ \ \ \ \ \ \ \ \ \ tmytime\ =\ newlatt-{}>{}time;
\ \ \ \ \ \ \ \ \ \ \ \ \ \ \ \ \ \ \ \ \}
\ \ \ \ \ \ \ \ \textsl{//cout<{}<{}"{}I\ am"{}<{}<{}newlatt-{}>{}time<{}<{}endl;}
\ \ \ \ \ \ \ \ \textsl{//cout<{}<{}"{}subcycle************************"{}<{}<{}newlatt-{}>{}redoflag<{}<{}"{}flag****************\ "{}<{}<{}ctr<{}<{}endl;}
\ \ \ \ \ \ \ \ \ \ \ \ \ \ \ \ \}
\ \ \ \ \ \ \ \ \ \ \ \ \}
\ \ \ \ \ \ \ \ \ \ \
\ \ \ \ \ \ \ \ \ \ \ \ \ \textsl{/*\ check\ how\ many\ processors\ have\ a\ change\ in\ the\ previous\ events\ */}
\ \ \ \ \ \ \ \ \ \ \ \ newlatt-{}>{}subcycle++;\ \textsl{/*\ increase\ number\ of\ iteration\ */}

\ \ \ \ \ \ \textsl{/*\ check\ how\ many\ processors\ were\ redone\ */}
\ \ \ \ \ \ \ \ \ \ \ \ mpi.allReduce(\&newlatt-{}>{}redoflag,\&tchange,1,MPI\underline\ INT,MPI\underline\ SUM);
\ \ \ \

\ \ \ \ \ \ \ \ if\ (tchange\ >{}\ 0)\ \{\ \textsl{/*\ some\ processors\ are\ unhappy:\ redo\ must\ be\ needed\ */}
\ \ \ \ \ \ \ \ \ \ \ \ \ \ \ \ \ \ \ \ \ \ \ \ newlatt-{}>{}bdycountrec[left]=0;
\ \ \ \ \ \ \ \ \ \ \ \ \ \ \ \ \ \ \ \ \ \ \ \ newlatt-{}>{}bdycountrec[right]=0;
\ \ \ \ \ \ \ \ \ \ \ \ \ \ \ \ \ \ \ \ \ \ \ \ \ sendmsgs(newlatt);\ \ \ \ \
\ \ \ \ \ \ \ \ \ \ \ \ \}
\}

\}



 }
\normalfont\normalsize


\end{code}

% COMM SOURCE FILES
\section{comm.cpp}
\begin{code}{comm.cpp}{sr/comm.cpp}
{\ttfamily \raggedright \footnotesize
\#include\ "{}lattice.h"{}
extern\ MPIWrapper\ mpi;
void\ sendmsgs(Lattice\ \ *\ newlatt)
\{

\ \ mpi.sendboundaryevent(newlatt-{}>{}bdyevent[left],newlatt-{}>{}bdycount[left],newlatt-{}>{}nbhr[left]);
\ \ mpi.sendboundaryevent(newlatt-{}>{}bdyevent[right],newlatt-{}>{}bdycount[left],newlatt-{}>{}nbhr[right]);

\ \ mpi.recvboundaryevent(newlatt-{}>{}bdyeventrec[left],newlatt-{}>{}nbhr[left]);
\ \ mpi.recvboundaryevent(newlatt-{}>{}bdyeventrec[right],newlatt-{}>{}nbhr[right]);

\ \ for(int\ j=0;j<{}2;j++)
\ \ \{
\ \ \ \ newlatt-{}>{}bdycount[j]=0;
\ \ \}
\}

 }
\normalfont\normalsize


\end{code}
\mychapter{Time Warp Code}

% MAIN SOURCE FILE
\section{main.cpp}
\begin{code}{main.cpp}{tw/main.cpp}
{\ttfamily \raggedright \footnotesize
\#include\ <{}iostream>{}
using\ std::cout;
using\ std::cerr;
using\ std::endl;
\ \\
\#include\ <{}fstream>{}
using\ std::ofstream;
\ \\
\#include\ "{}exception.h"{}
\#include\ "{}lattice.h"{}
\#include\ "{}mpiwrapper.h"{}
\ \\
const\ int\ globalSyncThreshold\ =\ 1000;
\ \\
string\ makeFileName(string,string,int);
\ \\
int\ main(int\ argc,\ char* \ argv[])\ \{
\ \ Lattice\ lattice;
\ \ string\ pngFilename\ =\ "{}"{};
\ \ string\ logFilename\ =\ "{}"{};
\ \ fstream\ logFile;
\ \ double\ minGlobalTime\ =\ 0.0;
\ \ double\ maxGlobalTime\ =\ 0.0;
\ \ int\ globalTimeCounter\ =\ 0;
\ \ double\ gConvergence\ =\ 0.0;
\ \ int\ eventCount\ =\ 0;
\ \\
\ \ \textsl{//\ setup\ the\ lattice\ mpi\ stuff}
\ \ lattice.mpi.init(\&argc,\&argv);
\ \\
\ \ try\ \{
\ \\
\ \ \ \ pngFilename\ =\ makeFileName("{}height-{}node"{},"{}png"{},lattice.mpi.getRank());
\ \ \ \ logFilename\ =\ makeFileName("{}log"{},"{}txt"{},lattice.mpi.getRank());
\ \\
\ \ \ \ logFile.open(logFilename.c\underline\ str(),fstream::out|fstream::trunc);
\ \\
\ \ \ \ if(!logFile)\ \{
\ \ \ \ \ \ string\ error\ =\ "{}Couldn't\ open\ log\ file\ "{}\ +\ logFilename;
\ \ \ \ \ \ throw(Exception(error));
\ \ \ \ \}
\ \\
\ \ \ \ lattice.setMinGlobalTime(0.0);
\ \\
\ \ \ \ lattice.mpi.barrier();
\ \\
\ \ \ \ \textsl{//\ MAIN\ LOOP}
\ \ \ \ while(gConvergence\ <{}\ 1.0)\ \{
\ \ \ \ \ \ \textsl{//\ retrive\ any\ remote\ events}
\ \ \ \ \ \ lattice.negoitateEvents(logFile);
\ \\
\ \ \ \ \ \ \textsl{//\ do\ the\ next\ event}
\ \ \ \ \ \ lattice.doNextEvent();
\ \ \ \ \ \
\ \ \ \ \ \ \textsl{//\ see\ if\ it's\ time\ for\ a\ global\ sync}
\ \ \ \ \ \ if(globalTimeCounter\ >{}\ globalSyncThreshold)\ \{
\ \ \ \ \ \ \ \ lattice.mpi.barrier();
\ \\
\ \ \ \ \ \ \ \ lattice.negoitateEvents(logFile);
\ \\
\ \ \ \ \ \ \ \ \textsl{//\ allreduce\ to\ find\ the\ min\ time}
\ \ \ \ \ \ \ \ minGlobalTime\ =\ lattice.mpi.allReduceDouble(lattice.getLocalTime(),MPI\underline\ MIN);
\ \ \ \ \ \ \ \ maxGlobalTime\ =\ lattice.mpi.allReduceDouble(lattice.getLocalTime(),MPI\underline\ MAX);
\ \ \ \ \ \ \ \ eventCount\ =\ lattice.mpi.allReduceInt(lattice.getEventCount(),MPI\underline\ SUM);
\ \\
\ \ \ \ \ \ \ \ \textsl{//\ set\ the\ global\ time\ in\ the\ lattice}
\ \ \ \ \ \ \ \ lattice.setMinGlobalTime(minGlobalTime);
\ \\
\ \ \ \ \ \ \ \ \textsl{//\ clear\ the\ counter}
\ \ \ \ \ \ \ \ globalTimeCounter\ =\ 0;
\ \\
\ \ \ \ \ \ \ \ \textsl{//\ calculate\ the\ global\ convergence}
\ \ \ \ \ \ \ \ gConvergence\ =\ (double)eventCount/(double)(lattice.mpi.getNodeCount()\ *\ SIZE);
\ \\
\ \ \ \ \ \ \ \ if(lattice.mpi.isRoot())\ \{
\ \ \ \ \ \ \ \ \ \ cout\ <{}<{}\ minGlobalTime\ <{}<{}\ "{}\ "{}\ <{}<{}\ maxGlobalTime\ <{}<{}\ "{}\ "{}\ <{}<{}\ gConvergence\ <{}<{}\ endl;
\ \ \ \ \ \ \ \ \ \ cout.flush();
\ \ \ \ \ \ \ \ \}
\ \ \ \ \ \ \}\ else
\ \ \ \ \ \ \ \ ++globalTimeCounter;
\ \ \ \ \}
\ \\
\ \ \ \ logFile\ <{}<{}\ "{}exit\ main\ loop"{}\ <{}<{}\ endl;
\ \ \ \ logFile.flush();
\ \\
\ \ \ \ lattice.mpi.barrier();
\ \\
\ \ \ \ \textsl{//\ rollback\ to\ minimum\ global\ time}
\ \\
\ \ \ \ \textsl{//lattice.printLatticeHeight(logFile);}
\ \ \ \ logFile\ <{}<{}\ "{}gCovergence\ =\ "{}\ <{}<{}\ gConvergence\ <{}<{}\ endl;
\ \ \ \ lattice.printStats(logFile);
\ \ \ \ lattice.createHeightMap(pngFilename);
\ \ \ \ lattice.mpi.printStats(logFile);
\ \\
\ \ \ \ lattice.cleanup(logFile);
\ \\
\ \ \ \ logFile.close();
\ \\
\ \ \ \ lattice.mpi.barrier();
\ \\
\ \ \}\ catch(Exception\ err)\ \{
\ \ \ \ cerr\ <{}<{}\ err.error\ <{}<{}\ endl;
\ \ \}
\ \\
\ \ \textsl{//\ close\ the\ mpi\ stuff}
\ \ lattice.mpi.shutdown();
\ \ return(0);
\}
\ \\
string\ makeFileName(string\ prefix,\ string\ ext,\ int\ rank)\ \{
\ \ string\ output\ =\ prefix\ +\ "{}."{};
\ \ output\ +=\ (char)('a'\ +\ rank);
\ \ return(output\ +\ "{}."{}\ +\ ext);
\}
\ \\
 }
\normalfont\normalsize


\end{code}

% LATTICE SOURCE FILES
\section{lattice.h}
\begin{code}{lattice.h}{tw/lattice.h}
{\ttfamily \raggedright \small
\#include\ <{}vector>{}\\
using\ std::vector;\\
\ \\
\#include\ <{}queue>{}\\
using\ std::priority\underline\ queue;\\
\ \\
\#include\ <{}stack>{}\\
using\ std::stack;\\
\ \\
\#include\ <{}iomanip>{}\\
using\ std::setw;\\
using\ std::hex;\\
using\ std::dec;\\
using\ std::setprecision;\\
\ \\
\#include\ <{}fstream>{}\\
using\ std::fstream;\\
\ \\
\#include\ <{}string>{}\\
using\ std::string;\\
\ \\
\#include\ <{}png.h>{}\\
\ \\
\#include\ "{}exception.h"{}\\
\#include\ "{}latprim.h"{}\\
\#include\ "{}latconst.h"{}\\
\#include\ "{}event.h"{}\\
\#include\ "{}randgen.h"{}\\
\#include\ "{}rewindlist.h"{}\\
\#include\ "{}mpiwrapper.h"{}\\
\ \\
\#ifndef LATTICE\underline\ H\\
\#define LATTICE\underline\ H\\
\ \\
\#define GET\underline\ DIR(a)\ ((a\ <{}\ LEFT\underline\ X\underline\ BOUNDRY)\ ?\ LEFT\ :\ RIGHT)\\
\ \\
class\ Lattice\ \{\\
public:\\
\ \ Lattice();\\
\ \ \textasciitilde Lattice();\\
\ \\
\ \ void\ cleanup(fstream\&);\\
\ \\
\ \ bool\ doNextEvent();\\
\ \\
\ \ double\ getLocalTime()\ \{\\
\ \ \ \ return(localTime);\\
\ \ \}\\
\ \\
\ \ bool\ setMinGlobalTime(double\ mGT)\ \{\\
\ \ \ \ minGlobalTime\ =\ mGT;\\
\ \ \ \ return(true);\\
\ \ \}\\
\ \\
\ \ double\ getMinGlobalTime()\ \{\\
\ \ \ \ return(minGlobalTime);\\
\ \ \}\\
\ \\
\ \ bool\ negoitateEvents(fstream\&);\\
\ \\
\ \ \textsl{//\ DEBUG\ FUNCTIONS}\\
\ \ void\ printLatticeHeight(fstream\&\ file)\ \{\\
\ \ \ \ for(int\ i=0;\ i\ <{}\ DIM\underline\ X\ +\ GHOST\ +\ GHOST;\ ++i)\ \{\\
\ \ \ \ \ \ for(int\ j=0;\ j\ <{}\ DIM\underline\ Y;\ ++j)\ \{\\
\ \ \ \ \ \ \ \ file\ <{}<{}\ lattice[i][j].h\ <{}<{}\ "{}\ "{};\\
\ \ \ \ \ \ \}\\
\ \ \ \ \ \ file\ <{}<{}\ endl;\\
\ \ \ \ \}\\
\ \ \ \ file\ <{}<{}\ "{}-{}-{}-{}-{}-{}-{}-{}-{}-{}-{}-{}-{}-{}-{}-{}-{}-{}-{}-{}-{}-{}-{}-{}-{}-{}-{}-{}-{}-{}-{}-{}-{}-{}-{}-{}-{}-{}-{}-{}-{}-{}-{}-{}-{}-{}-{}-{}-{}"{}\ <{}<{}\ endl\ <{}<{}\ endl;\\
\ \ \ \ file.flush();\\
\ \ \}\\
\ \\
\ \ void\ printLatticeIndex(fstream\&\ file)\ \{\\
\ \ \ \ for(int\ i=0;\ i\ <{}\ DIM\underline\ X\ +\ GHOST\ +\ GHOST;\ ++i)\ \{\\
\ \ \ \ \ \ for(int\ j=0;\ j\ <{}\ DIM\underline\ Y;\ ++j)\ \{\\
\ \ \ \ \ \ \ \ if(lattice[i][j].listIndex\ >{}=\ 0)\\
\ \ \ \ \ \ \ \ \ \ file\ <{}<{}\ setw(4)\ <{}<{}\ lattice[i][j].listIndex;\\
\ \ \ \ \ \ \ \ else\\
\ \ \ \ \ \ \ \ \ \ file\ <{}<{}\ setw(4)\ <{}<{}\ "{}x"{};\\
\ \ \ \ \ \ \ \ file\ <{}<{}\ "{}\ "{};\\
\ \ \ \ \ \ \}\\
\ \ \ \ \ \ file\ <{}<{}\ endl;\\
\ \ \ \ \}\\
\ \ \ \ file\ <{}<{}\ "{}-{}-{}-{}-{}-{}-{}-{}-{}-{}-{}-{}-{}-{}-{}-{}-{}-{}-{}-{}-{}-{}-{}-{}-{}-{}-{}-{}-{}-{}-{}-{}-{}-{}-{}-{}-{}-{}-{}-{}-{}-{}-{}-{}-{}-{}-{}-{}-{}"{}\ <{}<{}\ endl\ <{}<{}\ endl;\\
\ \ \ \ file.flush();\\
\ \ \}\\
\ \\
\ \ void\ printMonomerList(fstream\&\ file)\ \{\\
\ \ \ \ file\ <{}<{}\ "{}monomerList["{}\ <{}<{}\ monomerList.size()\ <{}<{}\ "{}]\ at\ time="{}\ <{}<{}\ localTime\ <{}<{}\ endl;\\
\ \ \ \ for(int\ i=0;\ i\ <{}\ monomerList.size();\ ++i)\ \{\\
\ \ \ \ \ \ site$\ast$\ s\ =\ monomerList[i];\\
\ \ \ \ \ \ file\ <{}<{}\ i\ <{}<{}\ "{}:\ ("{}\ <{}<{}\ s-{}>{}p.x\ <{}<{}\ "{},"{}\ <{}<{}\ s-{}>{}p.y\ <{}<{}\ "{})\ h="{}\ <{}<{}\ s-{}>{}h\ <{}<{}\ "{}\ listIndex="{}\ <{}<{}\ s-{}>{}listIndex\ <{}<{}\ "{}\ "{}\ <{}<{}\ hex\ <{}<{}\ s\ <{}<{}\ dec\ <{}<{}\ endl;\\
\ \ \ \ \}\\
\ \ \ \ file\ <{}<{}\ "{}-{}-{}-{}-{}-{}-{}-{}-{}-{}-{}-{}-{}-{}-{}-{}-{}-{}-{}-{}-{}-{}-{}-{}-{}-{}-{}-{}-{}-{}-{}-{}-{}-{}-{}-{}-{}-{}-{}-{}-{}-{}-{}-{}-{}-{}-{}-{}-{}"{}\ <{}<{}\ endl\ <{}<{}\ endl;\\
\ \ \ \ file.flush();\\
\ \ \}\\
\ \\
\ \ void\ printStats(fstream\&\ file)\ \{\\
\ \ \ \ file\ <{}<{}\ setprecision(10)\ <{}<{}\ endl;\\
\ \ \ \ file\ <{}<{}\ "{}COLLECTED\ STATISTICS"{}\ <{}<{}\ endl;\\
\ \ \ \ file\ <{}<{}\ "{}-{}-{}-{}-{}-{}-{}-{}-{}-{}-{}-{}-{}-{}-{}-{}-{}-{}-{}-{}-{}-{}-{}"{}\ <{}<{}\ endl;\\
\ \ \ \ file\ <{}<{}\ "{}Convergence\ =\ "{}\ <{}<{}\ getConvergence()\ <{}<{}\ endl;\\
\ \ \ \ file\ <{}<{}\ "{}Total\ Event\ Count\ =\ "{}\ <{}<{}\ countEvents\ <{}<{}\ endl;\\
\ \ \ \ file\ <{}<{}\ "{}Total\ Diffusion\ Events\ =\ "{}\ <{}<{}\ countDiffusion\ <{}<{}\ endl;\\
\ \ \ \ file\ <{}<{}\ "{}Total\ Deposition\ Events\ =\ "{}\ <{}<{}\ (countEvents\ -{}\ countDiffusion)\ <{}<{}\ endl;\\
\ \ \ \ file\ <{}<{}\ "{}Total\ Boundry\ Events\ =\ "{}\ <{}<{}\ countBoundry\ <{}<{}\ endl;\\
\ \ \ \ file\ <{}<{}\ "{}Total\ Number\ Remote\ Events\ =\ "{}\ <{}<{}\ countRemote\ <{}<{}\ endl;\\
\ \ \ \ file\ <{}<{}\ "{}Total\ Rollbacks\ Performed\ =\ "{}\ <{}<{}\ countRollback\ <{}<{}\ endl;\\
\ \ \ \ file\ <{}<{}\ "{}Monomer\ List\ Size\ =\ "{}\ <{}<{}\ monomerList.size()\ <{}<{}\ endl;\\
\ \ \ \ file\ <{}<{}\ "{}Local\ Time\ =\ "{}\ <{}<{}\ localTime\ <{}<{}\ endl;\\
\ \ \ \ file\ <{}<{}\ "{}Min\ Global\ Time\ =\ "{}\ <{}<{}\ minGlobalTime\ <{}<{}\ endl;\\
\ \ \ \ file\ <{}<{}\ "{}Size\ =\ "{}\ <{}<{}\ SIZE\ <{}<{}\ "{}\ DIM\underline\ X\ =\ "{}\ <{}<{}\ DIM\underline\ X\ <{}<{}\ "{}\ DIM\underline\ Y\ =\ "{}\ <{}<{}\ DIM\underline\ Y\ <{}<{}\ endl;\\
\ \ \ \ file\ <{}<{}\ "{}-{}-{}-{}-{}-{}-{}-{}-{}-{}-{}-{}-{}-{}-{}-{}-{}-{}-{}-{}-{}-{}-{}-{}-{}-{}-{}-{}-{}-{}-{}-{}-{}-{}-{}-{}-{}-{}-{}-{}-{}-{}-{}-{}-{}-{}-{}-{}-{}"{}\ <{}<{}\ endl\ <{}<{}\ endl;\\
\ \ \ \ file.flush();\\
\ \ \}\\
\ \\
\ \ int\ getEventCount()\ \{\ return(countEvents\ -{}\ countDiffusion);\ \}\\
\ \\
\ \ bool\ createHeightMap(string\ filename);\\
\ \\
\ \ MPIWrapper\ mpi;\ \ \textsl{//\ I\ can't\ think\ of\ a\ better\ place\ for\ this.}\\
\ \\
\ \ bool\ rollback(const\ double);\\
\ \\
\ \ double\ getConvergence()\ \{\\
\ \ \ \ return((double)(countEvents\ -{}\ countDiffusion)\ /\ (double)(DIM\underline\ X\ $\ast$\ DIM\underline\ Y));\\
\ \ \}\\
\ \\
private:\\
\ \ double\ computeTime();\\
\ \ bool\ deposit();\\
\ \ bool\ diffuse();\\
\ \ bool\ doKMC();\\
\ \ EventType\ getNextEventType();\\
\ \ bool\ commitEvent(Event$\ast$);\\
\ \ site$\ast$\ randomMove(site$\ast$);\\
\ \ bool\ isBoundry(point);\\
\ \ bool\ isBound(site$\ast$);\\
\ \ bool\ clearBonded(site$\ast$,const\ double);\\
\ \ bool\ translateMessages(vector<{}Event$\ast$>{}$\ast$\ ,\ vector<{}message>{}$\ast$);\\
\ \ message$\ast$\ makeMessage(Event$\ast$);\\
\ \ bool\ hasAntiEvent(Event$\ast$);\\
\ \\
\ \ double\ localTime;\ \ \ \ \textsl{//\ the\ time\ local\ to\ the\ lattice}\\
\ \ double\ minGlobalTime;\ \textsl{//\ the\ minimum\ Global\ time\ (point\ of\ no\ return)}\\
\ \\
\ \ RewindList<{}site\ $\ast$>{}\ monomerList;\ \textsl{//\ list\ of\ all\ unbound\ monomers}\\
\ \ \textsl{//site\ lattice[DIM\underline\ X\ +\ GHOST\ +\ GHOST][DIM\underline\ Y];\ \ //\ the\ lattice\ (the\ extra\ two\ are\ the\ ghost\ region)}\\
\ \ site$\ast$$\ast$\ lattice;\\
\ \\
\ \ float\ depositionRate;\ \textsl{//\ the\ deposition\ rate\ of\ monomers}\\
\ \ float\ diffusionRate;\ \ \textsl{//\ the\ diffusion\ rate\ of\ monomers}\\
\ \\
\ \ int\ countDiffusion;\\
\ \ int\ countEvents;\\
\ \ int\ countBoundry;\\
\ \ int\ countRemote;\\
\ \ int\ countRollback;\\
\ \\
\ \ priority\underline\ queue<{}Event$\ast$>{}\ remoteEventList;\ \textsl{//\ list\ of\ all\ the\ remote\ dep/diffusion\ events}\\
\ \ stack<{}Event$\ast$>{}\ eventList;\ \ \ \ \ \ \ \ \ \ \ \ \ \ \ \ \textsl{//\ stack\ of\ all\ events\ to\ rollback\ the\ simulation}\\
\ \ vector<{}Event$\ast$>{}\ antiEvents;\ \ \ \ \ \ \ \ \ \ \ \ \ \ \textsl{//\ list\ of\ anti-{}events\ that\ will\ occur\ in\ the\ future}\\
\ \\
\ \ RandGen\ rng;\ \textsl{//\ random\ number\ generator}\\
\ \\
\ \ point\ movementDir[NUM\underline\ DIR];\ \textsl{//\ array\ of\ movement\ types}\\
\ \ message\ m;\ \textsl{//\ message\ for\ sending\ events}\\
\};\\
\ \\
\#endif\\
\ \\
 }
\normalfont\normalsize


\end{code}

\section{latconst.h}
\begin{code}{latconst.h}{tw/latconst.h}
{\ttfamily \raggedright \small
\#ifndef LATCONST\underline\ H\\
\#define LATCONST\underline\ H\\
\ \\
\textsl{//\ lattice\ dimensions\ and\ area}\\
\textsl{//const\ int\ DIM\underline\ X\ =\ 4096;}\\
\textsl{//const\ int\ DIM\underline\ X\ =\ 2048;}\\
const\ int\ DIM\underline\ X\ =\ 1024;\ \ \ \ \ \ \ \ \ \ \ \ \ \ \ \ \ \ \ \ \ \ \textsl{//\ the\ total\ X\ dimension\ of\ the\ lattice}\\
\textsl{//const\ int\ DIM\underline\ X\ =\ 512;\ \ \ \ \ \ \ \ \ \ \ \ \ \ \ \ \ \ \ \ \ \ //\ the\ total\ X\ dimension\ of\ the\ lattice}\\
\textsl{//const\ int\ DIM\underline\ Y\ =\ 1024;\ \ \ \ \ \ \ \ \ \ \ \ \ \ \ \ \ \ \ \ \ \ \ //\ the\ total\ Y\ dimension\ of\ the\ lattice}\\
const\ int\ DIM\underline\ Y\ =\ 256;\\
const\ int\ GHOST\ =\ 1;\ \ \ \ \ \ \ \ \ \ \ \ \ \ \ \ \ \ \ \ \ \ \ \ \ \textsl{//\ the\ size\ of\ the\ ghost\ region}\\
const\ int\ LEFT\underline\ X\underline\ BOUNDRY\ =\ GHOST;\ \ \ \ \ \ \ \ \ \ \ \ \textsl{//\ the\ left\ boundry}\\
const\ int\ RIGHT\underline\ X\underline\ BOUNDRY\ =\ DIM\underline\ X\ -{}\ GHOST\ -{}\ 1;\ \ \ \textsl{//\ the\ right\ boundry}\\
const\ int\ SIZE\ =\ DIM\underline\ X\ $\ast$\ DIM\underline\ Y;\ \ \ \ \ \ \ \ \ \ \ \ \ \ \textsl{//\ the\ area\ (number)\ of\ sites\ in\ the\ lattice}\\
\ \\
\textsl{//\ enviroment\ attributes}\\
const\ int\ NUM\underline\ DIR\ =\ 8;\ \textsl{//\ number\ of\ directions}\\
\ \\
\textsl{//\ minimum\ local\ time}\\
const\ double\ MINIMUM\underline\ TIME\ =\ 0.001;\\
\ \\
\textsl{//\ random\ number\ vars}\\
const\ int\ NUMBER\underline\ START\ =\ 10000;\ \textsl{//\ the\ number\ of\ random\ numbers\ to\ start\ with}\\
const\ int\ SEED\ =\ 0;\ \ \ \ \ \ \ \ \ \ \ \ \ \textsl{//\ the\ initial\ seed}\\
\ \\
\#endif\\
\ \\
\ \\
 }
\normalfont\normalsize


\end{code}

\section{latprim.h}
\begin{code}{latprim.h}{tw/latprim.h}
{\ttfamily \raggedright \footnotesize
\#ifndef LATPRIM\underline\ H
\#define LATPRIM\underline\ H

typedef\ struct\ \{
\ \ int\ x;
\ \ int\ y;
\}\ point;

typedef\ struct\ \{
\ \ point\ p;
\ \ int\ h;
\ \ int\ listIndex;
\}\ site;

typedef\ point*\ pointPTR;\ \textsl{//\ pointer\ to\ a\ point}
typedef\ site*\ sitePTR;\ \ \ \textsl{//\ pointer\ to\ a\ site}

\#endif

 }
\normalfont\normalsize


\end{code}

\section{lattice.cpp}
\begin{code}{lattice.cpp}{tw/lattice.cpp}
{\ttfamily \raggedright \footnotesize
\#include\ "{}lattice.h"{}
point\ Lattice::ranmove(site\ mysite)
\{

\ \ \ \ point\ pt;
\ \ \ \ pt.x=0;pt.y=0;

\ \ \ \ \textsl{////cout<{}<{}xdir<{}<{}"{}\ \ "{};\ \ }
\ \ \ \
\ \ \ \ int\ prob=ranlist[iran]*4;\textsl{//rand()\%4;}
\ \ \ \ iran++;
\ \ \ \
\ \ \ \
\ \ \ \ pt.x=mysite.pos.x+mdir[prob].x;
\ \ \ \ pt.y=mysite.pos.y+mdir[prob].y;

\ \ \ \ \textsl{//Allow\ boundary\ movement}
\ \ \ \ \textsl{//No!!!}
\ \ \ \
\ \ \ \ if(pt.x>{}=size+2)\ pt.x-{}=1;
\ \ \ \ if(pt.x<{}0)pt.x+=1;
\ \ \ \
\ \ \ \
\ \ \ \ if(pt.y>{}=size)pt.y-{}=2;
\ \ \ \ if(pt.y<{}0)pt.y+=2;
\ \ \ \
\ \ \ \ return\ pt;
\}

Lattice::Lattice()
\{

\ \ \ \ float\ ratio=0.0;
\ \ \ \ float\ prob;
\ \ \ \ point\ newsite;
\ \ \ \ mcount=0;
\ \ \ \ deprate=1.0f,difrate=1.0e3;
\ \ \ \ ndep=0;
\ \ \ \ nevent=0;
\ \ \ \ time=0;
\ \ \ \ iran=0;
\ \ \ \ subcycle=0;
\ \ \ \ T=0.001;

\ \ \ \ int\ i=0,j=0;
\ \ \ \
\ \ \ \ for\ (i=0;i<{}2;i++)
\ \ \ \ \ \ \ \ bdycount[i]=0;

\ \ \ \ for\ (i=0;i<{}2;i++)
\ \ \ \ \ \ bdycountrec[i]=0;

\ \ \ \ for\ (i=0;i<{}2;i++)
\ \ \ \ \ \ oldbdycountrec[i]=0;

\ \ \ \ for\ (i=0;i<{}size+2;i++)
\ \ \ \ \{
\ \ \ \ for\ (j=0;j<{}size;j++)
\ \ \ \ \ \ \{
\ \ \ \ \ \ \ \ newsite.x=i;
\ \ \ \ \ \ \ \ newsite.y=j;
\ \ \ \ \ \ \ \ location[i][j].pos=newsite;
\ \ \ \ \ \ \ \ location[i][j].h=0;
\ \ \ \ \ \ location[i][j].index=-{}1;
\ \ \ \ \ \ \}
\ \ \ \ \}
\ \ \ \
\ \ \ \
\ \ \ \ \textsl{//generate\ random\ numbers}
\ \ \ \
\ \ \ \ \textsl{//randgen();}
\ \ \ \
\ \ \ \ \textsl{/*
\ \ \ \ generate\ points\ in\ the\ form\ e.g\ (1,-{}1)\ move\ one\ unit\ left\ in\ x\ and\ and\ one\ unit\ up
\ \ \ \ do\ not\ allow\ particle\ to\ remain\ (0,0)
\ \ \ \ */}


\ \ \ \ \textsl{/*Set\ directions*/}



\ \ \ \ mdir[0].y=\ 1;\ \ mdir[0].x=\ 0;
\ \ \ \ mdir[1].y=\ 0;\ \ mdir[1].x=\ 1;
\ \ \ \ mdir[2].y=-{}1;\ \ mdir[2].x=\ 0;
\ \ \ \ mdir[3].y=\ 0;\ \ mdir[3].x=-{}1;
\ \ \ \ mdir[4].y=\ 1;\ \ mdir[4].x=\ 1;
\ \ \ \ mdir[5].y=\ 1;\ \ mdir[5].x=-{}1;
\ \ \ \ mdir[6].y=-{}1;\ \ mdir[6].x=\ 1;
\ \ \ \ mdir[7].y=-{}1;\ \ mdir[7].x=-{}1;

\}
void\ Lattice::calctime()
\{
\ \ float\ \ \ monomer=(float)mcount,
\ \ \ \ Drate=difrate*monomer*0.25f,
\ \ \ \ N=(float)latsize,dt,prob,
\ \ \ \ totaldep=deprate*N;
\ \ \
\ \ prob\ =\ ranlist[iran];
\ \ \ \ \ \ \ \ iran++;
\ \ dt=-{}log(prob)/(Drate+totaldep);
\ \ time+=dt;
\}
Lattice::\textasciitilde Lattice()
\{

\}

int\ Lattice::getbonds(site\ mysite,point\ *\ bondpt)
\{
int\ ctr=0,i=0;
point\ pt;
pt=mysite.pos;
for(;i<{}dir;i++)
\{
\ \ \ pt.x+=mdir[i].x;
\ \ \ pt.y+=mdir[i].y;
\ \ \ if((pt.x<{}0)
\ \ \ \ \ \ ||(pt.x>{}=size+2)
\ \ \ \ \ \ ||(pt.y<{}0)
\ \ \ \ \ \ ||(pt.y>{}=size))
\ \ \{
\
\ \ \}
\ \ else
\ \ \{
\ \ \ \ \ if(location[pt.x][pt.y].h>{}=mysite.h)
\ \ \ \ \ \{
\ \ \ \ \ \ \ bondpt[ctr]=pt;
\ \ \ \ ctr++;
\ \ \ \ \ \}
\ \ \ \}
\ \ \ pt=mysite.pos;
\}
return\ ctr;
\}

int\ Lattice::getnbhrs(site\ mysite,point\ *\ bondpt)
\{
\ \ \ \ int\ ctr=0,i=0;
\ \ \ \ point\ pt;
\ \ \ \ pt=mysite.pos;
\ \ \ \ for(;i<{}dir;i++)
\ \ \ \ \{
\ \ \ \ pt.x+=mdir[i].x;
\ \ \ \ pt.y+=mdir[i].y;
\ \ \ \ if((pt.x<{}0)
\ \ \ \ \ \ ||(pt.x>{}=size+2)
\ \ \ \ \ \ ||(pt.y<{}0)
\ \ \ \ \ \ ||(pt.y>{}=size))
\ \ \ \ \{
\ \ \ \ \
\ \ \ \ \}
\ \ \ \ else
\ \ \ \ \{
\ \ \ \ \ \ \ \ \ bondpt[ctr]=pt;
\ \ \ \ \ \ ctr++;
\ \ \ \ \ \ \ \
\ \ \ \ \}
\ \ \ \ pt=mysite.pos;
\ \ \ \ \}
\ \ \ \ return\ ctr;
\}

void\ Lattice::deletemonomer(point\ pos)
\{
\ \ point\ lastpt;
\ \ if(mcount>{}1)
\ \ \{
\ \ \ \ lastpt=monomerloc[mcount-{}1];
\ \ \ \ addmonomerchange(diff,lastpt);
\ \ \ \ location[lastpt.x][lastpt.y].index=location[pos.x][pos.y].index;
\ \ \ \ monomerloc[location[pos.x][pos.y].index]=monomerloc[mcount-{}1];
\ \ \ \ location[pos.x][pos.y].index=-{}1;
\ \ \ \ mcount-{}-{};
\ \ \}
\ \ else
\ \ \{
\ \ \ \ location[pos.x][pos.y].index=-{}1;
\ \ \ \ mcount-{}-{};
\ \ \}
\}

bool\ Lattice::neighborIsMonomer(point\ pt,site\ mysite)
\{
\ \ bool\ bMonomer=false;
\ \ \textsl{//if\ index\ is\ not\ -{}1\ and\ same\ height\ as\ my\ location}
\ \ if\ (location[pt.x][pt.y].index!=-{}1\ \&\&\ (location[pt.x][pt.y].h==location[mysite.pos.x][mysite.pos.y].h))
\ \ \ \ \{
\ \ \ \ \ \ bMonomer=true;\ \ \ \
\ \ \ \ \}
\ \ return\ bMonomer;
\}
bool\ Lattice::checkupdatebonds(site\ mysite)
\{
\ \ \textsl{/*
\ \ Possible\ Scenarios
\ \ Deposit
\ \ 1.\ Monomer\ encounters\ no\ cluster\ or\ other\ neighbor\ monomer
\ \ -{}No\ bonds
\ \ 2.\ Monomer\ encounters\ cluster
\ \ -{}bond\ delete\ monomer\ from\ list
\ \ 3.\ Monomer\ encounters\ single\ monomer
\ \ -{}bond.\ delete\ BOTH\ from\ list

\ \ Diffusion
\ \ 1.\ Monomer\ encounters\ no\ cluster\ or\ other\ neighbor\ monomer
\ \ -{}No\ bonds
\ \ 2.\ Monomer\ encounters\ cluster
\ \ -{}bond\ delete\ monomer\ from\ list
\ \ 3.\ Monomer\ encounters\ single\ monomer
\ \ -{}bond.\ delete\ BOTH\ from\ list
\ \ */}
\ \ bool\ bond=false;
\ \ point\ pt,\ bondpt[dir],lastpt;
\ \ int\ i=0,j=0,ctr=0;
\ \ ctr=getbonds(mysite,bondpt);
\ \ if(ctr==0)
\ \ \{
\ \ \textsl{//No\ cluster\ or\ monomer;}
\ \ bond=false;
\ \ \}
\ \ else
\ \ \{
\ \ \textsl{//for\ each\ bond\ recieved}
\ \ bond=true;
\ \ for(;j<{}ctr;j++)
\ \ \ \ \ \ \ \{
\ \ \ \ pt=bondpt[j];
\ \ \ \ \textsl{//if\ monomer\ means\ it\ has\ an\ index}
\ \ \ \ if(neighborIsMonomer(pt,mysite))
\ \ \ \ \ \ \{
\ \ \ \ \ \ \textsl{//delete\ both\ you\ and\ monomer}
\ \ \ \ \ \ \textsl{//delete\ monomer}
\ \ \ \ \ \ deletemonomer(pt);
\ \ \ \ \ \ \}
\ \ \ \ \}
\ \ \ \ \ \ \textsl{//delete\ your\ self\ IF\ you\ are\ a\ monomer}
\ \ \ \ if(mysite.index!=-{}1)
\ \ \ \ \{
\ \ \ \ \textsl{//rearrange\ list\ if\ mcount\ is\ greater\ than\ 1}
\ \ \ \ \ \ deletemonomer(mysite.pos);
\ \ \ \ \}

\ \ \}
\ \ return\ bond;
\}

void\ Lattice::upnbhd(site\ mysite)
\{
\ \ \ \ \ \ \ \ bool\ bond=false;
\ \ point\ pt,\ bondpt[dir],lastpt;
\ \ int\ x,y;
\ \ int\ i=0,j=0,ctr=0;
\ \ ctr=getnbhrs(mysite,bondpt);
\ \ \textsl{//cout<{}<{}"{}getcte="{}<{}<{}ctr<{}<{}endl;}
\ \ \ checksite(mysite);
\ \ \ \ for(i=0;i<{}ctr;i++)
\ \ \ \ \{
\ \ \ \ \ \ \ x=bondpt[i].x;
\ \ \ \ \ \ \ y=bondpt[i].y;
\ \ \ \ \ \ \ checksite(location[x][y]);\textsl{//bond=checkupdatebonds(location[bondpt[i].x][bondpt[i].y]);}
\ \ \ \ \}
\ \ \ \ \ \ \ if(mysite.h<{}=0\ \&\&\ mysite.index!=-{}1)
\ \ \ \ \ \ \ \ \ \{
\ \ \ \ \ \ deletemonomer(mysite.pos);
\ \ \ \}

\}
void\ Lattice::checksite(site\ mysite)
\{
\ \ bool\ bond=false;
\ \ bond=checkupdatebonds(location[mysite.pos.x][mysite.pos.y]);\ \ \
\ \ \ \ if(bond==false)
\ \ \ \ \{
\ \ \ \ \ \ \ \ if((location[mysite.pos.x][mysite.pos.y].index==-{}1)\ \&\&(location[mysite.pos.x][mysite.pos.y].h>{}0))
\ \ \ \ \ \ \ \ \ \{
\ \ \ \ \ \ \ \ \ \ \ \ location[mysite.pos.x][mysite.pos.y].index=mcount;
\ \ \ \ \ \ \ \ \ \ \ \ monomerloc[mcount]=location[mysite.pos.x][mysite.pos.y].pos;
\ \ \ \ \ \ \ \ \ \ \ \ mcount++;
\ \ \ \ \ \ \ addmonomerchange(0,mysite.pos);
\ \ \ \ \ \ \ \ \ \}
\ \ \ \ \}
\}
void\ Lattice::addmonomerchange(int\ tag,int\ lastpoint)
\{
\ \ change[changecount].time=time;
\ \ if(tag==0)
\ \ \{
\ \ \ \ myevents[eventcount].oldval=monomerloc[count];
\ \ \ \ change[changecount].newsite=mcount;
\ \ \ \ change[changecount].newval=monomerloc[mcount];
\ \ \ \ change[changecount].tag=0;
\ \ \ \ changecount++;
\ \ \}
\ \ else
\ \ \{
\ \ \ \ change[changecount].oldsite=lastpoint;
\ \ \ \ change[changecount].oldval=monomerloc[lastpoint];
\ \ \ \ change[changecount].newsite=count-{}1;
\ \ \ \ change[changecount].newval=monomerloc[count-{}1];
\ \ \ \ change[changecount].tag=1;
\ \ \ \ changecount++;
\ \ \}

\}
void\ Lattice::restorelist(float\ Ctime)
\{
\ \ int\ oldloc,newloc,tag;
\ \ site\ oldval,newval;
\ \ int\ j=changecount-{}1;
\ \ while(change[j].time>{}Ctime)
\ \ \{
\ \ oldloc=change[j].oldsite;
\ \ newloc=change[j].newsite;
\ \ oldval=change[j].oldval;
\ \ newval=change[j].newval;
\ \ tag=change[j].tag;
\ \ if(tag==1)
\ \ \ \ \ \ \ \{
\ \ \ \ monomerloc[newloc]=newval;
\ \ \ \ \ \ \ monomerloc[oldloc]=oldval;
\ \ \ \ mcount++;
\ \ \ \ \}
\ \ else
\ \ \ \ \{
\ \ \ \ monomerloc[newloc]=oldval;
\ \ \ \ mcount-{}-{};
\ \ \ \ \}
\ \ \ \ j-{}-{};
\ \ \}
\}
void\ Lattice::saveconfig()
\{
\ \ int\ j;
\ \ for\ (j=0;j<{}mcount;j++)
\ \ \{
\ \ \ \ \ \ oldlist.monomerloc[j]=monomerloc[j];
\ \ \}
\ \ \
\ \ oldlist.mcount=mcount;
\ \ oldlist.ndep=ndep;
\}

void\ Lattice::restorelist()
\{
\ \ int\ j;
\ \ for\ (j=0;j<{}oldlist.mcount;j++)
\ \ \{
\ \ \ \ monomerloc[j]=oldlist.monomerloc[j];
\ \ \}
\
\ \ mcount=oldlist.mcount;
\ \ ndep=oldlist.ndep;
\}

void\ Lattice::restoreLattice()
\{
\ \ \ undoevent();
\ \ \ restorelist();
\ \ \ int\ i,j;
\ \ \ int\ x,y;
\ \ \ \textsl{/**clear\ lattice**/}
\ \ \ for(i=0;i<{}size+2;i++)
\ \ \ \{
\ \ \ \ \ for(j=0;j<{}size;j++)
\ \ \ \ \ \{
\ \ \ \ \ \ \ \ location[i][j].index=-{}1;
\ \ \ \ \ \}
\ \ \ \}
\ \
\ \ \ \textsl{/** restore\ indexes*/}
\ \ \ for(i=0;i<{}mcount;i++)
\ \ \ \{
\ \ \ \ \ x=monomerloc[i].x;
\ \ \ \ \ y=monomerloc[i].y;
\ \ \ \ \ location[x][y].index=i;
\ \ \ \}
\}

void\ Lattice::restoreLattice(float\ Ctime)
\{
\ \ undoevent(Ctime);
\ \ restorelist(Ctime);
\ \ int\ i,j;
\ \ int\ x,y;
\ \ \textsl{/**clear\ lattice**/}
\ \ for(i=0;i<{}size+2;i++)
\ \ \{
\ \ \ \ for(j=0;j<{}size;j++)
\ \ \ \ \{
\ \ \ \ \ \ location[i][j].index=-{}1;
\ \ \ \ \}
\ \ \}
\ \ \ \
\ \ \textsl{/** restore\ indexes*/}
\ \ for(i=0;i<{}mcount;i++)
\ \ \{
\ \ \ \ x=monomerloc[i].x;
\ \ \ \ y=monomerloc[i].y;
\ \ \ \ location[x][y].index=i;
\ \ \}
\}

void\ Lattice::addbdyevent(site\ oldsite,site\ newsite,float,int\ tag)
\{
\ \ //add\ boundary\ events\ to\ list
\ \ \ \ int\ bdyrightcount=bdycount[right],bdyleftcount=bdycount[left];
\ \ //erase\ pointers
\ \ \ \ oldsite.index=-{}1;newsite.index=-{}1;
\ \ if((oldsite.pos.x>{}=size)\ ||\ (newsite.pos.x>{}=size))
\ \ \{
\ \ \ \ if(oldsite.pos.x==size)
\ \ \ \ \{
\ \ \ \ \ \ oldsite.pos.x=0;
\ \ \ \ \}
\ \ \ \
\ \ \ \ if(oldsite.pos.x==size+1)
\ \ \ \ \{
\ \ \ \ \ \ oldsite.pos.x=1;
\ \ \ \ \}
\ \ \ \
\ \ \ \ if(newsite.pos.x==size)
\ \ \ \ \{
\ \ \ \ \ \ newsite.pos.x=0;
\ \ \ \ \}
\ \ \ \
\ \ \ \ if(newsite.pos.x==size+1)
\ \ \ \ \{
\ \ \ \ \ \ newsite.pos.x=1;
\ \ \ \ \}
\ \ \ \
\ \ \ \ bdyevent[right][bdyrightcount].oldsite=oldsite;
\ \ \ \ bdyevent[right][bdyrightcount].newsite=newsite;
\ \ \ \ bdyevent[right][bdyrightcount].t=time;
\ \ \ \ bdyevent[right][bdyrightcount].tag=tag;
\ \ \ \ bdycount[right]++;
\ \ \}

\ \ if((oldsite.pos.x<{}=1)\ ||\ (newsite.pos.x<{}=1))
\ \ \{\ \ \
\ \ \ \ if(oldsite.pos.x==0)
\ \ \ \ \{
\ \ \ \ \ \ oldsite.pos.x=size;
\ \ \ \ \}
\ \ \ \
\ \ \ \ if(oldsite.pos.x==1)
\ \ \ \ \{
\ \ \ \ \ \ oldsite.pos.x=size+1;
\ \ \ \ \}
\ \ \ \
\ \ \ \ if(newsite.pos.x==0)
\ \ \ \ \{
\ \ \ \ \ \ newsite.pos.x=size;
\ \ \ \ \}
\ \ \ \
\ \ \ \ if(newsite.pos.x==1)
\ \ \ \ \{
\ \ \ \ \ \ newsite.pos.x=size+1;
\ \ \ \ \}\
\ \ \ \ bdyevent[left][bdyleftcount].oldsite=oldsite;
\ \ \ \ bdyevent[left][bdyleftcount].newsite=newsite;
\ \ \ \ bdyevent[left][bdyleftcount].t=time;
\ \ \ \ bdyevent[left][bdyleftcount].tag=tag;
\ \ \ \ bdycount[left]++;
\ \ \}

\}


void\ Lattice::deposit()
\{
\ \ \textsl{/*
\ \ 1.\ Find\ a\ location
\ \ 2.\ place\ monomer\ in\ monomer\ list\ IF\ NO\ neighbours\ around!
\ \ 3.\ \ do\ not\ deposit\ on\ ghost\ region;
\ \ */}
\ \ \ \ float\ xrand=ranlist[iran];
\ \ \ \ iran++;
\ \ \ \ float\ yrand=ranlist[iran];
\ \ \ \ iran++;

\ \ \ \ int\ locx=xrand*(size)+1;
\ \ \ \ int\ locy=yrand*(size);

\ \ \ \ \textsl{//add\ height}
\ \ \ \ location[locx][locy].h+=1;
\ \ \ \ if((locx==1)\ ||\ (locy==size))
\ \ \ \ \{
\ \ \ \ \ \ \textsl{//add\ event\ to\ bdylist}
\ \ \ \ \ \ addbdyevent(location[locx][locy],location[locx][locy],time,depevent);
\ \ \ \ \ \ \ \ \ \
\ \ \ \ \}\ \ \ \ \ \ \ \
\ \ \ \ upnbhd(location[locx][locy]);
\ \ \ \ \textsl{//cout<{}<{}"{}newdeploc.x=\ "{}<{}<{}locx<{}<{}"{}\ newdeploc.y=\ "{}<{}<{}locy<{}<{}endl;\ }
\ \ \ \ \textsl{//add\ to\ eventlist;}
\ \ \
\ \ \ \ myeventlist[nevent].oldsite=location[locx][locy];
\ \ \ \ myeventlist[nevent].newsite=location[locx][locy];
\ \ \ \ myeventlist[nevent].ranseq=nevent;
\ \ \ \ myeventlist[nevent].t=time;
\ \ \ \ myeventlist[nevent].tag=depevent;
\}

void\ Lattice::diffuse()
\{

\ \ point\ newloc,oldloc,lastloc;
\ \ bool\ bonded=false;
\ \ \
\ \ float\ ranm=ranlist[iran];
\ \ iran++;
\ \ \
\ \ if(mcount>{}0)
\ \ \{
\ \ \ \ \ \ int\ loc=(ranm)*(mcount-{}1);
\ \ \ \ \ \ oldloc=monomerloc[loc];
\ \ \ \ \ \ newloc=ranmove(location[oldloc.x][oldloc.y]);
\ \ \ \ \ \ location[newloc.x][newloc.y].h+=1;
\ \ \ \ \ \ location[oldloc.x][oldloc.y].h-{}=1;
\ \ \ \ \textsl{//cout<{}<{}"{}locx"{}<{}<{}oldloc.x<{}<{}"{}locy"{}<{}<{}oldloc.y<{}<{}endl;}
\ \ \ \ \textsl{//cout<{}<{}"{}nlocx"{}<{}<{}newloc.x<{}<{}"{}nlocy"{}<{}<{}newloc.y<{}<{}endl;}
\ \ \ \ \textsl{//cout<{}<{}"{}bdyevent"{}<{}<{}bdycount<{}<{}endl;}
\ \ \ \ if(location[oldloc.x][oldloc.y].h<{}0)
\ \ \ \ \ \ \{
\ \ \ \ \ \ cout<{}<{}"{}fuuuuuuuuuuuuuuuuck"{}<{}<{}endl;
\ \ \ \ \}
\ \ \ \ \textsl{//boundary\ event}
\ \ \ \ \ \ \ \ \
\ \ \ \ if((newloc.x<{}1\ )||\ (newloc.x>{}size))
\ \ \ \ \ \ \{
\ \ \ \ \textsl{//add\ event\ to\ bdylist}
\ \ \ \ deletemonomer(oldloc);
\ \ \ \ addbdyevent(location[oldloc.x][oldloc.y],location[newloc.x][newloc.y],time,diffevent);
\ \ \ \ \
\ \ \ \ \}
\ \ \ \ if(newloc.x==1\ ||\ newloc.x==size)
\ \ \ \ \{
\ \ \ \ addbdyevent(location[oldloc.x][oldloc.y],location[newloc.x][newloc.y],time,diffevent);
\ \ \ \ \}
\ \ \
\ \ \ \ if((oldloc.x<{}=1\ )||\ (oldloc.x>{}=size))
\ \ \ \ \ \ \{
\ \ \ \ \textsl{//add\ event\ to\ bdylist}
\ \ \ \ addbdyevent(location[oldloc.x][oldloc.y],location[newloc.x][newloc.y],time,diffevent);
\ \ \ \ \
\ \ \ \ \}
\ \ \ \ \ \ \
\ \ \ \ \textsl{//diffusion\ may\ release\ trapped\ monomer\ but\ capture\ released\ monomer}
\ \ \ \ if(location[oldloc.x][oldloc.y].h>{}location[newloc.x][newloc.y].h)
\ \ \ \ \{
\ \ \ \ upnbhd(location[oldloc.x][oldloc.y]);\ \
\ \ \ \ \}
\ \ \ \ else
\ \ \ \ \{
\ \ \ \ \textsl{//Move\ Monomer\ by\ changing\ index\ location\ }
\ \ \ \ location[newloc.x][newloc.y].index=location[oldloc.x][oldloc.y].index;
\ \ \ \ monomerloc[location[oldloc.x][oldloc.y].index]=location[newloc.x][newloc.y].pos;
\ \ \ \ location[oldloc.x][oldloc.y].index=-{}1;
\ \ \ \ upnbhd(location[newloc.x][newloc.y]);
\ \ \ \ \}
\ \ \
\ \ \}
\ \ \ \
\ \ \textsl{//add\ to\ eventlist;}
\ \ myeventlist[nevent].oldsite=location[oldloc.x][oldloc.y];
\ \ myeventlist[nevent].newsite=location[newloc.x][newloc.y];
\ \ myeventlist[nevent].ranseq=nevent;
\ \ myeventlist[nevent].t=time;
\ \ myeventlist[nevent].tag=diffevent;
\ \ \ \
\ \ \textsl{//cout<{}<{}"{}****End\ Diffuson*************************"{}<{}<{}endl;\ \ }
\}
void\ Lattice::savebdylist()
\{
\ \ \ \ int\ a,b,i,bdyrightcrec=bdycountrec[right],bdyleftcrec=bdycountrec[left];

\ \ \ \ for(i=0;i<{}bdyleftcrec;i++)
\ \ \ \ \{
\ \ \ \ \ \ oldbdyeventrec[left][i]=bdyeventrec[left][i];
\ \ \ \ \}
\ \ \ \ oldbdycountrec[left]=bdyleftcrec;


\ \ \ \ for(i=0;i<{}bdyrightcrec;i++)
\ \ \ \ \{
\ \ \ \ \ \ oldbdyeventrec[right][i]=bdyeventrec[right][i];
\ \ \ \ \}

\ \ \ \ oldbdycountrec[right]=bdyrightcrec;
\}

int\ Lattice::comparebdylist()
\{
\ \ int\ a,\ b,\ acheck,\ bcheck;

\ \ \ \ acheck\ =\ 0;
\ \ \ \ bcheck\ =\ 0;
\ \ \ \ for\ (a=0;\ a\ <{}\ 2;\ a++)\ \{
\ \ \ \ \ \ if\ (oldbdycountrec[a]!=\ bdycountrec[a])\ \{
\ \ \ \ \ \ \ \ redoflag\ =\ 1;
\ \ \ \ \ \ \ \ acheck\ \ \ =\ 1;
\ \ \ \ \ \ \}\ else\ \{
\ \ \ \ \ \ \ \ for\ (b=0;\ b\ <{}\ bdycountrec[a];\ )\ \{
\ \ \ \ \ \ \ \ \ \ if\ (oldbdyeventrec[a][b].t\ !=\ bdyeventrec[a][b].t)\ \{
\ \ \ \ \ \ \ \ \ \ \ \ redoflag\ =\ 1;
\ \ \ \ \ \ \ \ \ \ \ \ bcheck\ \ \ =\ 1;
\ \ \ \ \ \ \ \ \ \ \ \ b\ \ \ \ \ \ \ \ =\ bdycountrec[a];
\ \ \ \ \ \ \ \ \ \ \}
\ \ \ \ \ \ \ \ \ \ if\ (oldbdyeventrec[a][b].newsite.pos.x\ !=\ bdyeventrec[a][b].newsite.pos.x)\ \{
\ \ \ \ \ \ \ \ \ \ redoflag\ =\ 1;
\ \ \ \ \ \ \ \ \ \ \ \ bcheck\ \ \ =\ 1;
\ \ \ \ \ \ \ \ \ \ \ \ b\ \ \ \ \ \ \ \ =\ bdycountrec[a];
\ \ \ \ \ \ \ \ \ \ \}
\ \ \ \ \ \ \ \ \ \ if\ (oldbdyeventrec[a][b].newsite.pos.y!=bdyeventrec[a][b].newsite.pos.y)\ \{
\ \ \ \ \ \ \ \ \ \ redoflag\ =\ 1;
\ \ \ \ \ \ \ \ \ \ \ \ bcheck\ \ \ =\ 1;
\ \ \ \ \ \ \ \ \ \ \ \ b\ \ \ \ \ \ \ \ =\ bdycountrec[a];
\ \ \ \ \ \ \ \ \ \ \}
\ \ \ \ \ \ \ \ \ \ if\ (oldbdyeventrec[a][b].newsite.h!=bdyeventrec[a][b].newsite.h)\ \{
\ \ \ \ \ \ \ \ \ \ \ \ redoflag\ =\ 1;
\ \ \ \ \ \ \ \ \ \ \ \ bcheck\ \ \ =\ 1;
\ \ \ \ \ \ \ \ \ \ \ \ b\ \ \ \ \ \ \ \ =\ bdycountrec[a];
\ \ \ \ \ \ \ \ \ \ \}
\ \ \ \ \ \ \ \ \ \ b++;
\ \ \ \ \ \ \ \ \}
\ \ \ \ \ \ \}
\ \ \ \ \}

\}


void\ Lattice::p()
\{
\ \ cout<{}<{}"{}**************S**********************************"{};
\ \ float\ theta=0,vacancy=(float)\ mcount\ ,lat=(float)latsize;
\ \ float\ x;
\ \ for\ (int\ i=0;i<{}size;i++)
\ \ \{
\ \ cout<{}<{}endl;
\ \ for\ (int\ j=0;j<{}size+2;j++)
\ \ cout<{}<{}location[j][i].h;
\ \ \}

\ \ cout<{}<{}endl<{}<{}"{}mcount="{}<{}<{}mcount<{}<{}endl;
\ \ \textsl{//cout<{}<{}"{}**************E**********************************"{}<{}<{}endl;}
\ \ theta=(lat-{}vacancy)/lat;
\ \ \textsl{//cout<{}<{}"{}Theta="{}<{}<{}theta<{}<{}endl;}
\ \ for\ (int\ i=0;i<{}=size;i++)
\ \ \{
\ \ cout<{}<{}endl;
\ \ for\ (int\ j=0;j<{}size+2;j++)
\ \ cout<{}<{}location[j][i].index<{}<{}"{}\ \ \ "{};
\ \ \}
\}
void\ Lattice::doKMC()
\{
\ \ \textsl{///create\ and\ save\ random\ number}
\ \ float\ ranX;

\ \ ranX=ranlist[iran];
\ \ iran++;

\ \ float\ Trate,Drate;
\ \ Drate=.25*mcount*difrate;
\ \ Trate=Drate+(deprate*\ (float)\ latsize);

\ \ float\ prob=(Drate/Trate);

\ \ \textsl{//cout<{}<{}"{}ranX="{}<{}<{}ranlist[iran]<{}<{}endl;}
\ \ if(ranX<{}prob)
\ \ diffuse();
\ \ else
\ \ \{
\ \ deposit();
\ \ ndep++;
\ \ \}
\ \ nevent++;
\
\}

void\ Lattice::undoevent()\ \{
\ \ \ \ int\ a,\ xi,\ yi,\ xf,\ yf,\ tag;
\ \ \ \ double\ t;

\ \ \ \ if\ (redoflag\ ==\ 0)\ \{
\ \ \ \ \ \ \ \ return;
\ \ \}

\ \ \ \ for\ (a=nevent-{}1;\ a\ >{}=0;\ a-{}-{})\ \{
\ \ \ \ \ \ \ \ tag\ \ =\ myeventlist[a].tag;
\ \ \ \ \ \ \ \ xi\ \ \ =\ myeventlist[a].oldsite.pos.x;
\ \ \ \ \ \ \ \ yi\ \ \ =\ myeventlist[a].oldsite.pos.y;
\ \ \ \ \ \ \ \ xf\ \ \ =\ myeventlist[a].newsite.pos.x;
\ \ \ \ \ \ \ \ yf\ \ \ =\ myeventlist[a].newsite.pos.y;
\ \ \ \ \ \ \ \ t\ \ \ \ =\ myeventlist[a].t;

\ \ \ \ \ \ \ \ switch\ (tag)\ \{
\ \ \ \ \ \ \ \ case\ 0:
\ \ \ \ \ \ \ \ \ \ \ \ location[xi][yi].h\ =\ myeventlist[a].oldsite.h;
\ \ \ \ \ \ \ \ \ \ \ \ if\ (myeventlist[a].newsite.h\ !=\ -{}1)
\ \ \ \ \ \ \ \ \ \ \ \ \ \ \ \ location[xf][yf].h\ =\ myeventlist[a].newsite.h;
\ \ \ \ \ \ \ \ \ \ \ \ break;
\ \ \ \ \ \ \ \ case\ 1:
\ \ \ \ \ \ \ \ \ \ \ \ location[xi][yi].h\ =\ location[xi][yi].h\ +\ 1;
\ \ \ \ \ \ \ \ \ \ \ \ location[xf][yf].h\ =\ location[xf][yf].h\ -{}\ 1;
\ \ \ \ \ \ \ \ \ \ \ \ break;
\ \ \ \ \ \ \ \ case\ 2:
\ \ \ \ \ \ location[xi][yi].h\ =\ location[xi][yi].h\ -{}\ 1;
\ \ \ \ \ \ \ \ \ \ \ \ ndep-{}-{};
\ \ \ \ \ \ \ \ \ \ \ \ break;
\ \ \ \ \ \ \ \ \ \ \
\ \ \ \ \ \ \ \ default:
\ \ \ \ \ \ \ \ \ \ \ \ cout<{}<{}"{}Error\ in\ tag"{}<{}<{}endl;
\ \ \ \ \ \ \ \ \ \ \ \ return;\ \textsl{/*\ SHOULDN'T\ THIS\ EXIT()?!?\ */}
\ \ \ \ \ \ \ \ \}
\ \ \ \ \}
\}

void\ Lattice::undoevent(float\ Ctime)\ \{
\ \ \ \ int\ a,\ xi,\ yi,\ xf,\ yf,\ tag;
\ \ \ \ double\ t;

\ \ \ \ if\ (redoflag\ ==\ 0)\ \{
\ \ \ \ \ \ \ \ return;
\ \ \}
\ \ \ \ a=nevent-{}1;
\ \ \ t\ \ \ \ =\ myeventlist[a].t;
\ \ \ \ while(t>{}Ctime)\ \{
\ \ \ \ \ \ \ \ tag\ \ =\ myeventlist[a].tag;
\ \ \ \ \ \ \ \ xi\ \ \ =\ myeventlist[a].oldsite.pos.x;
\ \ \ \ \ \ \ \ yi\ \ \ =\ myeventlist[a].oldsite.pos.y;
\ \ \ \ \ \ \ \ xf\ \ \ =\ myeventlist[a].newsite.pos.x;
\ \ \ \ \ \ \ \ yf\ \ \ =\ myeventlist[a].newsite.pos.y;
\ \ \ \ \ \ \ \ t\ \ \ \ =\ myeventlist[a].t;

\ \ \ \ \ \ \ \ switch\ (tag)\ \{
\ \ \ \ \ \ \ \ case\ 0:
\ \ \ \ \ \ \ \ \ \ \ \ location[xi][yi].h\ =\ myeventlist[a].oldsite.h;
\ \ \ \ \ \ \ \ \ \ \ \ if\ (myeventlist[a].newsite.h\ !=\ -{}1)
\ \ \ \ \ \ \ \ \ \ \ \ \ \ \ \ location[xf][yf].h\ =\ myeventlist[a].newsite.h;
\ \ \ \ \ \ \ \ \ \ \ \ break;
\ \ \ \ \ \ \ \ case\ 1:
\ \ \ \ \ \ \ \ \ \ \ \ location[xi][yi].h\ =\ location[xi][yi].h\ +\ 1;
\ \ \ \ \ \ \ \ \ \ \ \ location[xf][yf].h\ =\ location[xf][yf].h\ -{}\ 1;
\ \ \ \ \ \ \ \ \ \ \ \ break;
\ \ \ \ \ \ \ \ case\ 2:
\ \ \ \ \ \ location[xi][yi].h\ =\ location[xi][yi].h\ -{}\ 1;
\ \ \ \ \ \ \ \ \ \ \ \ ndep-{}-{};
\ \ \ \ \ \ \ \ \ \ \ \ break;
\ \ \ \ \ \ \ \ \ \ \
\ \ \ \ \ \ \ \ default:
\ \ \ \ \ \ \ \ \ \ \ \ cout<{}<{}"{}Error\ in\ tag"{}<{}<{}endl;
\ \ \ \ \ \ \ \ \ \ \ \ return;\ \textsl{/*\ SHOULDN'T\ THIS\ EXIT()?!?\ */}
\ \ \ \ \ \ \ \ \}
\ \ \ \ a-{}-{};
\ \ \ \ \}
\}

void\ Lattice::randgen()
\{
\ \ int\ i;
\ \ for(i=0;i<{}10000;i++)
\ \ \{
\ \ \ \ ranlist[i]=((float)rand()/(float)RAND\underline\ MAX);
\ \ \}
\}
void\ Lattice::updateBuffer(int\ iranflag)\ \{
\ \ \ \ int\ a,\ b,\ am1,\ x,\ y,\ xi,\ ii,\ abflag,\ mflag,\ sdir,\ dir,\ aid,\ i,\ j,\ hij,\ hxy,tag;
\ \ \ \ double\ newTrate,\ oldTrate;
\ \ \ \ point\ oldsite,newsite;
\ \ \
\ \ \ \ i=sortbdyevent[nupdate].oldsite.pos.x;
\ \ \ \ j=sortbdyevent[nupdate].oldsite.pos.y;
\ \ \
\ \ \ \ x=sortbdyevent[nupdate].newsite.pos.x;
\ \ \ \ y=sortbdyevent[nupdate].newsite.pos.y;
\ \ \
\ \ \ \ tag=sortbdyevent[nupdate].tag;

\ \ \ \ time\ =\ sortbdyevent[nupdate].t;
\ \
\ \ if\ (redoflag\ ==\ 0)\ \{
\ \ \ \ \ \ \ \ return;
\ \ \}
\ \ \ \textsl{//cout<{}<{}"{}update\ buffer!"{}<{}<{}endl;}
\ \ \ \textsl{//cout<{}<{}"{}x="{}<{}<{}x<{}<{}"{}\ y="{}<{}<{}nupdate<{}<{}endl;}
\ \ \ \textsl{//cout<{}<{}"{}i="{}<{}<{}x<{}<{}"{}\ j="{}<{}<{}y<{}<{}endl;}
\ \ \
\ \ \ \
\ \ \
\ \ \ \ if(tag==diffevent)
\ \ \{
\ \ \ \ \ \ \ \ location[i][j].h\ =\ sortbdyevent[nupdate].oldsite.h;
\ \ \ \ \ \ \ \ upnbhd(sortbdyevent[nupdate].oldsite);
\ \ \ \ \ \ \ \ \}
\ \ \ \ else
\ \ \{
\ \ location[x][y].h\ =\ sortbdyevent[nupdate].newsite.h;
\ \ \ \ \ \ \ \ upnbhd(sortbdyevent[nupdate].newsite);
\ \ \}\ \

\ \textsl{/*\ add\ this\ event\ in\ my\ event\ list\ */}
\ \ \ \ myeventlist[nevent].oldsite=location[x][y];
\ \ \ \ myeventlist[nevent].newsite=location[i][j];
\ \ \ \ myeventlist[nevent].ranseq=iran\ -{}\ iranflag;
\ \ \ \ myeventlist[nevent].t=time;
\ \ \ \ myeventlist[nevent].tag=0;
\ \
\ \ \ \ nupdate++;
\ \ \ \ nevent++;
\}

void\ Lattice::sorting\underline\ nbevent()\ \{
\ \ \ \ int\ a,\ b,\ nxcv,\ i,\ j,\ caselabel,\ dir,\ idn;
\ \ \ \ double\ t;
\ \ \ \ boundaryevent\ swap;

\ \ \ \ if\ (bdycountrec[left]\ >{}\ 0\ \&\&\ bdycountrec[right]\ ==\ 0)
\ \ \ \ \ \ \ \ caselabel\ =\ 0;
\ \ \ \ if\ (bdycountrec[left]\ ==\ 0\ \&\&\ bdycountrec[right]\ >{}\ 0)
\ \ \ \ \ \ \ \ caselabel\ =\ 1;
\ \ \ \ if\ (bdycountrec[left]\ >{}\ 0\ \&\&\ bdycountrec[right]\ >{}\ 0)
\ \ \ \ \ \ \ \ caselabel\ =\ 2;

\ \ \ \ switch(caselabel)\ \{
\ \ \ \ case\ 0:
\ \ \ \ \ \ \ \ for\ (a=0;\ a\ <{}\ bdycountrec[left];\ a++)\ \{
\ \ \ \ \ \ \ \ \ \ \ \ sortbdyevent[a]=\ bdyeventrec[0][a];
\ \ \ \ \ \ \ \ \ \ \ \ \}
\ \ \ \ \ \ \ \ tnbdyevent\ =\ bdycountrec[0];
\ \ \ \ \ \ \ \ break;
\ \ \ \ case\ 1:
\ \ \ \ \ \ \ \ for\ (a=0;\ a\ <{}\ bdycountrec[1];\ a++)\ \{
\ \ \ \ \ \ \ \ \ \ \ \ sortbdyevent[a]\ =\ bdyeventrec[1][a];
\ \ \ \ \}
\ \ \ \ \ \ \ \ tnbdyevent\ =\ bdycountrec[1];
\ \ \ \ \ \ \ \ break;
\ \ \ \ case\ 2:
\ \ \ \ \ \ \ \ tnbdyevent\ =\ bdycountrec[0]\ +\ bdycountrec[1];
\ \ \ \ \ \ \ \ nxcv\ =\ 0;
\ \ \ \ \ \ \ \ \textsl{/*\ sort\ the\ events\ in\ early\ time\ order\ */}
\ \ \ \ \ \ \ \ for\ (a\ =\ 0;\ a\ <{}\ bdycountrec[0];\ a++)\ \{
\ \ \ \ \ \ \ \ \ \ \ \ sortbdyevent[nxcv]=\ bdyeventrec[0][a];
\ \ \ \ \ \ \ \ \ \ \ \ nxcv++;
\ \ \ \ \ \ \ \ \}

\ \ \ \ \ \ \ \ for\ (a=0;\ a\ <{}\ bdycountrec[1];\ a++)\ \{
\ \ \ \ \ \ \ \ \ \ \ \ sortbdyevent[nxcv]=\ bdyeventrec[1][a];
\ \ \ \ \ \ \ \ \ \ \ \ nxcv++;
\ \ \ \ \ \ \ \ \}

\ \ \ \ \ \ \ \ for\ (j=1;\ j\ <{}\ tnbdyevent;\ j++)\ \{
\ \ \ \ \ \ \ \ \ \ \ \ swap=sortbdyevent[j];
\ \ \ \ \ \ \ \ \ \ \ \ i\ =\ j\ -{}\ 1;
\ \ \ \ \ \ \ \ \ \ \ \ while\ (i\ >{}=\ 0\ \&\&\ sortbdyevent[i].t\ >{}\ t)\ \{
\ \ \ \ \ \ \ \ \ \ \ \ \ \ \ \ sortbdyevent[i+1]\ =\ sortbdyevent[i];
\ \ \ \ \ \ \ \ \ \ \ \ \ \ \ \ i-{}-{};
\ \ \ \ \ \ \ \ \ \ \ \ \}
\ \ \ \ \ \ \ \ \ \ \ \ sortbdyevent[i+1]=swap;
\ \ \ \ \ \ \ \ \}
\ \ \ \ \ \ \ \ break;
\ \ \ \ default:
\ \ \ \ \ \ \ \ break;
\ \ \ \ \}
\}

 }
\normalfont\normalsize


\end{code}

% EVENT SOURCE FILES
\section{event.h}
\begin{code}{event.h}{tw/event.h}
{\ttfamily \raggedright \footnotesize
\#include\ "{}latprim.h"{}

\#ifndef EVENT\underline\ H
\#define EVENT\underline\ H

enum\ EventType\ \{eventDeposition,\ eventDiffusion\};

class\ Event\ \{
public:
\ \ Event(site*\ oldSite,\ site*\ newSite,\ double\ time,\ bool\ isLocal,\ EventType\ eventType,\textsl{/*\ Direction\ dir,*/}\ int\ listIndex)\ \{
\ \ \ \ this-{}>{}oldSite\ =\ oldSite;
\ \ \ \ this-{}>{}newSite\ =\ newSite;
\ \ \ \ this-{}>{}time\ =\ time;
\ \ \ \ this-{}>{}isLocal\ =\ isLocal;
\ \ \ \ this-{}>{}eventType\ =\ eventType;
\ \ \ \ \textsl{/*this-{}>{}dir\ =\ dir;*/}
\ \ \ \ this-{}>{}listIndex\ =\ listIndex;
\ \ \}

\ \ Event(site*\ newSite,\ double\ time,\ bool\ isLocal,\ EventType\ eventType\textsl{/*,\ Direction\ dir,\ int\ listIndex*/})\ \{
\ \ \ \ this-{}>{}newSite\ =\ newSite;
\ \ \ \ this-{}>{}time\ =\ time;
\ \ \ \ this-{}>{}isLocal\ =\ isLocal;
\ \ \ \ this-{}>{}eventType\ =\ eventType;
\ \ \ \ \textsl{/*this-{}>{}dir\ =\ dir;*/}
\ \ \ \ \textsl{/*this-{}>{}listIndex\ =\ listIndex;*/}
\ \ \}

\ \ site*\ oldSite;\ \ \ \ \ \ \ \textsl{//\ the\ old\ site\ (diffusion\ event)}
\ \ site*\ newSite;\ \ \ \ \ \ \ \textsl{//\ the\ new\ site\ (diffusion\ event)\ or\ the\ site\ for\ a\ deposition\ event}
\ \ double\ time;\ \ \ \ \ \ \ \ \ \textsl{//\ event\ time}
\ \ bool\ isLocal;\ \ \ \ \ \ \ \ \textsl{//\ true\ if\ a\ local\ event}
\ \ bool\ isBoundry;\ \ \ \ \ \ \textsl{//\ true\ if\ a\ boundry\ event}
\ \ EventType\ eventType;\ \textsl{//\ type\ of\ event}
\textsl{//\ \ Direction\ dir;\ \ \ \ \ \ \ //\ important\ if\ it's\ a\ boundry\ event,\ which\ neighbor\ to\ send\ it\ too}
\ \ int\ listIndex;\ \ \ \ \ \ \ \textsl{//\ index\ of\ the\ site\ in\ the\ list}
private:
\ \ Event()\ \{\ ;\ \}\ \textsl{//\ supress\ creating\ an\ event\ with\ a\ default\ constructor}
\};

\#endif
 }
\normalfont\normalsize


\end{code}

% REWIND LIST FILES
\section{rewindlist.h}
\begin{code}{rewindlist.h}{tw/rewindlist.h}
\input{twcode/rewindlist.h.tex}
\end{code}

% REWIND LIST FILES
\section{randgen.h}
\begin{code}{randgen.h}{tw/randgen.h}
{\ttfamily \raggedright \footnotesize
\#include\ <{}vector>{}
using\ std::vector;

\#include\ <{}stack>{}
using\ std::stack;

\#include\ <{}iostream>{}
using\ std::cout;
using\ std::endl;

\#ifndef RANDGEN\underline\ H
\#define RANDGEN\underline\ H

class\ RandGen\ \{
public:
\ \ RandGen(int\ size)\ :\ expand(size)\ \{
\ \ \ \ this-{}>{}seed\ =\ 0;
\ \ \ \ populateList(size);
\ \ \ \ position\ =\ randList.begin();
\ \ \}

\ \ RandGen(int\ size,int\ seed)\ :\ expand(size)\ \{
\ \ \ \ this-{}>{}seed\ =\ seed;
\ \ \ \ populateList(size);
\ \ \ \ position\ =\ randList.begin();
\ \ \}

\ \ bool\ rewind(double\ t)\ \{
\ \ \ \ if(times.empty())
\ \ \ \ \ \ return(true);

\ \ \ \ while(!times.empty()\ \&\&\ times.top()\ >{}\ t)\ \{
\ \ \ \ \ \ position-{}-{};
\ \ \ \ \ \ times.pop();
\ \ \ \ \}

\ \ \ \ if(position\ <{}=\ randList.begin())
\ \ \ \ \ \ position\ =\ randList.begin();
\ \ \ \ return(true);
\ \ \}

\ \ float\ getRandom(double\ t)\ \{
\ \ \ \ ++position;
\ \ \ \ times.push(t);
\ \ \ \ if(position\ !=\ randList.end())
\ \ \ \ \ \ return(*(position));
\ \ \ \ else\ \{
\ \ \ \ \ \ int\ offset\ =\ randList.size();
\ \ \ \ \ \ populateList(expand);
\ \ \ \ \ \ position\ =\ randList.begin()\ +\ offset;
\ \ \ \ \}
\ \ \ \ return(*(position));
\ \ \}

private:
\ \ vector<{}float>{}\ randList;
\ \ vector<{}float>{}::iterator\ position;
\ \ stack<{}double>{}\ times;

\ \ int\ seed;
\ \ int\ expand;

\ \ RandGen()\ \{\ ;\ \}

\ \ void\ populateList(int\ count)\ \{
\ \ \ \ for(int\ i=0;\ i\ <{}\ count;\ ++i)
\ \ \ \ \ \ randList.push\underline\ back((float)genRand());
\ \ \}

\ \ double\ genRand()\ \{
\ \ \ \ \textsl{/*-{}-{}-{}-{}-{}-{}-{}-{}-{}-{}-{}-{}-{}-{}-{}-{}-{}-{}-{}-{}-{}-{}-{}-{}-{}-{}-{}-{}-{}-{}-{}-{}-{}-{}-{}-{}-{}-{}-{}-{}-{}-{}-{}-{}-{}-{}-{}-{}-{}-{}-{}-{}-{}-{}-{}-{}-{}-{}-{}-{}-{}-{}-{}-{}-{}-{}-{}*/}
\ \ \ \ \textsl{/*\ A\ C-{}program\ for\ TT800\ :\ July\ 8th\ 1996\ Version\ */}
\ \ \ \ \textsl{/*\ by\ M.\ Matsumoto,\ email:\ matumoto@math.keio.ac.jp\ */}
\ \ \ \ \textsl{/*\ genrand()\ generate\ one\ pseudorandom\ number\ with\ double\ precision\ */}
\ \ \ \ \textsl{/*\ which\ is\ uniformly\ distributed\ on\ [0,1]-{}interval\ */}
\ \ \ \ \textsl{/*\ for\ each\ call.\ \ One\ may\ choose\ any\ initial\ 25\ seeds\ */}
\ \ \ \ \textsl{/*\ except\ all\ zeros.\ */}

\ \ \ \ \textsl{/*\ See:\ ACM\ Transactions\ on\ Modelling\ and\ Computer\ Simulation,\ */}
\ \ \ \ \textsl{/*\ Vol.\ 4,\ No.\ 3,\ 1994,\ pages\ 254-{}266.\ */}

\ \ \ \ const\ int\ NRan\ =\ 25;
\ \ \ \ int\ MRan\ =\ seed\ \%\ NRan;

\ \ \ \ unsigned\ long\ y;
\ \ \ \ \ static\ int\ k\ =\ 0;
\ \ \ \ static\ unsigned\ long\ x[NRan]=\{\ \textsl{/*\ initial\ 25\ seeds,\ change\ as\ you\ wish\ */}
\ \ \ \ \ \ \ \ \ \ \ \ \ \ \ \ \ \ \ \ \ \ \ \ \ \ \ \ \ \ \ \ \ \ \ \ \ 0x95f24dab,\ 0x0b685215,\ 0xe76ccae7,\ 0xaf3ec239,\ 0x715fad23,
\ \ \ \ \ \ \ \ \ \ \ \ \ \ \ \ \ \ \ \ \ \ \ \ \ \ \ \ \ \ \ \ \ \ \ \ \ 0x24a590ad,\ 0x69e4b5ef,\ 0xbf456141,\ 0x96bc1b7b,\ 0xa7bdf825,
\ \ \ \ \ \ \ \ \ \ \ \ \ \ \ \ \ \ \ \ \ \ \ \ \ \ \ \ \ \ \ \ \ \ \ \ \ 0xc1de75b7,\ 0x8858a9c9,\ 0x2da87693,\ 0xb657f9dd,\ 0xffdc8a9f,
\ \ \ \ \ \ \ \ \ \ \ \ \ \ \ \ \ \ \ \ \ \ \ \ \ \ \ \ \ \ \ \ \ \ \ \ \ 0x8121da71,\ 0x8b823ecb,\ 0x885d05f5,\ 0x4e20cd47,\ 0x5a9ad5d9,
\ \ \ \ \ \ \ \ \ \ \ \ \ \ \ \ \ \ \ \ \ \ \ \ \ \ \ \ \ \ \ \ \ \ \ \ \ 0x512c0c03,\ 0xea857ccd,\ 0x4cc1d30f,\ 0x8891a8a1,\ 0xa6b7aadb
\ \ \ \ \ \ \ \ \ \ \ \ \ \ \ \ \ \ \ \ \ \ \ \ \ \ \ \ \ \ \ \ \ \};
\ \ \ \ static\ unsigned\ long\ mag01[2]=\{0x0,\ 0x8ebfd028\ \textsl{/*\ this\ is\ magic\ vector\ `a',\ don't\ change\ */}\};
\ \ \ \ if\ (k==NRan)\ \{\ \textsl{/*\ generate\ NRan\ words\ at\ one\ time\ */}
\ \ \ \ \ \ int\ kk;
\ \ \ \ \ \ for\ (kk=0;kk<{}NRan-{}MRan;kk++)\ \{
\ \ \ \ \ \ \ \ x[kk]\ =\ x[kk+MRan]\ \textasciicircum \ (x[kk]\ >{}>{}\ 1)\ \textasciicircum \ mag01[x[kk]\ \%\ 2];
\ \ \ \ \ \ \}
\ \ \ \ \ \ for\ (;\ kk<{}NRan;kk++)\ \{
\ \ \ \ \ \ \ \ x[kk]\ =\ x[kk+(MRan-{}NRan)]\ \textasciicircum \ (x[kk]\ >{}>{}\ 1)\ \textasciicircum \ mag01[x[kk]\ \%\ 2];
\ \ \ \ \ \ \}
\ \ \ \ \ \ k=0;
\ \ \ \ \}
\ \ \ \ y\ =\ x[k];
\ \ \ \ y\ \textasciicircum =\ (y\ <{}<{}\ 7)\ \&\ 0x2b5b2500;\ \textsl{/*\ s\ and\ b,\ magic\ vectors\ */}
\ \ \ \ y\ \textasciicircum =\ (y\ <{}<{}\ 15)\ \&\ 0xdb8b0000;\ \textsl{/*\ t\ and\ c,\ magic\ vectors\ */}
\ \ \ \ y\ \&=\ 0xffffffff;\ \textsl{/*\ you\ may\ delete\ this\ line\ if\ word\ size\ =\ 32\ */}
\ \ \ \ \textsl{/*
\ \ \ \ \ \ \ the\ following\ line\ was\ added\ by\ Makoto\ Matsumoto\ in\ the\ 1996\ version
\ \ \ \ \ \ \ to\ improve\ lower\ bit's\ corellation.
\ \ \ \ \ \ \ Delete\ this\ line\ to\ o\ use\ the\ code\ published\ in\ 1994.
\ \ \ \ */}
\ \ \ \ y\ \textasciicircum =\ (y\ >{}>{}\ 16);\ \textsl{/*\ added\ to\ the\ 1994\ version\ */}
\ \ \ \ k++;
\ \ \ \ return(\ (double)\ y\ /\ (unsigned\ long)\ 0xffffffff);
\ \ \}

\};

\#endif


 }
\normalfont\normalsize


\end{code}

% MPIWRAPPER LIST FILES
\section{mpiwrapper.h}
\begin{code}{mpiwrapper.h}{tw/mpiwrapper.h}
{\ttfamily \raggedright \footnotesize
\#include\ <{}vector>{}
using\ std::vector;

\#include\ <{}fstream>{}
using\ std::fstream;
using\ std::endl;

\#include\ <{}iostream>{}
using\ std::cout;

\#include\ <{}mpi.h>{}

\#include\ "{}latprim.h"{}
\#include\ "{}latconst.h"{}
\#include\ "{}exception.h"{}
\#include\ "{}event.h"{}

\#ifndef MPIWRAPPER\underline\ H
\#define MPIWRAPPER\underline\ H

\#define LEFT(a,b)\ (((a\ -{}\ 1)\ >{}=\ 0)\ ?\ (a\ -{}\ 1)\ :\ -{}1)
\#define RIGHT(a,b)\ (((a\ +\ 1)\ <{}\ b)\ ?\ (a\ +\ 1)\ :\ -{}1)
\#define DIR(a)\ ((a\ ==\ LEFT)\ ?\ left\ :\ right)
const\ int\ BUFFER\underline\ SIZE\underline\ COUNT\ =\ 1024*1024*10;\ \textsl{//\ 10MB\ buffer\ (overkill,\ baby)}
const\ int\ ROOT\underline\ RANK\ =\ 0;
const\ int\ NUM\underline\ NEIGHBORS\ =\ 2;

const\ int\ TAG\underline\ MESSAGE\ =\ 0;
const\ int\ TAG\underline\ ANTI\underline\ MESSAGE\ =\ 1;

enum\ Direction\ \{LEFT,RIGHT\};

typedef\ struct\ \{
\ \ site\ oldSite;
\ \ site\ newSite;
\ \ double\ time;
\ \ EventType\ type;
\}\ message;

class\ MPIWrapper\ \{
public:
\ \ MPIWrapper();
\ \ \textasciitilde MPIWrapper();

\ \ bool\ init(int*,\ char***);
\ \ bool\ shutdown();

\ \ bool\ sendMessage(message*\ ,\ Direction);
\ \ bool\ recvMessages(vector<{}message>{}*);
\ \ bool\ isMessage();

\ \ bool\ sendAntiMessage(message*\ ,\ Direction);
\ \ bool\ recvAntiMessages(vector<{}message>{}*);
\ \ bool\ isAntiMessage();

\ \ float\ allReduceFloat(float,\ MPI\underline\ Op);
\ \ double\ allReduceDouble(double,\ MPI\underline\ Op);
\ \ int\ allReduceInt(int,\ MPI\underline\ Op);

\ \ void\ barrier();
\ \ double\ wallTime();

\ \ bool\ isRoot();

\ \ int\ getRank()\ \{
\ \ \ \ return(rank);
\ \ \}

\ \ int\ getNodeCount()\ \{
\ \ \ \ return(nodeCount);
\ \ \}

\ \ void\ printStats(fstream\&\ file)\ \{
\ \ \ \ file\ <{}<{}\ "{}-{}-{}-{}\ MPI\ STATS\ -{}-{}-{}"{}\ <{}<{}\ endl;
\ \ \ \ file\ <{}<{}\ \ "{}rank\ =\ "{}\ <{}<{}\ rank\ <{}<{}\ endl;
\ \ \ \ file\ <{}<{}\ "{}nodeCount\ =\ "{}\ <{}<{}\ nodeCount\ <{}<{}\ endl;
\ \ \ \ file\ <{}<{}\ "{}left\ =\ "{}\ <{}<{}\ left\ <{}<{}\ "{}\ right\ =\ "{}\ <{}<{}\ right\ <{}<{}\ endl;
\ \ \ \ file\ <{}<{}\ "{}Send\ Messages\ =\ "{}\ <{}<{}\ countSend\ <{}<{}\ "{}\ Recieved\ Messages\ =\ "{}\ <{}<{}\ countRecv\ <{}<{}\ endl;
\ \ \ \ file\ <{}<{}\ "{}Send\ Anti-{}Messages\ =\ "{}\ <{}<{}\ countSendAnti\ <{}<{}\ "{}\ Recieved\ Anti-{}Messages\ =\ "{}\ <{}<{}\ countRecvAnti\ <{}<{}\ endl;
\ \ \ \ file\ <{}<{}\ "{}Total\ Send\ =\ "{}\ <{}<{}\ (countSend\ +\ countSendAnti)\ <{}<{}\ "{}\ Total\ Recv\ =\ "{}\ <{}<{}\ (countRecv\ +\ countRecvAnti)\ <{}<{}\ endl;
\ \ \ \ file\ <{}<{}\ "{}-{}-{}-{}-{}-{}-{}-{}-{}-{}-{}-{}-{}-{}-{}-{}-{}-{}"{}\ <{}<{}\ endl\ <{}<{}\ endl;
\ \ \ \ file.flush();
\ \ \}

private:
\ \ int\ rank;
\ \ int\ nodeCount;
\ \ bool\ isInit;

\ \ char*\ buffer;
\ \ int\ bufferSize;

\ \ MPI\underline\ Datatype\ typeSite;
\ \ MPI\underline\ Datatype\ typePoint;
\ \ MPI\underline\ Datatype\ typeMessage;

\ \ int\ left;
\ \ int\ right;
\ \ message\ m;

\ \ MPI\underline\ Status\ status;
\ \ int\ flag;

\ \ int\ countSend;
\ \ int\ countRecv;
\ \ int\ countSendAnti;
\ \ int\ countRecvAnti;
\};

\#endif

 }
\normalfont\normalsize


\end{code}

\section{mpiwrapper.cpp}
\begin{code}{mpiwrapper.cpp}{tw/mpiwrapper.cpp}
{\ttfamily \raggedright \small
\#include\ <{}vector>{}\\
using\ std::vector;\\
\ \\
\#include\ <{}mpi.h>{}\\
\#include\ <{}stdlib.h>{}\\
\ \\
\#include\ "{}mpiwrapper.h"{}\\
\#include\ "{}latprim.h"{}\\
\#include\ "{}exception.h"{}\\
\#include\ "{}event.h"{}\\
\ \\
MPIWrapper::MPIWrapper()\ :\ rank(-{}1),\ nodeCount(-{}1),\ isInit(false),\ left(-{}1),\ right(-{}1),\ countSend(0),\ countRecv(0),countSendAnti(0),countRecvAnti(0)\ \{\ ;\ \}\\
\ \\
MPIWrapper::\textasciitilde MPIWrapper()\ \{\\
\ \ \textsl{//if(isInit)}\\
\ \ \textsl{//\ \ shutdown();}\\
\}\\
\ \\
bool\ MPIWrapper::init(int$\ast$\ argv,\ char$\ast$$\ast$\ argc[])\ \{\\
\ \ MPI\underline\ Aint$\ast$\ displacements;\\
\ \ MPI\underline\ Datatype$\ast$\ dataTypes;\\
\ \ int$\ast$\ blockLength;\\
\ \ MPI\underline\ Aint\ startAddress;\\
\ \ MPI\underline\ Aint\ address;\\
\ \ point\ p;\\
\ \ site\ s;\\
\ \\
\ \ \textsl{//\ see\ if\ init\ ==\ true,\ if\ that\ is\ so\ we've\ got\ big\ problems}\\
\ \ if(isInit)\\
\ \ \ \ throw(Exception("{}ERROR:\ Duplicate\ call\ to\ MPIWrapper::init()!"{}));\\
\ \\
\ \ \textsl{//\ call\ MPI\underline\ Init()\ to\ start\ this\ whole\ shebang}\\
\ \ MPI\underline\ Init(argv,argc);\\
\ \\
\ \ \textsl{//\ get\ the\ process\ rank\ and\ the\ number\ of\ nodes}\\
\ \ MPI\underline\ Comm\underline\ rank(MPI\underline\ COMM\underline\ WORLD,\&rank);\\
\ \ MPI\underline\ Comm\underline\ size(MPI\underline\ COMM\underline\ WORLD,\&nodeCount);\\
\ \\
\ \ \textsl{//\ make\ sure\ the\ shit\ didn't\ hit\ the\ fan}\\
\ \ if(rank\ <{}\ 0)\ \{\\
\ \ \ \ throw(Exception("{}ERROR:\ MPI\underline\ Comm\underline\ rank()\ failed\ to\ return\ useful\ value!"{}));\\
\ \ \}\\
\ \ if(nodeCount\ <{}\ 0)\ \{\\
\ \ \ \ throw(Exception("{}ERROR:\ MPI\underline\ Comm\underline\ size()\ filed\ to\ return\ useful\ value!"{}));\\
\ \ \}\\
\ \\
\ \ \textsl{//\ create\ the\ datatype\ for\ the\ point\ structure}\\
\ \ displacements\ =\ new\ MPI\underline\ Aint[2];\\
\ \ dataTypes\ =\ new\ MPI\underline\ Datatype[2];\\
\ \ blockLength\ =\ new\ int[2];\\
\ \\
\ \ blockLength[0]\ =\ 1;\\
\ \ blockLength[1]\ =\ 1;\\
\ \ dataTypes[0]\ =\ MPI\underline\ INT;\\
\ \ dataTypes[1]\ =\ MPI\underline\ INT;\\
\ \\
\ \ MPI\underline\ Address(\&p.x,\&startAddress);\\
\ \ displacements[0]\ =\ 0;\\
\ \ MPI\underline\ Address(\&p.y,\&address);\\
\ \ displacements[1]\ =\ address\ -{}\ startAddress;\\
\ \\
\ \ MPI\underline\ Type\underline\ struct(2,blockLength,displacements,dataTypes,\&typePoint);\\
\ \ MPI\underline\ Type\underline\ commit(\&typePoint);\\
\ \\
\ \ delete\ []\ displacements;\\
\ \ delete\ []\ dataTypes;\\
\ \ delete\ []\ blockLength;\\
\ \\
\ \ \textsl{//\ create\ the\ datatype\ for\ the\ site\ structure}\\
\ \ displacements\ =\ new\ MPI\underline\ Aint[3];\\
\ \ dataTypes\ =\ new\ MPI\underline\ Datatype[3];\\
\ \ blockLength\ =\ new\ int[3];\\
\ \\
\ \ blockLength[0]\ =\ 1;\\
\ \ blockLength[1]\ =\ 1;\\
\ \ blockLength[2]\ =\ 1;\\
\ \ dataTypes[0]\ =\ typePoint;\\
\ \ dataTypes[1]\ =\ MPI\underline\ INT;\\
\ \ dataTypes[2]\ =\ MPI\underline\ INT;\\
\ \\
\ \ MPI\underline\ Address(\&s.p,\&startAddress);\\
\ \ displacements[0]\ =\ 0;\\
\ \ MPI\underline\ Address(\&s.listIndex,\&address);\\
\ \ displacements[1]\ =\ address\ -{}\ startAddress;\\
\ \ MPI\underline\ Address(\&s.h,\&address);\\
\ \ displacements[2]\ =\ address\ -{}\ startAddress;\\
\ \\
\ \ MPI\underline\ Type\underline\ struct(3,blockLength,displacements,dataTypes,\&typeSite);\\
\ \ MPI\underline\ Type\underline\ commit(\&typeSite);\\
\ \\
\ \ delete\ []\ displacements;\\
\ \ delete\ []\ dataTypes;\\
\ \ delete\ []\ blockLength;\\
\ \\
\ \ \textsl{//\ create\ the\ datatype\ for\ the\ boundryEvent\ structure}\\
\ \ displacements\ =\ new\ MPI\underline\ Aint[4];\\
\ \ dataTypes\ =\ new\ MPI\underline\ Datatype[4];\\
\ \ blockLength\ =\ new\ int[4];\\
\ \\
\ \ blockLength[0]\ =\ 1;\\
\ \ blockLength[1]\ =\ 1;\\
\ \ blockLength[2]\ =\ 1;\\
\ \ blockLength[3]\ =\ 1;\\
\ \ dataTypes[0]\ =\ typeSite;\\
\ \ dataTypes[1]\ =\ typeSite;\\
\ \ dataTypes[2]\ =\ MPI\underline\ DOUBLE;\\
\ \ dataTypes[3]\ =\ MPI\underline\ INT;\\
\ \\
\ \ MPI\underline\ Address(\&m.oldSite,\&startAddress);\\
\ \ displacements[0]\ =\ 0;\\
\ \ MPI\underline\ Address(\&m.newSite,\&address);\\
\ \ displacements[1]\ =\ address\ -{}\ startAddress;\\
\ \ MPI\underline\ Address(\&m.time,\&address);\\
\ \ displacements[2]\ =\ address\ -{}\ startAddress;\\
\ \ MPI\underline\ Address(\&m.type,\&address);\\
\ \ displacements[3]\ =\ address\ -{}\ startAddress;\\
\ \\
\ \ MPI\underline\ Type\underline\ struct(4,blockLength,displacements,dataTypes,\&typeMessage);\\
\ \ MPI\underline\ Type\underline\ commit(\&typeMessage);\\
\ \\
\ \ delete\ []\ displacements;\\
\ \ delete\ []\ dataTypes;\\
\ \ delete\ []\ blockLength;\\
\ \\
\ \ \textsl{//\ attach\ the\ buffer\ to\ the\ MPI\ process}\\
\ \ MPI\underline\ Buffer\underline\ attach(malloc(BUFFER\underline\ SIZE\underline\ COUNT\ $\ast$\ sizeof(message)\ +\ MPI\underline\ BSEND\underline\ OVERHEAD),\ BUFFER\underline\ SIZE\underline\ COUNT\ $\ast$\ sizeof(message)\ +\ MPI\underline\ BSEND\underline\ OVERHEAD);\\
\ \\
\ \ \textsl{//\ get\ the\ node\ on\ my\ left}\\
\ \ left\ =\ LEFT(rank,nodeCount);\\
\ \\
\ \ \textsl{//\ get\ the\ node\ on\ my\ right}\\
\ \ right\ =\ RIGHT(rank,nodeCount);\\
\ \\
\ \ \textsl{//\ hey,\ we\ finished\ the\ init!\ \ so\ set\ the\ flag}\\
\ \ isInit\ =\ true;\\
\ \\
\ \ \textsl{//\ return\ the\ value\ of\ the\ flag\ (should\ be\ true)}\\
\ \ return(isInit);\\
\}\\
\ \\
bool\ MPIWrapper::shutdown()\ \{\\
\ \ \textsl{//\ make\ sure\ we\ had\ a\ successful\ init()\ call}\\
\ \ if(!isInit)\\
\ \ \ \ return(false);\\
\ \\
\ \ \textsl{//\ detach\ the\ buffer\ from\ the\ MPI\ process\ (COULD\ STALL\ PROGRAM\ EXECUTION}\\
\ \ \textsl{//\ SINCE\ ALL\ BUFFERED\ MESSAGES\ MUST\ BE\ DELIVERED\ BEFORE\ THE\ CALL\ EXITS)}\\
\ \ MPI\underline\ Buffer\underline\ detach(\&buffer,\&bufferSize);\\
\ \\
\ \ \textsl{//\ free\ the\ declared\ types}\\
\ \ MPI\underline\ Type\underline\ free(\&typeMessage);\\
\ \ MPI\underline\ Type\underline\ free(\&typeSite);\\
\ \ MPI\underline\ Type\underline\ free(\&typePoint);\\
\ \\
\ \ \textsl{//\ call\ the\ MPI\underline\ Finalize()\ function\ to\ make\ MPI\ clean\ up}\\
\ \ MPI\underline\ Finalize();\\
\ \\
\ \ \textsl{//\ set\ init\ to\ false\ so\ we\ don't\ do\ anything\ stupid}\\
\ \ isInit\ =\ false;\\
\ \\
\ \ \textsl{//\ return\ true\ so\ all\ is\ well}\\
\ \ return(!isInit);\\
\}\\
\ \\
bool\ MPIWrapper::sendMessage(message$\ast$\ m,\ Direction\ dir)\ \{\\
\ \\
\ \ \textsl{//\ send\ the\ message\ with\ a\ buffered\ send\ so\ we\ don't\ block}\\
\ \ if(DIR(dir)\ !=\ -{}1)\ \{\\
\ \ \ \ MPI\underline\ Bsend(m,\ 1,\ typeMessage,\ DIR(dir),\ TAG\underline\ MESSAGE,\ MPI\underline\ COMM\underline\ WORLD);\\
\ \ \ \ ++countSend;\\
\ \ \}\\
\ \\
\ \ \textsl{//\ return\ true}\\
\ \ return(true);\\
\}\\
\ \\
bool\ MPIWrapper::recvMessages(vector<{}message>{}$\ast$\ messages)\ \{\\
\ \\
\ \ \textsl{//\ loop\ until\ we\ don't\ have\ any\ more\ messages\ waiting}\\
\ \ while(isMessage())\ \{\\
\ \ \ \ \textsl{//\ recieve\ the\ message}\\
\ \ \ \ MPI\underline\ Recv(\&m,\ 1,\ typeMessage,\ MPI\underline\ ANY\underline\ SOURCE,\ TAG\underline\ MESSAGE,\ MPI\underline\ COMM\underline\ WORLD,\ \&status);\\
\ \ \ \ messages-{}>{}push\underline\ back(m);\\
\ \ \ \ ++countRecv;\\
\ \ \}\\
\ \\
\ \ \textsl{//\ return\ true}\\
\ \ return(true);\\
\}\\
\ \\
bool\ MPIWrapper::isMessage()\ \{\\
\ \ \textsl{//\ do\ an\ iprobe\ to\ get\ the\ value\ of\ flag\ (TRUE\ OR\ FALSE)}\\
\ \ MPI\underline\ Iprobe(MPI\underline\ ANY\underline\ SOURCE,\ TAG\underline\ MESSAGE,\ MPI\underline\ COMM\underline\ WORLD,\ \&flag,\ \&status);\\
\ \\
\ \ \textsl{//\ return\ the\ value\ compared\ to\ the\ true\ equiv\ of\ 1\ (since\ it's\ an\ int)}\\
\ \ return(flag\ ==\ 1);\\
\}\\
\ \\
bool\ MPIWrapper::sendAntiMessage(message$\ast$\ m,\ Direction\ dir)\ \{\\
\ \\
\ \ \textsl{//\ send\ the\ message\ with\ a\ buffered\ send\ so\ we\ don't\ block}\\
\ \ if(DIR(dir)\ !=\ -{}1)\ \{\\
\ \ \ \ MPI\underline\ Bsend(m,\ 1,\ typeMessage,\ DIR(dir),\ TAG\underline\ ANTI\underline\ MESSAGE,\ MPI\underline\ COMM\underline\ WORLD);\\
\ \ \ \ ++countSendAnti;\\
\ \ \}\\
\ \\
\ \ \textsl{//\ return\ true}\\
\ \ return(true);\\
\}\\
\ \\
bool\ MPIWrapper::recvAntiMessages(vector<{}message>{}$\ast$\ messages)\ \{\\
\ \\
\ \ \textsl{//\ loop\ until\ we\ don't\ have\ any\ more\ messages\ waiting}\\
\ \ while(isAntiMessage())\ \{\\
\ \ \ \ \textsl{//\ recieve\ the\ message}\\
\ \ \ \ MPI\underline\ Recv(\&m,\ 1,\ typeMessage,\ MPI\underline\ ANY\underline\ SOURCE,\ TAG\underline\ ANTI\underline\ MESSAGE,\ MPI\underline\ COMM\underline\ WORLD,\ \&status);\\
\ \ \ \ messages-{}>{}push\underline\ back(m);\\
\ \ \ \ ++countRecvAnti;\\
\ \ \}\\
\ \\
\ \ \textsl{//\ return\ true}\\
\ \ return(true);\\
\}\\
\ \\
bool\ MPIWrapper::isAntiMessage()\ \{\\
\ \ \textsl{//\ do\ an\ iprobe\ to\ get\ the\ value\ of\ flag\ (TRUE\ OR\ FALSE)}\\
\ \ MPI\underline\ Iprobe(MPI\underline\ ANY\underline\ SOURCE,\ TAG\underline\ ANTI\underline\ MESSAGE,\ MPI\underline\ COMM\underline\ WORLD,\ \&flag,\ \&status);\\
\ \\
\ \ \textsl{//\ return\ the\ value\ compared\ to\ the\ true\ equiv\ of\ 1\ (since\ it's\ an\ int)}\\
\ \ return(flag\ ==\ 1);\\
\}\\
\ \\
float\ MPIWrapper::allReduceFloat(float\ input,\ MPI\underline\ Op\ op)\ \{\\
\ \ float\ output;\\
\ \\
\ \ \textsl{//\ call\ MPI\underline\ Allreduce()\ using\ the\ provided\ input/output,\ the\ correct\ datatype}\\
\ \ \textsl{//\ and\ the\ user-{}provided\ op\ for\ the\ world\ communicator}\\
\ \ MPI\underline\ Allreduce(\&input,\&output,1,MPI\underline\ FLOAT,op,MPI\underline\ COMM\underline\ WORLD);\\
\ \\
\ \ \textsl{//\ return\ the\ output\ value}\\
\ \ return(output);\\
\}\\
\ \\
double\ MPIWrapper::allReduceDouble(double\ input,\ MPI\underline\ Op\ op)\ \{\\
\ \ double\ output;\\
\ \\
\ \ \textsl{//\ call\ MPI\underline\ Allreduce()\ using\ the\ provided\ input/output,\ the\ correct\ datatype}\\
\ \ \textsl{//\ and\ the\ user-{}provided\ op\ for\ the\ world\ communicator}\\
\ \ MPI\underline\ Allreduce(\&input,\&output,1,MPI\underline\ DOUBLE,op,MPI\underline\ COMM\underline\ WORLD);\\
\ \\
\ \ \textsl{//\ return\ the\ output\ value}\\
\ \ return(output);\\
\}\\
\ \\
int\ MPIWrapper::allReduceInt(int\ input,\ MPI\underline\ Op\ op)\ \{\\
\ \ int\ output;\\
\ \\
\ \ \textsl{//\ call\ MPI\underline\ Allreduce()\ using\ the\ provided\ input/output,\ the\ correct\ datatype}\\
\ \ \textsl{//\ and\ the\ user-{}provided\ op\ for\ the\ world\ communicator}\\
\ \ MPI\underline\ Allreduce(\&input,\&output,1,MPI\underline\ INT,op,MPI\underline\ COMM\underline\ WORLD);\\
\ \\
\ \ \textsl{//\ return\ the\ output\ value}\\
\ \ return(output);\\
\}\\
\ \\
bool\ MPIWrapper::isRoot()\ \{\\
\ \ \textsl{//\ return\ the\ value\ of\ this\ compare}\\
\ \ return(rank\ ==\ ROOT\underline\ RANK);\\
\}\\
\ \\
void\ MPIWrapper::barrier()\ \{\\
\ \ MPI\underline\ Barrier(MPI\underline\ COMM\underline\ WORLD);\\
\}\\
\ \\
double\ MPIWrapper::wallTime()\ \{\\
\ \ return(MPI\underline\ Wtime());\\
\}\\
\ \\
 }
\normalfont\normalsize


\end{code}

% EXCEPTION LIST FILES
\section{exception.h}
\begin{code}{exception.h}{tw/exception.h}
{\ttfamily \raggedright \small
\#include\ <{}string>{}\\
using\ std::string;\\
\ \\
\#ifndef EXCEPTION\underline\ H\\
\#define EXCEPTION\underline\ H\\
\ \\
class\ Exception\ \{\\
public:\\
\ \ Exception(char$\ast$\ c)\ :\ error(c)\ \{\ ;\ \}\\
\ \ Exception(string\ \&s)\ :\ error(s)\ \{\ ;\ \}\\
\ \ Exception()\ :\ error("{}General\ Error"{})\ \{\ ;\ \}\\
\ \\
\ \ string\ error;\\
\ \\
\};\\
\ \\
\#endif\\
\ \\
 }
\normalfont\normalsize


\end{code}



\bibliography{bibliography}


\end{document}
